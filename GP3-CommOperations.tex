\chapter{One-sided Communication with GP Arrays}

The GP library relies on the same one-sided communication model used in Global
Arrays. However, because the data elements in a GP array can each have an
arbitrary size, the communication functions themselve are more complicated. For
a GA, the size of the local buffer needed to hold a request can be readily
calculated from the bounds of the requested array section and the size of the
array elements. If each element can have its own size, this calculation is no
longer possible. Because of this, the GP toolkit provides functions that allow
users to determine the size of a request before actually trying to access the
data. Once the size of the request is known, the user can allocate an
appropriate sized buffer for use in actually requesting the data. For GP
\emph{get} or \emph{gather} operations, the user should first determine the size
of the request, then allocate buffers that are large enough to receive the data,
and then make the actual call to \emph{get} or \emph{gather} to get the data
from the GP and store it locally.

For \emph{put} and \emph{scatter}, it is not necessary to determine the total
size of the request since presumably the data already exists locally and there
is no need to allocate additional buffers to receive it. Some buffers may need to
be created to hold pointers to local data, but since pointers are of uniform size,
this is easily done based on the number of elements in the request.

Because communication is one-sided, only the processor requesting the data
transfer needs to participate. Remote processors that may all or part of the
data transfered are not required to participate in any way.

\section{Accessing Remote Data Using Put and Get}
Remote data can be accessed in the GP toolkit using either \emph{get} or
\emph{gather} to transfer data from a GP array into a local buffer. In the case
of GP\_Get, the first step is to determine how large the requested data is using
the function

\begin{verbatim}
      void GP_Get_size(int g_p, int *lo, int *hi, int *size);
\end{verbatim}

\noindent
The arrays lo[] and hi[] are the bounding indices of the request and size is the
returned value representing the total amount of data, in bytes, of the requested
block of elements.

Using the information obtained from the function GP\_Get\_size, the user can
allocated a local buffer of size `size'. This buffer can then be used to store
the data. The full \emph{get} operation has the form

\begin{verbatim}
      void GP_Get(int g_p, int *lo, int *hi,
                  void *buf, void **buf_ptr, int *ld,
                  void *buf_size, int *ld_sz, int *size,
                  int setbuf);
\end{verbatim}

\noindent
This function is quite complicated and has several parts. The arrays lo[] and hi[]
contain the bounding indices of the block of elements that are being requested.
The buffers in this call can be used in two ways, depending on the value of the
setbuf parameter. If setbuf is 0, then the buffer represented by buf is assumed
to be a valid pointer to a memory segment that was allocated to receive all
incoming data. The size required for this buffer can be determined using the
GP\_Get\_size call. After the call
to GP\_Get is complete, all the data from requested elements will be in buf.
However, it will still be difficult to locate the data for each
individual element within buf. Pointers to individual array elements within buf
are contained in the array of pointers, buf\_ptr. The stride array ld[] is input
that describes how buf\_ptr is laid out locally in memory. There are ndim-1
entries in ld[] and they run from the slowest stride to the fastest.  Similarly,
the array buf\_size
contains integers that describe the size of each array element in buf\_ptr. The
array buf\_size is a void pointer instead of an int, because depending on
whether the underlying GA and GP libraries are built with 4 or 8 byte integers,
the size of the integers buf\_size will change. Users should cast to the correct
size pointer before accessing the contents. The stride array ld\_sz[] is input
that describes how buf\_size is laid out in memory.  This operation also returns
the total size of the data request, which may be redundant if a call to
GP\_Get\_size has already been made.

If setbuf is not 0, then this call assumes that buf\_ptr buf\_size contain valid
pointers and sizes that can be used to hold incoming data. The buf pointer is
ignored for this case. Assuming the buf\_ptr and buf\_size already point to
valid memory locations and contain the correct sizes can lead to performance
gains. This situation can occur if multiple GP\_Get calls are being made to a GP
array or if part of one GP array is being copied into a part of another GP
array.

A call is also available that allows users to put data back into the GP array.
This call has the form

\begin{verbatim}
      void GP_Put(int g_p, int *lo, int *hi, void **buf_ptr,
                  int *ld, void *buf_size, int *ld_sz,
                  int *size, int checksize);
\end{verbatim}

\noindent
Note that the signature for this call differs from that of the GP\_Get call,
unlike the corresponding calls in GA, which are completely symmetrical. The
arrays lo[] and hi[] again refer to the bounding indices of the block of array
elements that are to be over written, buf\_ptr is an array of pointers
containing pointers to the local memory segments containing the data to be
written to the GP array, ld[] is an integer array of strides for buf\_ptr. The
array buf\_size contains integers representing the size of the of the array
elements. The type of integer should match the size of integers specified when
building GA and GP (4 or 8 bytes). ld\_sz is the corresponding stride array for
buf\_size and size is a return value representing the total size of the data
sent. Because the data already resides on the local process, which presumably
knows how big each data element is, there is no need to use the size information
stored in the GP array to arrange the data transfer. However,
for debugging purposes it may be useful to compare it with size information in
the GP. The logical variable checksize allows users to do this. If it is set to
true, then the GP\_Put call will do a comparison between the data in the
buf\_size array and the data stored in the GP array, otherwise it will skip this
check. Note that the comparison requires extra data transfers and memory
allocations, so for working code this should be turned off.

\section{Scatter and Gather}

The \emph{gather} and \emph{scatter} functionality in GP are similar to
\emph{get} and \emph{put} except that instead of moving blocks of GP data array,
these calls move lists of random GP elements back and forth between local
buffers and the GP array. Otherwise, the use and structure of these calls is
very similar to \emph{get} and \emph{put}. If gathering a random collection of
array elements into a local buffer, it is convenient to first find out how large
the request is by making a call to the function

\begin{verbatim}
      void GP_Gather_size(int g_p, int nv, int *subscript,
                          int *size);
\end{verbatim}

\noindent
The number of elements that are requested is nv and the subscripts of each
element are stored in the array subscript. The dimension of subscript must be at
least ndim*nv. The subscripts are stored in the order

\begin{verbatim}
0,...,ndim-1,ndim,...,2*ndim-1,.......,(nv-1)*ndim,...,nv*ndim-1
\end{verbatim}

\noindent
Upon return, the variable size contains the size in bytes of the data request.
Once the size of the request is known, a local void* buffer can be allocated and
the actual data can be copied to it using the function

\begin{verbatim}
      void GP_Gather(int g_p, int nv, int *subscript, void *buf,
                     void **buf_ptr, int *buf_size, int *size,
                     int setbuf);
\end{verbatim}

\noindent
The variable buf is a pointer to the contiguous memory segment that will hold
the entire request, buf\_ptr is an array of pointers to individual array
elements, buf\_size is an array of integers that gives the size of each array
element and size returns the size of the request.

The parameter setbuf works in a similar way to the GP\_Get call. If setbuf is 0,
then buf is assumed to be a valid pointer and no useful information is in
buf\_ptr and buf\_size. These buffers are returned with valid pointers and sizes
when the call completes. If setbuf is not 0, then buf\_ptr and buf\_size are
assumed to already contain valid pointers and the correct sizes and buf is
ignored.

Data can be written to random locations in a GP array with the call

\begin{verbatim}
      void GP_Scatter(int g_p, int nv, int *subscript,
                      void **buf_ptr, int *buf_size, int *size,
                      int checksize)
\end{verbatim}

\noindent
The arguments are the same as for the GP\_Scatter call. The void* buf argument
is dropped and the checksize option has been added. As in the case of GP\_Put,
this argument can be used to verify that the sizes in buf\_size match the
information in the GP array.

\section{Data Consistency}

Because the one-sided communication model used in both the GA and GP libraries
allows processes to arbitrarily read and write to memory locations on other
processors without any active participation by the remote processor, data
consistency must be managed explicitly by the user. Unlike message-passing,
where both the sending and recieving processors must reach well-defined points
in the code before the data transfer can take place, there is no \emph{a priori}
guarantee of ordering of messages in one-sided communication that allows users
to know when a data transfer has completed. This can cause data race conditions
if a program is trying to read and write to remote locations while other
processors are simultaneously trying to modify data. To prevent this from
happening, several strategies can be employed when writing code using one-sided
communication.

\begin{itemize}

\item The primary mechanism for guaranteeing data consistency is to call a GA or
GP synchronization function. This forces all processes to reach the same point
in the algorithm before proceeding further and also guarantees that all
outstanding communication has been flushed from the system.

\item Divide calculations into phases, where each phase represents a different
communication pattern. For example, during one phase a program may read from a
GP array and modify the data in local buffers. Because nobody is modifying the
GP array, all users are guaranteed that the GP array is in a known, consistent
state. Data consistency between phases can be guaranteed using synchronization.

\item Do not simultaneously read and write to the same GP array without an
intervening synchronization call unless each process is simultaneously reading
and writing to the \emph{same} set of elements and each set of elements is being
read and overwritten by at most \emph{one} process. If it is necessary to do a
read/modify/write then separate GP arrays should be used for the data source and
destinations.

\item Design algorithms to write to GP arrays in non-overlapping patches so that
no two processors are simultaneously trying to write to the same array location
during the same communication phase.

\end{itemize}

The synchronization call in the GP toolkit is the same function as in the GA
toolkit and they can be used interchangeably. The function has the form

\begin{verbatim}
      void GP_Sync();
\end{verbatim}

\noindent
There are no arguments to this function. It must be called by all processors,
otherwise the program will hang.

\section{Direct Data Access}

Direct data access does not represent communication as such, but it does fall
under the general category of accessing an modifying data so it is included
here. The data stored in GP arrays can be accessed directly without
communication if it is local to the process that wants to modify it. One way to
do this is to store the pointer to the data that is returned from the original
GP\_Malloc call, but this may be inconvenient if the data is modified in a
different subroutine or library than the one that created it. A better way to
access the data is to retrieve a pointer to it directly from the GP array. A
pointer to any local data element can be recovered using the function

\begin{verbatim}
      void GP_Access_element(int g_p, int *subscript, void *ptr,
                             int *size);
\end{verbatim}

\noindent
This function returns a pointer to the element corresponding to the data element
at the location in the array specified in the array subscript. It also returns
the size of the data element in the variable size. The values in subscript must
lie within the array bounds specified by a call to the GP\_Distribution
function. The contents of ptr can then be read or modified by the user. If the
contents are only be read, then access to the element can be released by calling
the function

\begin{verbatim}
      void GP_Release_element(int g_p, int *subscript);
\end{verbatim}

\noindent
If the data is modified, then access to the element can be released by calling
the function

\begin{verbatim}
      void GP_Release_update_element(int g_p, int *subscript);
\end{verbatim}

\noindent
Note that the \emph{access} and \emph{release} functions are purely local and
have no effect on data residing on other processors.

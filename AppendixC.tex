\chapter{Global Arrays on Older Systems}

\label{Appendix_C}

Global Arrays supports many computing platforms.  This appendix discusses those
platforms, including older systems.

The web page \url{www.emsl.pnl.gov/docs/global/support.html} contains updated
information about using GA on different platforms. Please refer to this page
frequently for most recent updates and platform information. 

\section{Platform and Library Dependencies }

The following platforms are supported by Global Arrays. 


\section{Supported Platforms}
\begin{itemize}
\item IBM SP, CRAY T3E/J90/SV1, SGI Origin, Fujitsu VX/VPP, Hitachi 
\item Cluster of workstations: Solaris, IRIX, AIX, HPUX, Digital/Tru64 Unix,
Linux, NT 
\item Standalone uni- or multi-processor workstations or servers 
\item Standalone uni- or multi-processor Windows NT workstations or servers
\end{itemize}
Older versions of GA supported some additional (now obsolete) platforms
such as: IPSC, KSR, PARAGON, DELTA, CONVEX. They are not supported
in the newer (>3.1) versions because we do not have access to these
systems. We recommend using GA 2.4 on these platforms.

For most of the platforms, there are two versions available: 32-bit
and 64-bit. This table specifies valid TARGET names for various supported
platforms. 

\begin{longtable}{|>{\centering}p{3cm}|c|>{\centering}p{3cm}|>{\centering}p{3cm}|}
\hline 
Platform & 32-bit TARGET name & 64-bit TARGET name & Remarks\tabularnewline
\hline
\endhead
\hline 
Sun Ultra & SOLARIS & SOLARIS64 & 64-bit version added in GA 3.1\tabularnewline
\hline 
IBM BlueGene/P &  & BGP & supported in GA 4.1 and later (Contact your BlueGene sys admin for
GA instalation). More info in support page...\tabularnewline
\hline 
IBM BlueGene/L &  & BGL & added in GA 4.0.2 (Contact your BlueGene sys admin for GA instalation).
More info in support page...\tabularnewline
\hline 
Cray XT3/XT4 &  & LINUX64 (or) CATAMOUNT & TARGET based on the OS in compute Nodes (Catamount/Linux). More info
and sample settings in support page...\tabularnewline
\hline 
IBM RS/6000 & IBM & IBM64 & 64-bit version added in GA 3.1\tabularnewline
\hline 
IBM SP & LAPI & LAPI64 & no support yet for user-space communication in the 64-bit mode by
IBM\tabularnewline
\hline 
Compaq/DEC alpha & not available & DECOSF & \tabularnewline
\hline 
HP pa-risc & HPUX & HPUX64 & 64-bit version added in GA 3.1\tabularnewline
\hline 
Linux (32-bit): x86, ultra, powerpc & LINUX & not available & \tabularnewline
\hline 
Linux (64-bit): ia64 (Itanium), x86\_64 (Opteron), ppc64, etc & not available & LINUX64 & \tabularnewline
\hline 
Linux alpha & not available & LINUX64 & 64-bit version added in GA 3.1; Compaq compilers rather than GNU required\tabularnewline
\hline 
Cray T3E & not available & CRAY-T3E & \tabularnewline
\hline 
Cray J90 & not available & CRAY-YMP & \tabularnewline
\hline 
Cray SV1 & not available & CRAY-SV1 & \tabularnewline
\hline 
Cray X1 & not available & CRAY-SV2 & In X1, by default, TARGET is defined by the operating system as cray-sv2\tabularnewline
\hline 
SGI IRIX mips & SGI\_N32, SGI & SGIFP & \tabularnewline
\hline 
Hitachi SR8000 & HITACHI & not available & \tabularnewline
\hline 
Fujitsu VPP systems & FUJITSU-VPP & FUJITSU-VPP64 & 64-bit version added in GA 3.\tabularnewline
\hline 
NEC SX series &  & NEC & \tabularnewline
\hline 
Apple & MACX & MACX64 & Running MAC X or higher\tabularnewline
\hline
\end{longtable}

To aid development of fully portable applications, in 64-bit mode
Fortran integer datatype is 64-bits. It is motivated by 1) the need
of applications to use very large data structures and 2) Fortran INTEGER{*}8
not being fully portable. The 64-bit representation of integer datatype
is accomplished by using the appropriate Fortran compiler flag.

Because of limited interest in heterogenous computing among known
GA users, the Global Arrays library \emph{still does not support heterogeonous
platforms}. This capability can be added if required by new applications. 


\section{Selection of the communication network for ARMCI }

Some cluster installations can be equipped with a high performance
network which offer instead, or in addition to TCP/IP some special
communication protocol, for example GM on Myrinet network. To achieve
high performance in Global Arrays, \href{http://www.emsl.pnl.gov/docs/parsoft/armci}{ARMCI}
must be built to use these protocols in its implementation of one-sided
communication. Starting with GA 3.1, this is accomplished by setting
an environment variable ARMCI\_NETWORK to specify the protocol to
be used. In addition, the it might be necessary to provide location
for the header files and library path corresponding to location of
s/w supporting the appropriate protocol API, see g/armci/config/makecoms.h
for details. 

\begin{tabular}{|c|c|>{\centering}p{3cm}|>{\centering}p{3cm}|}
\hline 
Network & Protocol Name & ARMCI\_NETWORK Setting & Supported Platforms\tabularnewline
\hline
\hline 
Ethernet & TCP/IP & SOCKETS (optional/default) & workstation clusters (32 and 64-bit)\tabularnewline
\hline 
Quadrics/QsNet & Elan3/Shmem & QUADRICS or ELAN3 & Linux (alpha,x86,IA64,..), Compaq \tabularnewline
\hline 
Quadrics/QsNet II & Elan4 & ELAN4 & Linux (32 and 64-bit) \tabularnewline
\hline 
Infiniband & OpenIB & OPENIB & Linux (32 and 64-bit). NOTE: This network is supported in GA versions>=4.1.
For more info see the \href{http://www.emsl.pnl.gov/docs/global/support.html}{Support}
page...\tabularnewline
\hline 
Infiniband & VAPI & MELLANOX & Linux (32 and 64-bit)\tabularnewline
\hline 
Myrinet & GM & GM & Linux (x86,ultra,IA64)\tabularnewline
\hline 
Giganet cLAN & VIA & VIA & Linux (32 and 64-bit)\tabularnewline
\hline 
 & MPI & MPI-SPAWN & Supported in GA 4.1 or higher. This network setting can be used on
any platform that has MPI-2 dynamic process management support. Using
this setting is recommended only if your network is not listed above.\tabularnewline
\hline
\end{tabular}

\emph{Other Platforms:} (More settings info for these platforms in
the\href{http://www.emsl.pnl.gov/docs/global/support.html}{Support}
page) 

\begin{tabular}{|c|c|c|}
\hline 
Platforms & Protocol Name & ARMCI\_NETWORK Setting\tabularnewline
\hline
\hline 
IBM BG/L & BGML & BGMLMPI \tabularnewline
\hline 
Cray XT3/XT4  & Shmem Portals  & CRAY-SHMEM PORTALS 2.1.3\tabularnewline
\hline
\end{tabular}


\section{Selection of the message-passing library }

As explained in Section \ref{sec:Initialization-and-Termination},
GA works with either MPI or TCGMSG message-passing libraries. That
means that GA applications can use either of these interfaces. Selection
of the message-passing library takes place when GA is built. Since
the TCGMSG library is small and compiles fast, it is included with
the GA distribution package and built on Unix workstations by default
so that the package can be built as fast and as convenientlly to the
user as possible. There are three possible configurations for running
GA with the message-passing libraries:
\begin{enumerate}
\item \textcolor{black}{\underbar{GA with MPI}} (\emph{recommended}): directly
with MPI. In this mode, GA program should contain MPI initialization
calls. 
\item \textcolor{black}{\underbar{GA with TCGMSG-MPI}} (MPI and TCGMSG emulation
library): TCGMSG-MPI implements functionality of TCGMSG using MPI.
In this mode, the message passing library is initialized using a TCGMSG
PBEGIN(F) call which internally references MPI\_Initialize. To enable
this mode, define the environmental variable USE\_MPI. 
\item \textcolor{black}{\underbar{GA with TCGMSG}}: directly with TCGMSG.
In this mode, GA program should contain TCGMSG initialization calls.
\end{enumerate}
For the MPI versions, the optional environmental variables MPI\_LIB
and MPI\_INCLUDE are used to point to the location of the MPI library
and include directories if they are not in the standard system location(s).
GA programs are started with the mechanism that any other MPI programs
use on the given platform.

The recent versions of MPICH (an MPI implementation from ANL/Mississippi
State) keep the MPI header files in more than one directory and provide
compiler wrappers that implicitly point to the appropriate header
files. One can :
\begin{itemize}
\item use MPI\_INCLUDE by expanding the string with another directory component
prefixed with \textquotedbl{}-I\textquotedbl{} (you are passing include
directory names as a part of compiler flags), or (starting with GA
3.1) separated by comma \textquotedbl{},\textquotedbl{} and withot
the prefix, OR 
\item use MPI aware compiler wrappers e.g., mpicc and mpif77 to build GA
right out of the box on UNIX workstations: \end{itemize}
\begin{verbatim}
make~FC=mpif77~CC=mpicc~
\end{verbatim}
One disadvantage of the second approach it that GA makefile in some
circumstances might be not able to determine which compiler (e.g.,
GNU or PGI) is called underneath by the MPICH compiler wrappers. Since
different compilers provide different Fortran/C interface, the package
might fail to build. This problem is most likely to occur on non-Linux
Unix systems with non-native compilers (e.g., gcc).

On Windows NT, the current version of GA was tested with WMPI, an
NT implementation derived from MPICH in Portugal. 


\section{Dependencies on other software }

In addition to the message-passing library, GA requires:
\begin{itemize}
\item \href{http://www.emsl.pnl.gov/docs/parsoft/ma/MAapi.html}{MA}MA (Memory
Allocator), a library for managment of local memory; 
\item \href{http://www.emsl.pnl.gov/docs/parsoft/armci}{ARMCI}, a one-sided
communication library that GA uses as its run-time system; 
\item BLAS library is required for the eigensolver and ga\_dgemm; 
\item LAPACK library is required for the eigensolver (a subset is included
with GA, which is built into \emph{liblinalg.a});
\end{itemize}
GA may also depend on other software depending on the functions being
used.
\begin{itemize}
\item GA eigensolver, ga\_diag, is a wrapper for the eigensolver from the
PEIGS library; (Please contact \href{mailto:fanngi@ornl.gov}{George Fann}about
PEIGS) 
\item SCALAPACK, PBBLAS, and BLACS libraries are required for \emph{ga\_lu\_solve,
ga\_cholesky, ga\_llt\_solve, ga\_spd\_invert, ga\_solve}. If these
libraries are not installed, the named operations will not be available. 
\item If one would like to generate trace information for GA calls, an additional
library \emph{libtrace.a} is required, and the -DGA\_TRACE define
flag should be specified for C and Fortran compilers.
\end{itemize}

\section{Writing GA Programs }

C programs that use Global Arrays should include files \emph{`global.h',
'ga.h', `macdecls.h'}. Fortran programs should include the files \emph{`mafdecls.fh',
`global.fh'}. Fortran source must be preprocessed as a part of compilation.

The GA program should look like:
\begin{itemize}
\item When GA runs with MPI\end{itemize}
\begin{verbatim}
\textcolor{blue}{Fortran~}~~~~~~~~~~~~~~\textcolor{green}{C}

\textcolor{blue}{call~mpi\_init(..)~}~~~\textcolor{green}{MPI\_Init(..)~}~~~!~start~MPI~

\textcolor{blue}{call~ga\_initialize()}~\textcolor{green}{GA\_Initialize()~}!~start~global~arrays~

\textcolor{blue}{status~=~ma\_init(..)}~\textcolor{green}{MA\_Init(..)~}~~~~!~start~memory~allocator

\textcolor{blue}{....~do~work~~}~~~~~~~\textcolor{green}{....~do~work}

\textcolor{blue}{call~ga\_terminate()}~~\textcolor{green}{GA\_Terminate()~}~!~tidy~up~global~arrays~call

\textcolor{blue}{mpi\_finalize()}~~~~~~~\textcolor{green}{MPI\_Finalize()}~~!~tidy~up~MPI~~

\textcolor{blue}{stop}~~~~~~~~~~~~~~~~~~~~~~~~~~~~~~~~~!~exit~program\end{verbatim}
\begin{itemize}
\item When GA runs with TCGMSG or TCGMSG-MPI\end{itemize}
\begin{verbatim}
\textcolor{blue}{Fortran~C}

\textcolor{blue}{call~pbeginf()~}~~~~~~~~\textcolor{green}{PBEGIN\_(..)~}~~~~!~start~TCGMSG~

\textcolor{blue}{call~ga\_initialize()}~~~\textcolor{green}{GA\_Initialize()}~!~start~global~arrays~

\textcolor{blue}{status~=~ma\_init(..)}~~~\textcolor{green}{MA\_Init(..)}~~~~~!~start~memory~allocator

\textcolor{blue}{....~do~work~}~~~~~~~~~~~\textcolor{green}{....~do~work}

\textcolor{blue}{call~ga\_terminate()}~~~~\textcolor{green}{GA\_Terminate()~}~!~tidy~up~global~arrays~

\textcolor{blue}{call~pend()}~~~~~~~~~~~~\textcolor{green}{PEND\_()}~~~~~~~~~!~tidy~up~tcgmsg~

\textcolor{blue}{stop~}~~~~~~~~~~~~~~~~~~~~~~~~~~~~~~~~~~!~exit~program~
\end{verbatim}
The \emph{ma\_init }call looks like :
\begin{verbatim}
status~=~ma\_init(type,~stack\_size,~heap\_size)
\end{verbatim}
and it basically just goes to the OS and gets \emph{stack\_size+heap\_size}
elements of size type. The amount of memory MA allocates need to be
sufficient for storing global arrays on some platforms. Please refer
to section \ref{sub:Memory-Allocation} for the details and information
on more advanced usage of MA in GA programs. 


\section{Building GA }

Use\emph{ GNU make} to build the GA library and application programs
on Unix and Microsoft nmake on Windows. The structure of the available
makefiles are
\begin{itemize}
\item GNUmakefile: Unix makefile 
\item MakeFile: Windows NT makefile 
\item config/makefile.h: definitions \& include symbols
\end{itemize}
The user must specify TARGET as an environment variable (setenv TARGET
TARGET\_name) or in the GNUmakefile or on the command line when calling
make. For example: 

(for IBM/SP platform)
\begin{verbatim}
setenv~TARGET~LAPI~
\end{verbatim}
(or) from the command line, 
\begin{verbatim}
gmake~TARGET=LAPI
\end{verbatim}
Valid TARGET\_name for various supported platforms can be found in
the above table. Valid TARGETs can also be listed by calling make
in the top level distribution directory on UNIX family of systems
when TARGET is not defined. On Windows, WIN32, CYGNUS and INTERIX
(previously known as OpenNT) are supported. 


\paragraph{Compiler Settings (optional): }

For various supported platforms, the default compilers and compiler
options are specified in config/makefile.h. One could change the predefined
default compilers and compiler flags in GA package either by specifying
them on the command line or in the file config/makefile.h. Note: editing
config/makefile.h for any platform requires extra care and is intended
for intermediate/advanced users.
\begin{itemize}
\item CC - name of the C compiler (e.g., gcc, cc, or ccc ) 
\item FC - name of the Fortran compiler (e.g., g77, f90, mpif77 or fort) 
\item COPT - optimization or debug flags for the C compiler (e.g., -g, -O3) 
\item FOPT - optimization or debug flags for the Fortran compiler (e.g.,
-g, -O1)
\end{itemize}
For example,
\begin{verbatim}
gmake~FC=f90~CC=cc~FOPT=-O4~COPT=-g
\end{verbatim}
Note that GA provides only Fortran-77 interfaces. To use and compile
with a Fortran 90 compiler, it has to support a subset of Fortran-77. 


\subsection{Unix Environment }

As mentioned in an earlier section, there are three possible configurations
for building GA.
\begin{enumerate}
\item \textcolor{black}{\underbar{GA with MPI}} (recommended): To build
GA directly with MPI, the user needs to define environmental variables
MPI\_LIB and MPI\_INCLUDE which should point to the location of the
MPI library and include directories. Additionally, the make/environmental
variable MSG\_COMMS must be defined as MSG\_COMMS = MPI. (In csh/tcsh,
setenv MSG\_COMMS MPI)
\item \textcolor{black}{\underbar{GA with TCGMSG-MPI}}: To build GA with
the TCGMSG-MPI, user needs to define environmental variables USE\_MPI,
MPI\_LIB and MPI\_INCLUDE which should point to the location of the
MPI library and include directories.


\emph{Example}: using csh/tcsh (assume using MPICH installed in /usr/local
on IBM workstation)
\begin{verbatim}
setenv~USE\_MPI~y~

setenv~MPI\_LOC~/usr/local/mpich~

setenv~MPI\_LIB~\$MPI\_LOC/lib/rs6000/ch\_shmem~

setenv~MPI\_INCLUDE~\$MPI\_LOC/include
\end{verbatim}
\item \textcolor{black}{\underbar{GA with TCGMSG}}: To build GA directly
with TCGMSG, the user must define the environmental variable MSG\_COMMS=TCGMSG.
Note: When MSG\_COMMS=TCGMSG, make sure to unset the environment variable
USE\_MPI (e.g. unsetenv USE\_MPI).
\end{enumerate}
After chosing the configuration, to build the GA library, type 
\begin{verbatim}
make~
\end{verbatim}
or 
\begin{verbatim}
gmake
\end{verbatim}
If the build is successful, a test program test.x will be created
in global/testing directory. Refer to the Section \textquotedbl{}Running
GA programs\textquotedbl{} on how to run this test.

To build an application based on GA located in g/global/testing, for
example, the application's name is app.c (or app.F, app.f), type 
\begin{verbatim}
make~app.x~
\end{verbatim}
or 
\begin{verbatim}
gmake~app.x
\end{verbatim}
Please refer to compiler flags in file g/config/makefile.h to make
sure that Fortran and C compiler flags are consistent with flags uses
to compile your application. This may be critical when Fortran compiler
flags are used to change the default length of the integer datatype.


\paragraph{Interface to ScaLAPACK}

GA interface routines to ScaLAPACK are only available, when GA is
build with MPI and ScaLAPACK. Before building GA, the user is required
to define the environment variables USE\_SCALAPACK or USE\_SCALAPACK\_I8
(for scalapack libraries compiled with 8-byte integers), and the location
of ScaLAPACK \& Co. libraries in the env variable SCALAPACK.

\emph{Example}: using csh/tcsh
\begin{verbatim}
setenv~USE\_SCALAPACK~y~(or)~setenv~USE\_SCALAPACK\_I8~y~

setenv~SCALAPACK~'-L/msrc/proj/scalapack/LIB/rs6000~

~~~~~~~-lscalapack~-lpblas~-ltools~-lblacsF77cinit~-lblacs'~

setenv~USE\_MPI~y
\end{verbatim}
Since there are certain interdependencies between blacs and blacsF77cinit,
some system might require specification of -lblacs twice to fix the
unresolved external symbols from these libs.


\paragraph{Installing GA C++ Bindings}

By default, GA C++ bindings are not built. GA++ is built only if GA\_C\_CORE
is defined as follows: 
\begin{verbatim}
setenv~GA\_C\_CORE~y~

cd~GA\_HOME~

make~clean;~make~
\end{verbatim}
(This will build GA with C core and C++ binding).


\paragraph{Using GA\_C\_CORE}

GA's internal core is implemented using Fortran and C. When GA\_C\_CORE
is set, core Fortran functionalites are replaced by their C counterparts
to eliminate the hassle involved in mixing Fortran and C with C++
bindings on certain platforms or for some compilers (like, missing
Fortran symols/libraries during the linking phase). NOTE: C and C++
compilers should be from the same family. GA\_C\_CORE doesnot support
mixing C and C++ compilers (e.g.using Intel compiler for C and GNU
compiler for C++). 
\begin{verbatim}
make~FC=ifort~CC=icc~CXX=g++~(not~supported~if~GA\_C\_CORE~is~set)~

make~FC=ifort~CC=icc~CXX=icpc~(Intel~compiler~family~-~supported)
\end{verbatim}

\subsection{Windows NT }

To buid GA on Windows NT, MS Power Fortran 4 or DEC Visual Fortran
5 or later, and MS Visual C 4 or later are needed. Other compilers
might need the default compilation flags modified. When commercial
Windows compilers are not available, one can choose to use CYGNUS
or INTERIX and build it as any other Unix box using GNU compilers.

First of all, one needs to set environment variables (same as in Unix
enviroment). GA needs to know where find the MPI include files and
libraries. To do this, select the Environment tab under the Control
Panel, then set the variables to point to the location of MPI, for
example for WMPI on disk D:
\begin{verbatim}
set~MPI\_INCLUDE~as~d:\textbackslash{}Wmpi\textbackslash{}Include~

set~MPI\_LIB~as~d:\textbackslash{}Wmpi\textbackslash{}Console
\end{verbatim}
Make sure that the dynamic link libraries required by the particular
implementation of MPI are copied to the appropriate location for the
system DLLs. For WMPI, copy VWMPI.dll to \textbackslash{}winnt.

In the top directory do,
\begin{verbatim}
nmake
\end{verbatim}
The GA test.exe program can be built in the g\textbackslash{}global\textbackslash{}testing
directory:
\begin{verbatim}
nmake~test.exe
\end{verbatim}
In addition, the HPVM package from UCSD offers the GA interface in
the NT/Myrinet cluster environment.

GA could be built on Windows 95/98. However, due to the DOS shell
limitations, the top level NTmakefile will not work. Therefore, each
library has to be made separately in its own directory. The environment
variables referring to MPI can be hardcoded in the NT makefiles.


\subsection{Writing and building new GA programs}

For small programs contained in a single file, the most convenient
approach is to put your program file into the g/global/testing directory.
\emph{The existing GNU make suffix rules would build an executable
with the \textquotedbl{}.x\textquotedbl{} suffix from any C or Fortran
source file}. You do not have to modify makefiles in g/global/testing
at all. For example, if your program is contained in myfile.c or myfile.F
and you place it in that directory, all you need to do to create an
executable called myfile.x is to type: 
\begin{verbatim}
make~myfile.x
\end{verbatim}
Windows nmake is not as powerful as GNU make - you would need to modify
the NT makefile.

This approach obviously is not feasible for large packages that contain
multiple source files and directories. In that case you need to provide
apropriate definitions in your makefile:
\begin{itemize}
\item to header files located in the include directory, g/include, where
all public header files are copied in the process of building GA 
\item add references to libglobal.a (Unix) global.lib (Windows) and libma.a
(Unix) ma.lib (Windows) in g/lib/\$(TARGET) and for the message-passing
libraries 
\item follow compilation flags for the GA test programs in GNU and Windows
makefiles g/config/makefile.h. The recommended approach is to include
g/config/makefile.h in your makefile.
\end{itemize}
Starting with GA 3.1, one could simplify linking of applications by
including g/armci/config/makecoms.h and g/armci/config/makemp.h that
define all the necessary platform specific libraries that are required
by GA. 2.4 


\section{Running GA Programs }

Assume the GA program app.x had already been built. To run it,

\textbf{\textcolor{black}{Running on shared memory systems and clusters}}\emph{:
}(i.e., network of workstations/linux clusters)

If the app.x is built based on MPI, run the program the same way as
any other MPI programs. 

\emph{Example}: to run on four processes on clusters, use 
\begin{verbatim}
mpirun~-np~4~app.x
\end{verbatim}
\emph{Example}: If you are using MPICH (or MPICH-like Implementations),
and mpirun requires a machinefile or hostfile, then run the GA program
same as any other MPI programs. \textit{\textcolor{black}{The only
change required is to make sure the hostnames are specified in a consecutive
manner in the machinefile}}. Not doing this will prevent SMP optimizations
and would lead to poor resource utilization.
\begin{verbatim}
mpirun~-np~4~-machinefile~machines.txt~app.x
\end{verbatim}
\emph{Contents of machines.txt}: (Let us say we have two 2-way SMP
nodes (host1 and host2, and correct formats for a 4-processor machinefile
is shown in the table below).

\begin{tabular}{|>{\centering}p{2cm}|>{\centering}p{2cm}|>{\raggedright}p{3cm}|}
\hline 
Correct & Correct & Incorrect\tabularnewline
\hline
\hline 
host1

host1

host2

host2 & host2

host2

host1

host1 & host1

host2

host1 (This is wrong, the same hosts should be specified together)

host2\tabularnewline
\hline
\end{tabular}

If app.x is built based on TCGMSG (not including, Fujitsu, Cray J90,
and Windows, because there are no native ports of TCGMSG), to execute
the program on Unix workstations/servers, one should use the 'parallel'
program (built in tcgmsg/ipcv4.0). After building the application,
a file called 'app.x.p' would also be generated (If there is not such
a file, make it: 
\begin{verbatim}
make~app.x.p
\end{verbatim}
This file can be edited to specify how many processors and tasks to
use, and how to load the executables. Make sure that 'parallel' is
accessible (you might copy it into your 'bin' directory). To execute,
type:
\begin{verbatim}
parallel~app.x~\end{verbatim}
\begin{enumerate}
\item On MPPs, such as Cray XT3/XT4, or IMB SPs, use the appropriate system
command to specify the number of processors, load and run the programs.
Example: 

\begin{itemize}
\item to run on IBM SP, run as any other parallel programs (i.e., using
\emph{poe}) 
\item to run on Cray XT3/XT4, use \emph{yod}.
\end{itemize}
\item On Microsoft NT, there is no support for TCGMSG, which means you can
only build your application based on MPI. Run the application program
the same way as any other MPI programs. For, WMPI you need to create
the .pg file. Example: \end{enumerate}
\begin{verbatim}
R:\textbackslash{}nt\textbackslash{}g\textbackslash{}global\textbackslash{}testing>~start~/b~test.exe
\end{verbatim}

\section{Building Intel Trace Analyzer (VAMPIR) Instrumented Global Arrays}


\subsection{Introduction}

The following topics are covered in this section.
\begin{itemize}
\item New functions needed for the instrumentation 
\item Build instructions 
\item Further information 
\item Known problems
\end{itemize}

\subsection{New Functions Needed for the Instrumentation}
\begin{itemize}
\item To instrument the GA three C-functions are defined (see g/ga\_vt.c): 
\item vampir\_symdef is defined to associate integer identifiers with user
defined states and activities. It handles any errors that might occur. 
\item vampir\_begin is defined to register entering a user defined state.
It uses a global counter called <vampirtrace\_level> to avoid tracing
the use of libraries within libraries. 
\item vampir\_end is defined to register leaving a user defined state.
\end{itemize}
The interfaces of these functions are defined below.
\begin{verbatim}
void~\textcolor{blue}{vampir\_symdef}~(int~id,~char~{*}state,~char~{*}activity,~

~~~~~~~~~~~~~~~~~~~~char~{*}file,~int~line);~

void~\textcolor{blue}{vampir\_begin}~(int~id,~char~{*}file,~int~line);~

void~\textcolor{blue}{vampir\_end}~(int~id,~char~{*}file,~int~line);
\end{verbatim}
In addition to these functions two functions are defined to initialise
and finalise MPI when needed. The use of MPI is required because Vampirtrace
uses it internally. The functions are
\begin{verbatim}
void~\textcolor{blue}{vampir\_init}~(int~argc,~char~{*}{*}argv,~char~{*}file,~

~~~~~~~~~~~~~~~~~~int~line);~

void~\textcolor{blue}{vampir\_finalize}~(char~{*}file,~int~line);
\end{verbatim}
If the cpp flag -DMPI is provided then these two functions will turn
into null functions. In that case the use of MPI within the GAs will
ensure that Vampirtrace will be initialised properly.

The values for <file> and <line> are substitute with \_\_FILE\_\_
and \_\_LINE\_\_ macros. On compilation the C-preprocessor replaces
these macros with the actual file name and line number. These values
are used to generate error messages if needed. These functions are
defined in the file g/ga\_vt.h.

For each of the instrumented libraries an initialisation routine must
be defined that sets the state and activity tables up.
\begin{itemize}
\item \emph{tcgmsg}: tcgmsg\_vampir\_init in g/tcgmsg/tcgmsg\_vampir.c.
This routine is called from within PBEGINF. 
\item \emph{tcgmsg-mpi}: tcgmsg\_vampir\_init in g/tcgmsg-mpi/tcgmsg\_vampir.c
called from ALT\_PBEGIN\_ in misc.c. 
\item \emph{armci}: armci\_vampir\_init in g/armci/src/armci\_vampir.c called
from ARMCI\_Init in armci.c. 
\item \emph{global}: ga\_vampir\_init in g/global/src/ga\_vampir.c called
from ga\_initialize\_ and ga\_initialize\_ltd\_ in global.armci.c
\end{itemize}

\subsection{Build Instructions }

To build GA with Vampir (now called, Intel Trace Analyzer) set the
environment variable GA\_USE\_VAMPIR.

e.g., \texttt{\textcolor{black}{setenv GA\_USE\_VAMPIR y}}

to compile the GAs including all the Vampirtrace instrumentation.
Further environment variables that are required are
\begin{verbatim}
LIBVT~:~The~name~of~the~library,~normally~-lVT~which~

~~~~~~~~~~~~~is~the~default.~

VT\_LIB~:~The~path~to~the~library,~-L<library-path>~

~~~~~~~~~~~~~e.g.~setenv~VT\_LIB~/usr/local/vampir/lib~

VT\_INCLUDE:~The~path~to~the~include~file~VT.h,~

~~~~~~~~~~~~-I<include-path>.~e.g.~setenv~VT\_INCLUDE

~~~~~~~~~~~~/usr/local/vampir/include
\end{verbatim}
On some platforms it may be necessary to set LIBMPI to -lpmpi to load
the MPI profile interfaces that vampirtrace needs.

There are no defaults for VT\_PATH and VT\_INCLUDE. Beyond this point
simply follow the GA make instructions.

\emph{Note:} that libVT.a should be loaded before mpi or pmpi otherwise
the vampirtracing will be ignored. 


\subsection{Further Information }

More information on using Intel Trace Analyzer can be found on the
Intel website at

\href{http://www.intel.com/software/products/cluster/tanalyzer/}{http://www.intel.com/software/products/cluster/tanalyzer/}

From this location Vampir and Vampirtrace can be downloaded for various
platforms including validation licenses if needed.


\subsection{Known Problems}
\begin{itemize}
\item Vampirtrace and LAM-MPI clash
\end{itemize}
In an attempt to produce traces while running with LAM-MPI the program
would always abort in MPI\_Init due to a segmentation violation. The
Pallas website does not mention LAM-MPI at all, but does explicitly
state that Vampirtrace does work with MPICH. Indeed the latter has
been confirmed in tests. Therefore it is not recommended to use the
Vampirtrace instrumentation with LAM-MPI. 

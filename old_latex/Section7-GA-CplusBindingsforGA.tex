\chapter{GA++: C++ Bindings for Global Arrays}

\section{Overview }

GA++ provides a C++ interface to global arrays (GA) libraries. The
doxygen documentation of GA++ is located here: \href{https://hpc.pnl.gov/globalarrays/api/cxxapi.shtml}{https://hpc.pnl.gov/globalarrays/api/cxxapi.shtml}.
The GA C++ bindings are a layer built directly on top of the GA C
bindings. GA++ provides new names for the C bindings of GA functions
(For example, GA\_Add\_patch() is renamed as addPatch()). 


\section{GA++ Classes }

All GA classes (GAServices, GlobalArray) are declared within the scope
of GA namespace.

\noindent \textbf{\textcolor{black}{Namespace issue: }}Although namespace is
part of the ANSI C++ standard, not all C++ compilers support namespaces
(A non-instantiable GA class is provided for implementations using
compilers without namespace). \textbf{Note}: define the variable \_GA\_USENAMESPACE\_
as 0 in ga++.h if your compiler does not support namespaces.

\begin{lstlisting}[style=cppnoframe]
namespace GA {
    class GAServices;
    class GlobalArray;
};
\end{lstlisting}

\noindent The current implementation has no derived classes (no (virtual) inheritance),
templates, or exception handling. Eventually, more object oriented
functionalities will be added, and standard library facilities will
be used without affecting the performance. 


\section{Initialization and Termination}

GA namespace has the following static functions for initialization
and termination of Global Arrays.

GA::Initialize(): Initialize Global Arrays, allocates and initializes
internal data structures in Global Arrays. This is a collective operation.

GA::Terminate(): Delete all active arrays and destroy internal data
structures. This is a collective operation.

\begin{lstlisting}[style=cpp]
namespace GA { 
    _GA_STATIC_ void Initialize(int argc, char *argv[]}, size_t limit = 0); 
    _GA_STATIC_ void Initialize(int argc, char *argv[], unsigned long heapSize, unsigned long stackSize, int type, size_t limit = 0); 
    _GA_STATIC_ void Terminate();
};
\end{lstlisting}

\noindent \emph{\underbar{Example}}
\begin{lstlisting}[style=cppnoframe]
#include <iostream.h> 
#include "ga++.h"

int main(int argc, char *argv[]) { 
    GA::Initialize(argc, argv, 0); 
    cout << "Hello World!\n";
    GA::Terminate(); 
}
\end{lstlisting}

\section{GAServices}

GAServices class has member functions that does all the global operations
(non-array operations) like Process Information (number of processes,
process id, ..), Inter-process Synchronization (sync, lock, broadcast,
reduce,..), etc,.

SERVICES Object: GA namespace has a global \textquotedbl{}SERVICES\textquotedbl{}
object (of type \textquotedbl{}GAServices\textquotedbl{}), which can
be used to invoke the non-array operations. To call the functions
(for example, sync()), we invoke them on this SERVICES object (for
example, GA::SERVICES.sync()). As this object is in the global address
space, the functions can be invoked from anywhere inside the program
(provided the ga++.h is included in that file/program). 


\section{Global Array}

GlobalArray class has member functions that perform:

{*} Array operations {*} One-sided (get/put), {*} Collective array
operations, {*} Utility operations, etc,. 

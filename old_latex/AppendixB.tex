
\section*{Appendix B}

GA\_INITIALIZE

subroutine ga\_initialize()

Allocate and initialize internal data structures for Global Arrays.
This is a collective operation. GA\_INITIALIZE\_LTD

subroutine ga\_initialize\_ltd(limit) integer limit - amount of memory
in bytes perprocess {[}input{]}

Allocate and initialize internal data structures for global arrays
and set limit for memory used in global arrays.

limit < 0 means \textquotedbl{}allow unlimited memory usage\textquotedbl{}

This is a collective operation. GA\_PGROUP\_CREATE

integer function ga\_pgroup\_create(list, size) integer size - number
of processors in group {[}input{]} integer list(size) - list of processors
in processor group {[}input{]}

This command is used to create a processor group. At present, it must
be invoked by all processors in the current default processor group.
The list of processors use the indexing scheme of the default processor
group. If the default processor group is the world group, then these
indices are the usual processor indices. This function returns a process
group handle that can be used to reference this group by other functions.

This is a collective operation on the default processor group. GA\_PGROUP\_DESTROY

logical function ga\_pgroup\_destroy(p\_handle) integer p\_handle
- processor group handle {[}input{]}

This command is used to free up a processor group handle. It returns
false if the processor group handle was not previously active.

This is a collective operation on the default processor group. GA\_PGROUP\_SET\_DEFAULT

subroutine ga\_pgroup\_set\_default(p\_handle) integer p\_handle -
processor group handle {[}input{]}

This subroutine can be used to reset the default processor group on
a collection of processors. All processors in the group referenced
by p\_handle must make a call to this subroutine. Any standard global
array call that is made resetting the default processor group will
be restricted to processors in that group. Global arrays that are
created after resetting the default processor group will only be defined
on that group and global operations such as ga\_sync or ga\_igop will
be restricted to processors in that group. The ga\_pgroup\_set\_default
call can be used to rapidly convert large applications, written with
GA, into routines that run on processor groups.

The default processor group can be overridden by using GA calls that
require an explicit group handle as one of the arguments.

This is a collective operation on the group represented by the handle
p\_handle. GA\_CREATE

logical function ga\_create(type, dim1, dim2, array\_name, chunk1,
chunk2, g\_a) character{*}({*}) array\_name - a unique character string
{[}input{]} integer type - MA type {[}input{]} integer dim1/2 - array(dim1,dim2)
as in FORTRAN {[}input{]} integer chunk1/2 - minimum size that dimensions
should be chunked up into {[}input{]} integer g\_a - handle for future
references {[}output{]}

Creates a 2-dimensional global array.

Setting chunk1=dim1 gives distribution by vertical strips (chunk2{*}columns);
setting chunk2=dim2 gives distribution by horizontal strips (chunk1{*}rows).
Actual chunks will be modified so that they are at least the size
of the minimum and each process has either zero or one chunk. Specifying
chunk1/2 as < 1 will cause that dimension to be distributed evenly.

Return value: .true. means the call was succesful.This is a collective
operation.

logical function nga\_create(type, ndim, dims, array\_name, chunk,
g\_a) character{*}({*}) array\_name - a unique character string {[}input{]}
integer type - data type (MT\_DBL,MT\_INT,MT\_DCPL) {[}input{]} integer
ndim - number of array dimensions {[}input{]} integer dims(ndim) -
array of dimensions {[}input{]} integer chunk(ndim) - array of chunks,
each element specifies minimum size that given dimensions should be
chunked up into {[}input{]} integer g\_a - integer handle for future
references {[}output{]}

Creates an ndim-dimensional array using the regular distribution model
and returns an integer handle representing the array.

The array can be distributed evenly or not evenly. The control over
the distribution is accomplished by specifying chunk (block) size
for all or some of array dimensions. For example, for a 2-dimensional
array, setting chunk(1)=dim(1) gives distribution by vertical strips
(chunk(1){*}dims(1)); setting chunk(2)=dim(2) gives distribution by
horizontal strips (chunk(2){*}dims(2)). Actual chunks will be modified
so that they are at least the size of the minimum and each process
has either zero or one chunk. Specifying chunk(i) as <1 will cause
that dimension (i-th) to be distributed evenly.

Return value: .true. means the call was succesful.This is a collective
operation.

GA\_CREATE\_CONFIG

logical function nga\_create\_config(type, ndim, dims, array\_name,
chunk, p\_handle, g\_a) character{*}({*}) array\_name - a unique character
string {[}input{]} integer type - data type (MT\_DBL,MT\_INT,MT\_DCPL)
{[}input{]} integer ndim - number of array dimensions {[}input{]}
integer dims(ndim) - array of dimensions {[}input{]} integer chunk(ndim)
- array of chunks, each element specifies minimum size that given
dimensions should be chunked up into {[}input{]} integer p\_handle
- processor group handle {[}input{]} integer g\_a - integer handle
for future references {[}output{]}

Creates an ndim-dimensional array using the regular distribution model
but with an explicitly specified processor group handle and returns
an integer handle representing the array.

This call is essentially the same as the nga\_create call except for
the processor group handle p\_handle. It can be used to create mirrored
arrays.

Return value: .true. means the call was succesful.This is a collective
operation. GA\_CREATE\_GHOSTS

logical function nga\_create\_ghosts(type, ndim, dims, width, array\_name,
chunk, g\_a) character{*}({*}) array\_name - a unique character string
{[}input{]} integer type - data type (MT\_DBL,MT\_INT,MT\_DCPL) {[}input{]}
integer ndim - number of array dimensions {[}input{]} integer dims(ndim)
- array of dimensions {[}input{]} integer width(ndim) - array of ghost
cell widths {[}input{]} integer chunk(ndim) - array of chunks, each
element specifies minimum size that given dimensions should be chunked
up into {[}input{]} integer g\_a - integer handle for future references
{[}output{]}

Creates an ndim-dimensional array with a layer of ghost cells around
the visible data on each processor using the regular distribution
model and returns an integer handle representing the array.

The array can be distributed evenly or not evenly. The control over
the distribution is accomplished by specifying chunk (block) size
for all or some of the array dimensions. For example, for a 2-dimensional
array, setting chunk(1)=dim(1) gives distribution by vertical strips
(chunk(1){*}dims(1)); setting chunk(2)=dim(2) gives distribution by
horizontal strips (chunk(2){*}dims(2)). Actual chunks will be modified
so that they are at least the size of the minimum and each process
has either zero or one chunk. Specifying chunk(i) as <1 will cause
that dimension (i-th) to be distributed evenly. The width of the ghost
cell layer in each dimension is specified using the array width().
The local data of the global array residing on each processor will
have a layer width(n) ghosts cells wide on either side of the visible
data along the dimension n.

Return value: .true. means the call was succesful.This is a collective
operation.

GA\_CREATE\_GHOSTS\_CONFIG

logical function nga\_create\_ghosts\_config(type, ndim, dims, width,
array\_name, chunk, p\_handle, g\_a) character{*}({*}) array\_name
- a unique character string {[}input{]} integer type - data type (MT\_DBL,MT\_INT,MT\_DCPL)
{[}input{]} integer ndim - number of array dimensions {[}input{]}
integer dims(ndim) - array of dimensions {[}input{]} integer width(ndim)
- array of ghost cell widths {[}input{]} integer chunk(ndim) - array
of chunks, each element specifies minimum size that given dimensions
should be chunked up into {[}input{]} integer p\_handle - processor
group handle {[}input{]} integer g\_a - integer handle for future
references {[}output{]}

Creates an ndim-dimensional array with a layer of ghost cells around
the visible data on each processor using the regular distribution
model but with an explicitly specified processor group handle and
returns an integer handle representing the array.

This call is essentially the same as the nga\_create\_ghosts call
except for the processor group handle p\_handle. It can be used to
create mirrored arrays.

Return value: .true. means the call was succesful.This is a collective
operation. GA\_CREATE\_IRREG

logical function ga\_create\_irreg(type, dim1, dim2, array\_name,
map1, nblock1, map2, nblock2, g\_a)

character{*}({*}) array\_name - a unique character string {[}input{]}
integer type - MA type {[}input{]} integer dim1/2 - array(dim1,dim2)
as in FORTRAN {[}input{]} integer nblock1 - no. of blocks first dimension
is divided into {[}input{]} integer nblock2 - no. of blocks second
dimension is divided into{[}input{]} integer map1({*}) - ilo for each
block {[}input{]} integer map2({*}) - jlo for each block {[}input{]}
integer g\_a - integer handle for future references {[}output{]}

Creates an array by following the user-specified distribution.

Return value: .true. means the call was succesful.This is a collective
operation.

logical function nga\_create\_irreg(type, ndim, dims, array\_name,
map, nblock, g\_a) character{*}({*}) array\_name - a unique character
string {[}input{]} integer type - data type (MT\_DBL,MT\_INT,MT\_DCPL)
{[}input{]} integer ndim - number of array dimensions {[}input{]}
integer dims(ndim) - array of dimensions {[}input{]} integer nblock(ndim)
- no. of blocks each dimension is divided into{[}input{]} integer
map(s) - starting index for for each block; the size s is a sum of
all elements of nblock array {[}input{]} integer g\_a - integer handle
for future references {[}output{]}

Creates an array by following the user-specified distribution and
returns integer handle representing the array.

The distribution is specified as a Cartesian product of distributions
for each dimension. For example, the following figure demonstrates
distribution of a 2-dimensional array 8x10 on 6 (or more) processors.
nblock(2)=\{3,2\}, the size of map array is s=5 and array map contains
the following elements map=\{1,3,7, 1, 6\}. The distribution is nonuniform
because, P1 and P4 get 20 elements each and processors P0,P2,P3, and
P5 only 10 elements each. 

5 

5 

P0 

P3 2 P1 

P4 4 P2 

P5 2

Return value: .true. means the call was succesful. This is a collective
operation.

GA\_CREATE\_IRREG\_CONFIG

logical function nga\_create\_irreg\_config(type, ndim, dims, array\_name,
map, nblock, p\_handle, g\_a) character{*}({*}) array\_name - a unique
character string {[}input{]} integer type - data type (MT\_DBL,MT\_INT,MT\_DCPL)
{[}input{]} integer ndim - number of array dimensions {[}input{]}
integer dims(ndim) - array of dimensions {[}input{]} integer nblock(ndim)
- no. of blocks each dimension is divided into{[}input{]} integer
map(s) - starting index for for each block; the size s is a sum of
all elements of nblock array {[}input{]} integer p\_handle - processor
group handle {[}input{]} integer g\_a - integer handle for future
references {[}output{]}

Creates an array by following the user-specified distribution but
with an explicitly specified processor group and returns an integer
handle representing the array.

This call is essentially the same as the nga\_create\_irreg call except
for the processor group handle p\_handle. It can be used to create
mirrored arrays.

This is a collective operation. GA\_CREATE\_GHOSTS\_IRREG

logical function nga\_create\_ghosts\_irreg(type, ndim, dims, width,
array\_name, map, nblock, g\_a) character{*}({*}) array\_name - a
unique character string {[}input{]} integer type - data type (MT\_DBL,MT\_INT,MT\_DCPL)
{[}input{]} integer ndim - number of array dimensions {[}input{]}
integer dims(ndim) - array of dimensions {[}input{]} integer width(ndim)
- array of ghost cell widths {[}input{]} integer nblock(ndim) - no.
of blocks each dimension is divided into{[}input{]} integer map(s)
- starting index for for each block; the size s is a sum of all elements
of nblock array {[}input{]} integer g\_a - integer handle for future
references {[}output{]}

Creates an array with ghost cells by following the user-specified
distribution and returns integer handle representing the array.

The distribution is specified as a Cartesian product of distributions
for each dimension. For example, the following figure demonstrates
distribution of a 2-dimensional array 8x10 on 6 (or more) processors.
nblock(2)=\{3,2\}, the size of map array is s=5 and array map contains
the following elements map=\{1,3,7, 1, 6\}. The distribution is nonuniform
because, P1 and P4 get 20 elements each and processors P0,P2,P3, and
P5 only 10 elements each. 

5 

5 

P0 

P3 2 P1 

P4 4 P2 

P5 2

The array width() is used to control the width of the ghost cell boundary
around the visible data on each processor. The local data of the global
array residing on each processor will have a layer width(n) ghosts
cells wide on either side of the visible data along the dimension
n.

Return value: .true. means the call was succesful. This is a collective
operation.

GA\_CREATE\_GHOSTS\_IRREG\_CONFIG

logical function nga\_create\_ghosts\_irreg\_config(type, ndim, dims,
width, array\_name,map, nblock, p\_handle, g\_a) character{*}({*})
array\_name - a unique character string {[}input{]} integer type -
data type (MT\_DBL,MT\_INT,MT\_DCPL) {[}input{]} integer ndim - number
of array dimensions {[}input{]} integer dims(ndim) - array of dimensions
{[}input{]} integer width(ndim) - array of ghost cell widths {[}input{]}
integer nblock(ndim) - no. of blocks each dimension is divided into{[}input{]}
integer map(s) - starting index for for each block; the size s is
a sum of all elements of nblock array {[}input{]} integer p\_handle
- processor group handle {[}input{]} integer g\_a - integer handle
for future references {[}output{]}

Creates an array with ghost cells by following the user-specified
distribution and an explicitly specified processor group and returns
an integer handle representing the array.

This call is essentially the same as the nga\_create\_ghosts\_irreg
call except for the processor group handle p\_handle. It can be used
to create mirrored arrays.

This is a collective operation.

GA\_CREATE\_HANDLE

integer function ga\_create\_handle()

This function returns a global array handle that can then be used
to create a new global array. This is part of a new API for creating
global arrays that is designed to replace the old interface built
around the ga\_create\_xxx calls. The sequence of operations is to
begin with a call to ga\_create\_handle to get a new array handle.
The attributes of the array, such as dimension, size, type, etc. can
then be set using successive calls to the nga\_set\_xxx subroutines.
When all array attributes have been set, the ga\_allocate subroutine
is called and the global array is actually created and memory for
it is allocated.

This is a collective operation. GA\_SET\_ARRAY\_NAME

subroutine ga\_set\_array\_name(g\_a, name) integer g\_a - global
array handle {[}input{]} character{*}({*}) name - a unique character
string {[}input{]}

This subroutine can be used to assign a unique character string name
to a global array handle that was obtained using the nga\_create\_handle
function.

This is a collective operation.

GA\_SET\_DATA

subroutine ga\_set\_data(g\_a, ndim, dims, type) integer g\_a - global
array handle {[}input{]} integer ndim - dimension of array {[}input{]}
integer dims(ndim) - array dimensions {[}input{]} integer type - data
type (MT\_DBL,MT\_INT,etc.) {[}input{]}

This subroutine can be used to set the array dimension, coordinate
dimenstions, and data type assigned to a global array handle that
was obtained using the nga\_create\_handlefunction.

This is a collectiveoperation. GA\_SET\_IRREG\_DISTR

subroutine ga\_set\_irreg\_distr(g\_a, mapc, nblock) integer g\_a
- global array handle {[}input{]} integer map(s) - starting index
for for each block; the size s is a sum of all elements of nblock
array {[}input{]} integer nblock(ndim) - no. of blocks each dimension
is divided into {[}input{]}

This subroutine can be used to partition the array data among the
individual processors for a global array handle obtained using the
nga\_create\_handle function.

The distribution is specified as a Cartesian product of distributions
for each dimension. For example, the following figure demonstrates
distribution of a 2-dimensional array 8x10 on 6 (or more) processors.
nblock(2)=\{3,2\}, the size of map array is s=5 and array map contains
the following elements map=\{1,3,7, 1, 6\}. The distribution is nonuniform
because, P1 and P4 get 20 elements each and processors P0,P2,P3, and
P5 only 10 elements each. 

5 

5 

P0 

P3 2 P1 

P4 4 P2 

P5 2

The array width() is used to control the width of the ghost cell boundary
around the visible data on each processor. The local data of the global
array residing on each processor will have a layer width(n) ghosts
cells wide on either side of the visible data along the dimension
n.

This is a collective operation. GA\_SET\_PGROUP

subroutine ga\_set\_pgroup(g\_a, p\_handle) integer g\_a - global
array handle {[}input{]} integer p\_handle - processor group handle
{[}input{]}

This subroutine can be used to set the processor configuration assigned
to a global array handle that was obtained using the ga\_create\_handle
function. It can also be used to create mirrored arrays by using the
mirrored array processor configuration in this function call. It can
also be used to create an array on a processor group by using a processor
group handle in this call.

This is a collective operation. GA\_SET\_GHOSTS

subroutine ga\_set\_ghosts(g\_a, width) integer g\_a - global array
handle {[}input{]} integer width(ndim)- array of ghost cell widths
{[}input{]}

This subroutine can be used to set the ghost cell widths for a global
array handle that was obtained using the ga\_create\_handle function.
The ghost cell widths indicate how many ghost cells are used to pad
the locally held global array data along each dimension. The padding
can be set independently for coordinate direction.

This is a collective operation. GA\_SET\_CHUNK

subroutine ga\_set\_chunk(g\_a, chunk) integer g\_a - global array
handle {[}input{]} integer chunk(ndim)- array of chunk widths {[}input{]}

This subroutine is used to set the chunk array for a global array
handle obtained using the ga\_create\_handle function. The chunk array
is used to determine the minimum number of array elements that are
assigned to each processor along each coordinate direction.

This is a collective operation GA\_SET\_BLOCK\_CYCLIC

subroutine ga\_set\_block\_cyclic(g\_a, dims) integer g\_a - global
array handle {[}input{]} integer dims(ndim)- array of block dimensions
{[}input{]}

This subroutine is used to create a global array with a simple block-cyclic
data distribution. The array is broken up into blocks of size dims
and each block is numbered sequentially using a column major indexing
scheme. The blocks are then assigned in a simple round-robin fashion
to processors. This is illustrated in the figure below for an array
containing 25 blocks distributed on 4 processors. Blocks at the edge
of the array may be smaller than the block size specified in dims.
In the example below, blocks 4,9,14,19,20,21,22,23, and 24 might be
smaller thatn the remaining blocks. Most global array operations are
insensitive to whether or not a block-cyclic data distribution is
used, although performance may be slower in some cases if the global
array is using a block-cyclic data distribution. Individual data blocks
can be accessesed using the block-cyclic access functions.

0 P0 5 P1 10 P2 15 P3 20 P0 1 P1 6 P2 11 P3 16 P0 21 P1 2 P2 7 P3
12 P0 17 P1 22 P2 3 P3 8 P0 13 P1 18 P2 23 P3 4 P0 9 P1 14 P2 19 P3
24 P4

This is a collective operation. GA\_SET\_BLOCK\_CYCLIC\_PROC\_GRID

subroutine ga\_set\_block\_cyclic\_proc\_grid(g\_a, dims, proc\_grid)
integer g\_a - global array handle {[}input{]} integer dims(ndim)
- array of block dimensions {[}input{]} integer proc\_grid(ndim)-
processor grid dimensions {[}input{]}

This subroutine is used to create a global array with a SCALAPACK-type
block cyclic data distribution. The user specifies the dimensions
of the processor grid in the array proc\_grid. The product of the
processor grid dimensions must equal the number of total number of
processors and the number of dimensions in the processor grid must
be the same as the number of dimensions in the global array. The data
blocks are mapped onto the processor grid in a cyclic manner along
each of the processor grid axes. This is illustrated below for an
array consisting of 25 data blocks disributed on 6 processors. The
6 processors are configured in a 3 by 2 processor grid. Blocks at
the edge of the array may be smaller than the block size specified
in dims. Most global array operations are insensitive to whether or
not a block-cyclic data distribution is used, although performance
may be slower in some cases if the global array is using a block-cyclic
data distribution. Individual data blocks can be accessesed using
the block-cyclic access functions.

(0,0) P0 (0,1) P3 (0,0) P0 (0,1) P3 (0,0) P0 (1,0) P1 (1,1) P4 (1,0)
P1 (1,1) P4 (1,0) P1 (2,0) P2 (2,1) P5 (2,0) P2 (2,1) P5 (2,0) P2
(0,0) P0 (0,1) P3 (0,0) P0 (0,1) P3 (0,0) P0 (1,0) P1 (1,1) P4 (1,0)
P1 (1,1) P4 (1.0) P1

This is a collective operation. GA\_ALLOCATE

logical function ga\_allocate(g\_a) integer g\_a - global array handle
{[}input{]}

This function allocates memory for the global array handle originally
obtained using the ga\_create\_handle function and returns true if
the allocation was successful. At a minimum, the nga\_set\_data subroutine
must be called before the memory is allocated. Other nga\_set\_xxx
functions can also be called before invoking this function.

This is a collective operation. GA\_UPDATE\_GHOSTS

subroutine ga\_update\_ghosts(g\_a) integer g\_a {[}input{]}

This call updates the ghost cell regions on each processor with the
corresponding neighbor data from other processors. The operation assumes
that all data is wrapped around using periodic boundary data so that
ghost cell data that goes beyound an array boundary is wrapped around
to the other end of the array.

This is a collective operation. GA\_UPDATE\_GHOST\_DIR

logical function nga\_update\_ghost\_dir(g\_a,dimension,idir,cflag)
integer g\_a {[}input{]} integer dimension - array dimension that
is to be updated {[}input{]} integer idir - direction of update (+/-
1) {[}input{]} logical cflag - flag to include corners in update {[}input{]}

This function can be used to update the ghost cells along individual
directions. It is designed for algorithms that can overlap updates
with computation. The variable dimension indicates which coordinate
direction is to be updated (e.g. dimension = 2 would correspond to
the y axis in a two or three dimensional system), the variable idir
can take the values +/-1 and indicates whether the side that is to
be updated lies in the positive or negative direction, and cflag indicates
whether or not the corners on the side being updated are to be included
in the update. The following calls would be equivalent to a call to
ga\_update\_ghosts for a 2-dimensional system: 

status = nga\_update\_ghost\_dir(g\_a,1,-1,.true.) status = nga\_update\_ghost\_dir(g\_a,1,1,.true.)
status = nga\_update\_ghost\_dir(g\_a,2,-1,.false.) status = nga\_update\_ghost\_dir(g\_a,2,1,.false.)

The variable cflag is set equal to .true. in the first two calls so
that the corner ghost cells are update, it is set equal to .false.
in the second two calls to avoid redundant updates of the corners.
Note that updating the ghosts cells using several independent calls
to the nga\_update\_ghost\_dir functions is generally not as efficient
as using ga\_update\_ghosts unless the individual calls can be effectively
overlapped with computation.

This is a collective operation. GA\_HAS\_GHOSTS

logical function ga\_has\_ghosts(g\_a) integer g\_a {[}input{]}

This function returns true if the global array has some dimensions
for which the ghost cell width is greater than zero, it returns false
otherwise. This is a collective operation. GA\_ACCESS\_GHOSTS

subroutine nga\_access\_ghosts(g\_a, dims, index, ld) integer g\_a
{[}input{]} integer dims(ndim) - array of dimensions of local patch,
including ghost cells {[}output{]}

integer index - returns an index corresponding to the origin the global
array patch held locally on the processor {[}output{]} integer ld(ndim)
- physical dimenstions of the local array patch, including ghost cells
{[}output{]}

Provides access to the local patch of the global array. Returns leading
dimension ld and and MA-like index for the data. This routine will
provide access to the ghost cell data residing on each processor.
Calls to nga\_access\_ghosts should normally follow a call to ga\_distribution
that returns coordinates of the visible data patch associated with
a processor. You need to make sure that the coordinates of the patch
are valid (test values returned from nga\_distribution).

Your code should include a MA include file, mafdecls.h. The addressing
convention:

dbl\_mb(index) - for double precision data int\_mb(index) - for integer
data dcpl\_mb(index) - for double complex data

refers the first element of the patch. However, you can only pass
that reference to another subroutine where it could be used like a
normal array, see the following example. This constraint caused by
the HP fortran compiler inability to reference shared memory data
properly. The C interface has no such restrictions.

Example

For a given global array in 2 dimensions with a local data patch whose
characteristics are

number of rows of visible data : irow number of columns of visible
data : jcol width of ghost cell data in first dimension: iwidth width
of ghost cell data in second dimension: jwidth

a call to ga\_access\_ghosts returns

dims(1) = irow + 2{*}iwidth dims(2) = jrow + 2{*}jwidth ld(1) = irow
+ 2{*}iwidth

After choosing the variables

nrows = dims(1) ncols = dims(2) lda = ld(1)

the ghost cell data can be accessed creating a dummy routine

subroutine foo(A, nrows, ncols lda) double precision A(lda,{*}) integer
nrows, ncols .... end

you can reference A(1:irow+2{*}iwidth,1:jrow+2{*}jwidth) in the following
way:

call foo(dbl\_mb(index), nrows, ncols, lda)

It will be up to the programmer to keep track of the fact that the
visible portion of the global array data corresponds to the ranges
(iwidth+1:irow+iwidth,jwidth+1:jcol+jwidth). Correctly accessing this
data may require additional arguments in the call to foo.

NOTE: You have to worry about mutual exclusion in simultaneous overlapping
access to the data by multiple processors.

Each call to ga\_access\_ghosts has to be followed by a call to either
ga\_release or ga\_release\_update. You can access only local data.
This is a local operation. GA\_ACCESS\_GHOST\_ELEMENT

subroutine nga\_access\_ghost\_element(g\_a, index, subscript, ld)
integer g\_a {[}input{]} integer index - index pointing to location
of element indexed by subscript() {[}output{]} integer subscript(ndim)
- array of integers that index desired element {[}input{]} integer
ld(ndim-1) - array of strides for local data patch. These include
ghost cell widths. {[}output{]}

This function can be used to return an index to any data element in
the locally held portion of the global array and can be used to directly
access ghost cell data. The array subscript refers to the local index
of the element relative to the origin of the local patch (which is
assumed to be indexed by (1,1,...)). This is a local operation. GA\_TOTAL\_BLOCKS

integer function ga\_total\_blocks(g\_a) integer g\_a {[}input{]}

This function returns the total number of blocks contained in a global
array with a block-cyclic data distribution. This is a local operation.
GA\_GET\_BLOCK\_INFO

subroutine ga\_get\_block\_info(g\_a, num\_blocks, block\_dims) integer
g\_a {[}input{]} integer num\_blocks(ndim) - number of blocks along
each axis {[}output{]} integer block\_dims(ndim) - dimensions of block
{[}output{]}

This subroutine returns information about the block-cyclic distribution
associated with global array g\_a. The number of blocks along each
of the array axes are returned in the array num\_blocks and the dimensions
of the individual blocks, specified in the ga\_set\_block\_cyclic
or ga\_set\_block\_cyclic\_proc\_grid subroutine, are returned in
block\_dims. This is a local function.

GA\_DUPLICATE

logical function ga\_duplicate(g\_a, g\_b, array\_name) integer g\_a,
g\_b; character{*}({*}) array\_name;

array\_name - a character string {[}input{]} g\_a - Integer handle
for reference array {[}input{]} g\_b - Integer handle for new array
{[}output{]}

Creates a new array by applying all the properties of another existing
array. This is a collective operation. GA\_DESTROY

logical function ga\_destroy(g\_a) integer g\_a {[}input{]}

Deallocates the array and frees any associated resources. This is
a collective operation. GA\_TERMINATE

subroutine ga\_terminate()

Delete all active arrays and destroy internal data structures. This
is a collective operation. GA\_SYNC

subroutine ga\_sync()

Synchronize processes (a barrier) and ensure that all global array
operations are complete (or at least appear complete). This is a collective
operation. GA\_MASK\_SYNC

subroutine ga\_mask\_sync(first,last)

logical first - mask for prior internal synchronization {[}input{]}
logical last - mask for post internal synchronization {[}input{]}

This subroutine can be used to remove synchronization calls from around
collective operations. Setting the parameter first = .false. removes
the synchronization prior to the collective operation, setting last
= .false. removes the synchronization call after the collective operation.
This call is applicable to all collective operations. It most be invoked
before each collective operation. This is a collective operation.
GA\_ZERO

subroutine ga\_zero(g\_a)

integer g\_a {[}input{]}

Sets value of all elements in the array to zero. This is a collective
operation. GA\_FILL

subroutine ga\_fill(g\_a, s)

integer g\_a {[}input{]} double precision/complex/integer s {[}input{]}

Assign a single value to all elements in the array. This is a collective
operation. GA\_DDOT

double precision function ga\_ddot(g\_a, g\_b)

integer g\_a, g\_b {[}input{]}

Computes the element-wise dot product of the two arrays which must
be double precision, the same shape and identically aligned:

ga\_ddot = SUM\_ij a(i,j){*}b(i,j)

This is a collective operation. GA\_ZDOT

double complex function ga\_zdot(g\_a, g\_b)

integer g\_a, g\_b {[}input{]}

Computes the element-wise dot product of the two arrays which must
be double complex, the same shape and identically aligned:

ga\_zdot = SUM\_ij a(i,j){*}b(i,j)

This is a collective operation.

GA\_DDOT\_PATCH

double precision function ga\_ddot\_patch(g\_a, ta, ailo, aihi, ajlo,
ajhi, g\_b, tb, bilo, bihi, bjlo, bjhi) 

integer g\_a, g\_b {[}input{]} integer ailo, aihi, ajlo, ajhi {[}input{]}
g\_a patch coordinates integer bilo, bihi, bjlo, bjhi {[}input{]}
g\_b patch coordinates character{*}1 ta, tb {[}input{]} transpose
flags

Computes the element-wise dot product of the two (possibly transposed)
patches which must be double precision and have the same number of
elements. This is a collective operation.

double precision function nga\_ddot\_patch(g\_a, ta, alo, ahi, g\_b,
tb, bio, bhi) 

integer g\_a, g\_b {[}input{]} integer ndim number of dimensions integer
alo(ndim), ahi(ndim) {[}input{]} g\_a patch coordinates integer blo(ndim),
bhi(ndim) {[}input{]} g\_b patch coordinates character{*}1 ta, tb
{[}input{]} transpose flags

Computes the element-wise dot product of the two (possibly transposed)
patches which must be double precision and have the same number of
elements. This is a collective operation. GA\_ZDOT\_PATCH

double complex function ga\_zdot\_patch(g\_a, ta, ailo, aihi, ajlo,
ajhi, g\_b, tb, bilo, bihi, bjlo, bjhi)

integer g\_a, g\_b {[}input{]} integer ailo, aihi, ajlo, ajhi {[}input{]}
g\_a patch coordinates integer bilo, bihi, bjlo, bjhi {[}input{]}
g\_b patch coordinates character{*}1 ta, tb {[}input{]} transpose
flags

Computes the element-wise dot product of the two (possibly transposed)
patches which must be double complex and have the same number of elements.
This is a collective operation.

double complex function nga\_zdot\_patch(g\_a, ta, alo, ahi, g\_b,
tb, blo, bhi)

integer g\_a, g\_b {[}input{]} integer ndim number of dimensions integer
alo(ndim), ahi(ndim) {[}input{]} g\_a patch coordinates integer blo(ndim),
bhi(ndim) {[}input{]} g\_b patch coordinates character{*}1 ta, tb
{[}input{]} transpose flags

Computes the element-wise dot product of the two (possibly transposed)
patches which must be double complex and have the same number of elements.
This is a collective operation. GA\_MATMUL\_PATCH

subroutine ga\_matmul\_patch(transa, transb, alpha, beta, g\_a, ailo,
aihi, ajlo, ajhi, g\_b, bilo, bihi, bjlo, bjhi, g\_c, cilo, cihi,
cjlo, cjhi)

integer g\_a, ailo, aihi, ajlo, ajhi {[}input{]} patch of g\_a integer
g\_b, bilo, bihi, bjlo, bjhi {[}input{]} patch of g\_b integer g\_c,
cilo, cihi, cjlo, cjhi {[}input{]} patch of g\_c double precision/complex
alpha, beta {[}input{]} character{*}1 transa, transb {[}input{]}

ga\_matmul\_patch is a patch version of ga\_dgemm:

C{[}cilo:cihi,cjlo:cjhi{]} := alpha{*} AA{[}ailo:aihi,ajlo:ajhi{]}
{*} BB{[}bilo:bihi,bjlo:bjhi{]} ) + beta{*}C{[}cilo:cihi,cjlo:cjhi{]},

where AA = op(A), BB = op(B), and op( X ) is one of

op( X ) = X or op( X ) = X',

Valid values for transpose arguments: 'n', 'N', 't', 'T'. It works
for both double precision and double complex data tape. This is a
collective operation. GA\_MATMUL\_PATCH

subroutine nga\_matmul\_patch(transa, transb, alpha, beta, g\_a, alo,
ahi, g\_b, blo, bhi, g\_c, clo, chi)

integer g\_a, alo, ahi {[}input{]} patch of g\_a integer g\_b, blo,
bhi {[}input{]} patch of g\_b integer g\_c, clo, chi {[}input{]} patch
of g\_c double precision/complex alpha, beta {[}input{]} character{*}1
transa, transb {[}input{]}

nga\_matmul\_patch is a n-dimensional patch version of ga\_dgemm:

C{[}clo{[}{]}:chi{[}{]}{]} := alpha{*} AA{[}alo{[}{]}:ahi{[}{]}{]}
{*} BB{[}blo{[}{]}:bhi{[}{]}{]} ) + beta{*}C{[}clo{[}{]}:chi{[}{]}{]},

where AA = op(A), BB = op(B), and op( X ) is one of

op( X ) = X or op( X ) = X',

Valid values for transpose arguments: 'n', 'N', 't', 'T'. It works
for both double precision and double complex data tape. This is a
collective operation. GA\_SCALE

subroutine ga\_scale(g\_a, s) integer g\_a {[}input{]} double precision/complex/integer
s {[}input{]}

Scales an array by the constant s. This is a collective operation.
GA\_ZERO\_PATCH

subroutine nga\_zero\_patch(g\_a, alo, ahi)

integer g\_a {[}input{]} integer ndim number of dimensions integer
alo(ndim), ahi(ndim) {[}input{]} g\_a patch coordinates

Set all the elements in the patch to zero. This is a collective operation.
GA\_SCALE\_PATCH

subroutine ga\_scale\_patch(g\_a, ailo, aihi, ajlo, ajhi, s)

integer g\_a {[}input{]} double precision/complex/integer s {[}input{]}
integer ailo, aihi, ajlo, ajhi {[}input{]} g\_a patch coordinates

Scales a patch of an array by the constant s. This is a collective
operation.

subroutine nga\_scale\_patch(g\_a, alo, ahi, s)

integer g\_a {[}input{]} double precision/complex/integer s {[}input{]}
integer ndim number of dimensions integer alo(ndim), ahi(ndim) {[}input{]}
g\_a patch coordinates

Scales a patch of an array by the constant s. This is a collective
operation. GA\_ADD

subroutine ga\_add(alpha, g\_a, beta, g\_b, g\_c)

integer g\_a, g\_b, g\_c {[}input{]} double precision/complex/integer
alpha, beta {[}input{]}

The arrays (which must be the same shape and identically aligned)
are added together element-wise

c = alpha {*} a + beta {*} b.

The result (c) may replace one of the input arrays (a/b). This is
a collective operation. GA\_ADD\_PATCH

subroutine ga\_add\_patch (alpha, g\_a, ailo, aihi, ajlo, ajhi, beta,
g\_b, bilo, bihi, bjlo, bjhi, g\_c, cilo, cihi, cjlo, cjhi)

integer g\_a, g\_b, g\_c {[}input{]} double precision/complex/integer
alpha, beta {[}input{]} integer ailo, aihi, ajlo, ajhi {[}input{]}
g\_a patch coordinates integer bilo, bihi, bjlo, bjhi {[}input{]}
g\_b patch coordinates integer cilo, cihi, cjlo, cjhi {[}input{]}
g\_c patch coordinates

Patches of arrays (which must have the same number of elements) are
added together element-wise. This is a collective operation.

subroutine nga\_add\_patch (alpha, g\_a, alo, ahi, beta, g\_b, blo,
bhi g\_c, clo, chi) integer g\_a, g\_b, g\_c {[}input{]} double precision/complex/integer
alpha, beta {[}input{]} integer ndim number of dimensions integer
alo(ndim), ahi(ndim) {[}input{]} g\_a patch coordinates integer blo(ndim),
bhi(ndim) {[}input{]} g\_b patch coordinates integer clo(ndim), chi(ndim)
{[}input{]} g\_c patch coordinates

Patches of arrays (which must have the same number of elements) are
added together element-wise.

This is a collective operation. GA\_FILL\_PATCH

subroutine ga\_fill\_patch(g\_a, ailo, aihi, ajlo, ajhi, s)

integer g\_a {[}input{]} double precision/complex/integer s {[}input{]}
integer ailo, aihi, ajlo, ajhi {[}input{]} g\_a patch coordinates

Fills the patch of an array with value s. Type of s and array must
match.

This is a collective operation.

subroutine nga\_fill\_patch(g\_a, alo, ahi, s)

integer g\_a {[}input{]} double precision/complex/integer s {[}input{]}
integer ndim number of dimensions integer alo(ndim), ahi(ndim) {[}input{]}
g\_a patch coordinates

Fills the patch of an array with value s. Type of s and array must
match.

This is a collective operation. GA\_SELECT\_ELEM

subroutine nga\_select\_elem(g\_a, op, val, index)

integer g\_a - array handle Control {[}input{]} character{*} op -
operator \{'min','max'\} {[}input{]} val - address where selected
value should be stored {[}output{]} index{[}ndim{]} - array index
for the selected element {[}output{]}

Returns the value and index for an element that is selected by the
specified operator in a global array corresponding to g\_a handle.
This is a collective operation. GA\_SUMMARIZE

subroutine ga\_summarize(verbose) logical verbose {[}input{]}! If
true print distribution info

Prints info about allocated arrays. This is a local operation. GA\_SYMMETRIZE

subroutine ga\_symmetrize(g\_a)

integer g\_a {[}input/output{]}

Symmetrizes matrix A represented with handle g\_a: A:= .5 {*} (A+A').
This is a collective operation. GA\_TRANSPOSE

subroutine ga\_transpose(g\_a, g\_b)

integer g\_a {[}input{]} ! remains unchanged integer g\_b {[}output{]}

Transposes a matrix: B = A', where A and B are represented by handles
g\_a and g\_b. This is a collective operation. GA\_DIAG

subroutine ga\_diag(g\_a, g\_s, g\_v, eval)

integer g\_a {[}input{]} ! Matrix to diagonalize integer g\_s {[}input{]}
! Metric integer g\_v {[}output{]} ! Global matrix to return evecs
double precision eval({*}) {[}output{]} ! Local array to return evals

Solve the generalized eigen-value problem returning all eigen-vectors
and values in ascending order. The input matrices are not overwritten
or destroyed. This is a collective operation. GA\_DIAG\_REUSE

subroutine ga\_diag\_reuse(control, g\_a, g\_s, g\_v, eval)

integer control {[}input{]} ! Control flag integer g\_a {[}input{]}
! Matrix to diagonalize integer g\_s {[}input{]} ! Metric integer
g\_v {[}output{]} ! Global matrix to return evecs double precision
eval({*}) {[}output{]} ! Local array to return evals

Solve the generalized eigen-value problem returning all eigen-vectors
and values in ascending order. Recommended for REPEATED calls if g\_s
is unchanged. Values of the control flag:

value action/purpose 0 indicates first call to the eigensolver >0
consecutive calls (reuses factored g\_s) <0 only erases factorized
g\_s; g\_v and eval unchanged (should be called after previous use
if another eigenproblem, i.e., different g\_a and g\_s, is to be solved)

The input matrices are not destroyed. This is a collective operation.
GA\_DIAG\_STD

subroutine ga\_diag\_std(g\_a, g\_v, eval)

integer g\_a {[}input{]} ! Matrix to diagonalize integer g\_v {[}output{]}
! Global matrix to return evecs double precision eval({*}) {[}output{]}
! Local array to return evals

Solve the standard (non-generalized) eigenvalue problem returning
all eigenvectors and values in the ascending order. The input matrix
is neither overwritten nor destroyed.

This is a collective operation. GA\_LU\_SOLVE

subroutine ga\_lu\_solve(trans, g\_a, g\_b)

character trans {[}input{]} ! transpose or not transpose integer g\_a
{[}input{]} ! coefficient matrix integer g\_b {[}output/output{]}
! rhs/solution matrix

Solve the system of linear equations op(A)X = B based on the LU factorization.

op(A) = A or A' depending on the parameter trans: trans = 'N' or 'n'
means that the transpose operator should not be applied. trans = 'T'
or 't' means that the transpose operator should be applied. Matrix
A is a general real matrix. Matrix B contains possibly multiple rhs
vectors. The array associated with the handle g\_b is overwritten
by the solution matrix X.

This is a collective operation. GA\_SPD\_INVERT

integer function ga\_spd\_invert(g\_a) integer g\_a {[}input/output{]}
! matrix

It computes the inverse of a double precision using the Cholesky factorization
of a NxN double precision symmetric positive definite matrix A stored
in the global array represented by g\_a. On successful exit, A will
contain the inverse.

It returns

= 0 : successful exit > 0 : the leading minor of this order is not
positive definite and the factorization could not be completed < 0
: it returns the index i of the (i,i) element of the factor L/U that
is zero and the inverse could not be computed

This is a collective operation. GA\_LLT\_SOLVE

integer function ga\_llt\_solve(g\_a, g\_b)

integer g\_a {[}input{]} ! coefficient matrix integer g\_b {[}output/output{]}
! rhs/solution matrix

Solves a system of linear equations

A {*} X = B

using the Cholesky factorization of an NxN double precision symmetric
positive definite matrix A (epresented by handle g\_a). On successful
exit B will contain the solution X. It returns:

= 0 : successful exit > 0 : the leading minor of this order is not
positive definite and the factorization could not be completed

This is a collective operation. GA\_SOLVE

integer function ga\_solve(g\_a, g\_b) integer g\_a {[}input{]} !
coefficient matrix integer g\_b {[}output/output{]} ! rhs/solution
matrix

Solves a system of linear equations

A {*} X = B

It first will call the Cholesky factorization routine and, if sucessfully,
will solve the system with the Cholesky solver. If Cholesky will be
not be able to factorize A, then it will call the LU factorization
routine and will solve the system with forward/backward substitution.
On exit B will contain the solution X.

It returns

= 0 : Cholesky factoriztion was succesful > 0 : the leading minor
of this order is not positive definite, Cholesky factorization could
not be completed and LU factoriztion was used

This is a collective operation. GA\_GEMM

subroutine ga\_dgemm(transa, transb, m, n, k, alpha, g\_a, g\_b, beta,
g\_c )

subroutine ga\_sgemm(transa, transb, m, n, k, alpha, g\_a, g\_b, beta,
g\_c )

subroutine ga\_zgemm(transa, transb, m, n, k, alpha, g\_a, g\_b, beta,
g\_c )

Character{*}1 transa, transb {[}input{]} Integer m, n, k {[}input{]}
Double Precision alpha, beta {[}input{]} (DGEMM) Double Complex alpha,
beta {[}input{]} (ZGEMM) Integer g\_a, g\_b, {[}input{]} Integer g\_c
{[}output{]}

Performs one of the matrix-matrix operations:

C := alpha{*}op( A ){*}op( B ) + beta{*}C,

where op( X ) is one of

op( X ) = X or op( X ) = X',

alpha and beta are scalars, and A, B and C are matrices, with op(
A ) an m by k matrix, op( B ) a k by n matrix and C an m by n matrix.

On entry, transa specifies the form of op( A ) to be used in the matrix
multiplication as follows:

transa = 'N' or 'n', op( A ) = A. transa = 'T' or 't', op( A ) = A'.
m - On entry, m specifies the number of rows of the matrix op( A )
and of the matrix C. m must be at least zero. n - On entry, n specifies
the number of columns of the matrix op( B ) and the number of columns
of the matrix C. n must be at least zero. k - On entry, k specifies
the number of columns of the matrix op( A ) and the number of rows
of the matrix op( B ). K must be at least zero.

This is a collective operation.

GA\_COPY

subroutine ga\_copy(g\_a, g\_b) integer g\_a, g\_b {[}input{]}

Copies elements in array represented by g\_a into the array represented
by g\_b. The arrays must be the same type, shape, and identically
aligned.

This is a collective operation. GA\_COPY\_PATCH

subroutine ga\_copy\_patch(trans, g\_a, ailo, aihi, ajlo, ajhi, g\_b,
bilo, bihi, bjlo, bjhi) character trans {[}input{]} transpose operator
integer g\_a, g\_b {[}input{]} integer ailo, aihi, ajlo, ajhi {[}input{]}
g\_a patch coordinates integer bilo, bihi, bjlo, bjhi {[}input{]}
g\_b patch coordinates

Copies elements in a patch of one array into another one. The patches
of arrays should not overloap (when g\_a=g\_b) and must have the same
number of elements. trans = 'N' or 'n' means that the transpose operator
should not be applied. trans = 'T' or 't' means that transpose operator
should be applied.

This is a collective operation.

subroutine nga\_copy\_patch(trans, g\_a, alo, ahi, g\_b, blo, bhi)
character trans {[}input{]} transpose operator integer g\_a, g\_b
{[}input{]} integer ndim number of dimensions integer alo(ndim), ahi(ndim)
{[}input{]} g\_a patch coordinates integer blo(ndim), bhi(ndim) {[}input{]}
g\_b patch coordinates

Copies elements in a patch of one array into another one. The patches
of arrays should not overloap (when g\_a=g\_b) and must have the same
number of elements. trans = 'N' or 'n' means that the transpose operator
should not be applied. trans = 'T' or 't' means that transpose operator
should be applied. This is a collective operation. GA\_GET

subroutine ga\_get(g\_a, ilo, ihi, jlo, jhi, buf, ld)

integer g\_a {[}input{]} integer ilo, ihi, jlo, jhi {[}input{]} double
precision/complex/integer buf {[}output{]} integer ld {[}input{]}

Performs the equivalent of the following Fortran-90.

dimension A(1:dim1,1:dim2) dimension buf(1:ld, 1:{*})

buf(1:ihi-ilo+1, 1:jhi-jlo+1) = A(ilo:ihi, jlo:jhi)

where g\_a represents a handle to the array A. Array bounds are always
checked. This is a one-sided/independent operation.

subroutine nga\_get(g\_a, lo, hi, buf, ld) integer g\_a - global array
handle {[}input{]} integer ndim - number of dimensions of the global
array integer lo(ndim) - array of starting indices for global array
section {[}input{]} integer hi(ndim) - array of ending indices for
global array section {[}input{]} <type> buf - local buffer array where
the data goes to {[}output{]} integer ld(ndim-1) - array specifying
leading dimensions for buffer array {[}input{]}

Copies data from global array section to the local array buffer. The
local array is assumed to be have the same number of dimensions as
the global array. Any detected inconsitencies/errors in the input
arguments are fatal.

Example: For ga\_get operation transfering data from the (11:15,1:5)
section of 2-dimensional 15x10 global array into local buffer 5x10
array we have: lo=\{11,1\}, hi=\{15,5\}, ld=\{10\} 

15 5 

10 

10 

This is a one-sided/independent operation. GA\_PERIODIC\_GET subroutine
nga\_periodic\_get(g\_a, lo, hi, buf, ld) integer g\_a - global array
handle {[}input{]} integer ndim - number of dimensions of the global
array integer lo(ndim) - array of starting indices for global array
section {[}input{]} integer hi(ndim) - array of ending indices for
global array section {[}input{]} <type> buf - local buffer array where
the data goes to {[}output{]} integer ld(ndim-1) - array specifying
leading dimensions for buffer array {[}input{]}

Same as nga\_get except the indices can extend beyond the array boundary/dimensions
in which case the library wraps them around. This is a one-sided/independent
operation. GA\_STRIDED\_GET

subroutine nga\_strided\_get(g\_a, lo, hi, skip, buf, ld) integer
g\_a - global array handle {[}input{]} integer ndim - number of dimensions
of the global array integer lo(ndim) - array of starting indices for
global array section {[}input{]} integer hi(ndim) - array of ending
indices for global array section {[}input{]} integer skip(ndim) -
array of strides for each dimension {[}input{]} <type> buf - local
buffer array where the data comes from {[}output{]} integer ld(ndim-1)
- array specifying leading dimensions for buffer array {[}input{]}

This operation is the same as nga\_get, except that the values corresponding
to dimension n in buf correspond to every skip(n) values of the global
array g\_a. This is a onesided/independent operation. GA\_PUT

subroutine ga\_put(g\_a, ilo, ihi, jlo, jhi, buf, ld)

integer g\_a {[}input{]} integer ilo, ihi, jlo, jhi {[}input{]} double
precision/complex/integer buf {[}output{]} integer ld {[}input{]}

Performs the equivalent of the following Fortran-90: dimension g\_a(1:dim1,1:dim2)
dimension buf(1:ld, 1:{*})

g\_a(ilo:ihi, jlo:jhi) = buf(1:ihi-ilo+1, 1:jhi-jlo+1)

where g\_a represents a handle to the array A. Array bounds are always
checked. This is a one-sided/independent operation.

subroutine nga\_put(g\_a, lo, hi, buf, ld) integer g\_a - global array
handle {[}input{]} integer ndim - number of dimensions of the global
array integer lo(ndim) - array of starting indices for global array
section {[}input{]} integer hi(ndim) - array of ending indices for
global array section {[}input{]} <type> buf - local buffer array where
the data comes from {[}output{]} integer ld(ndim-1) - array specifying
leading dimensions for buffer array {[}input{]}

Copies data from local array buffer to the global array section .
The local array is assumed to be have the same number of dimensions
as the global array. Any detected inconsitencies/errors in input arguments
are fatal. This is a one-sided/independent operation. GA\_PERIODIC\_PUT

subroutine nga\_periodic\_put(g\_a, lo, hi, buf, ld) integer g\_a
- global array handle {[}input{]} integer ndim - number of dimensions
of the global array integer lo(ndim) - array of starting indices for
global array section {[}input{]} integer hi(ndim) - array of ending
indices for global array section {[}input{]} <type> buf - local buffer
array where the data comes from {[}output{]} integer ld(ndim-1) -
array specifying leading dimensions for buffer array {[}input{]}

Same as nga\_put except the indices can extend beyond the array boundary/dimensions
in which case the library wraps them around. This is a one-sided/independent
operation. GA\_STRIDED\_PUT

subroutine nga\_strided\_put(g\_a, lo, hi, skip, buf, ld) integer
g\_a - global array handle {[}input{]} integer ndim - number of dimensions
of the global array integer lo(ndim) - array of starting indices for
global array section {[}input{]} integer hi(ndim) - array of ending
indices for global array section {[}input{]} integer skip(ndim) -
array of strides for each dimension {[}input{]} <type> buf - local
buffer array where the data comes from {[}output{]} integer ld(ndim-1)
- array specifying leading dimensions for buffer array {[}input{]}

This operation is the same as nga\_put, except that the values corresponding
to dimension n in buf are copied to every skip(n) value of the global
array g\_a. This is a onesided/independent operation. GA\_ACC

subroutine ga\_acc(g\_a, ilo, ihi, jlo, jhi, buf, ld, alpha)

integer g\_a {[}input{]} integer ilo, ihi, jlo, jhi {[}input{]} double
precision/complex buf {[}input{]} integer ld {[}input{]} double precision/complex
alpha {[}input{]}

Performs the equivalent of the following Fortran-90. Array bounds
are always checked. Double precision/complex types are supported.
The operation is atomic.

dimension g\_a(1:dim1,1:dim2) dimension buf(1:ld, 1:{*})

A(ilo:ihi, jlo:jhi) = A(ilo:ihi, jlo:jhi) + alpha{*}buf(1:ihi-ilo+1,
1:jhi-jlo+1)

where g\_a represents a handle to the array A.

This is a one-sided/independent and atomic operation.

subroutine nga\_acc(g\_a, lo, hi, buf, ld, alpha) integer g\_a - global
array handle {[}input{]} integer ndim - number of dimensions of the
global array integer lo(ndim) - array of starting indices for global
array section {[}input{]} integer hi(ndim) - array of ending indices
for global array section {[}input{]} <type> buf - local buffer array
where the local data is {[}output{]} integer ld(ndim-1) - array specifying
leading dimensions for buffer array {[}input{]} <type> alpha - scale
argument for accumulate {[}input{]}

Combines data from local array buffer with data in the global array
section. The local array is assumed to be have the same number of
dimensions as the global array.

global array section (lo(),hi()) += alpha {*} buffer()

This is a one-sided/independent and atomic operation. GA\_PERIODIC\_ACC

subroutine nga\_periodic\_acc(g\_a, lo, hi, buf, ld, alpha) integer
g\_a - global array handle {[}input{]} integer ndim - number of dimensions
of the global array integer lo(ndim) - array of starting indices for
global array section {[}input{]} integer hi(ndim) - array of ending
indices for global array section {[}input{]} <type> buf - local buffer
array where the local data is {[}output{]} integer ld(ndim-1) - array
specifying leading dimensions for buffer array {[}input{]} <type>
alpha - scale argument for accumulate {[}input{]}

Same as nga\_acc except the indices can extend beyond the array boundary/dimensions
in which case the library wraps them around. This is a one-sided/independent
and atomic operation. GA\_STRIDED\_ACC

subroutine nga\_strided\_acc(g\_a, lo, hi, skip, buf, ld, alpha) integer
g\_a - global array handle {[}input{]} integer ndim - number of dimensions
of the global array integer lo(ndim) - array of starting indices for
global array section {[}input{]} integer hi(ndim) - array of ending
indices for global array section {[}input{]} integer skip(ndim) -
array of strides for each dimension {[}input{]} <type> buf - local
buffer array where the data comes from {[}output{]} integer ld(ndim-1)
- array specifying leading dimensions for buffer array {[}input{]}
<type> alpha - scale argument for accumulate {[}input{]}

This operation is the same as nga\_acc, except that the values corresponding
to dimension n in buf are accumulated to every skip(n) value of the
global array g\_a. This is a onesided/independent operation. GA\_DISTRIBUTION

subroutine ga\_distribution(g\_a, iproc, ilo, ihi, jlo, jhi) integer
g\_a - array handle {[}input{]} integer iproc - process number {[}input{]}
integer ilo,ihi,jlo,jhi - range held by process iproc {[}output{]}

Returns the range of a global array held by a specified process. If
no array elements are owned by process iproc, the range is returned
as {[}0, -1{]} for the i and {[}0,-1{]} for thej dimensions. This
is a local operation.

subroutine nga\_distribution(g\_a, iproc, lo, hi) integer g\_a - array
handle {[}input{]} integer iproc - process number {[}input{]} integer
ndim - number of dimensions integer lo(ndim),hi(ndim) - range held
by process iproc {[}output{]}

Returns the range of a global array held by a specified process. If
no array elements are owned by process iproc, the range is returned
as lo(i) =0 and hi(i)= -1 for i=1,ndim dimensions. This is the n-dimensional
version of ga\_distribution.

This is a local operation. GA\_COMPARE\_DISTR

logical function ga\_compare\_distr(g\_a, g\_b)

integer g\_a, g\_b {[}input{]}

Compares distributions of two global arrays. Returns .TRUE. if distributions
are identical. This is a local operation (as of release 3.0). GA\_ACCESS

subroutine ga\_access(g\_a, ilo, ihi, jlo, jhi, index, ld)

integer g\_a {[}input{]} integer ilo, ihi, jlo, jhi {[}input{]} integer
index {[}output{]} integer ld {[}output{]}

Provides access to the specified patch of array. Returns leading dimension
ld and and MA-like index for the data. This routine is intended for
writing new GA operations. Call to ga\_access should normally follow
a call to ga\_distribution that returns coordinates of the patch associated
with a processor. You need to make sure that the coordinates of the
patch are valid (test values returned from ga\_distribution).

Your code should include a MA include file, mafdecls.h. The addressing
convention:

dbl\_mb(index) - for double precision data int\_mb(index) - for integer
data dcpl\_mb(index) - for double complex data

refers the first element (ilo,jlo) of the patch. However, you can
only pass that reference to another subroutine where it could be used
like a normal array, see the following example. This constraint caused
by the HP fortran compiler inability to reference shared memory data
properly. The C interface has no such restrictions.

Example

For a given subroutine:

subroutine foo(A, nrows, ncols lda) double precision A(lda,{*}) integer
nrows, ncols .... end

you can reference A(ilo:ihi,jlo:jhi) in the following way:

call foo(dbl\_mb(index), ihi-ilo+1, jhi-jlo+1, lda)

NOTE: You have to worry about mutual exclusion in simultaneous overlapping
access to the data by multiple processors.

Each call to ga\_access has to be followed by a call to either ga\_release
or ga\_release\_update. You can access only local data. This is a
local operation.

subroutine nga\_access(g\_a, lo, hi, index, ld)

integer g\_a array handle {[}input{]} integer ndim number of array
dimensions integer lo(ndim),hi(ndim) patch specification {[}input{]}
integer index reference to local data {[}output{]} integer ld(ndim-1)
array of leading dimensions {[}output{]}

An n-dimensional version of ga\_access. This is a local operation.
GA\_ACCESS\_BLOCK\_SEGMENT

subroutine nga\_access\_block\_segment(g\_a, proc, index, len)

integer g\_a array handle {[}input{]} integer proc processor ID {[}input{]}
integer index reference to local data {[}output{]} integer len length
of data on processor {[}output{]}

This function can be used to gain access to the all the locally held
data on a particular processor that is associated with a block-cyclic
distributed array. Once the index has been returned, local data can
be accessed as described in the documentation for ga\_access. The
parameter len is the number of data elements that are held locally.
The data inside this segment has a lot of additional structure so
this function is not generally useful to developers. It is primarily
used inside the GA library to implement other GA routines. Each call
to ga\_access\_block\_segment should be followed by a call to either
nga\_release\_block\_segment or nga\_release\_update\_block\_segment.

This is a local operation. GA\_ACCESS\_BLOCK

subroutine nga\_access\_block(g\_a, idx, index, ld)

integer g\_a array handle {[}input{]} integer ndim number of array
dimensions integer idx block index {[}input{]} integer index reference
to local data {[}output{]} integer ld(ndim-1) array of leading dimensions
{[}output{]}

This function can be used to gain direct access to the data represented
by a single block in a global array with a block-cyclic data distribution.
The index idx is the index of the block in the array assuming that
blocks are numbered sequentially in a column-major order. A quick
way of determining whether a block with index idx is held locally
on a processor is to calculate whether mod(idx,nproc) equals the processor
ID, where nproc is the total number of processors. Once the index
has been returned, local data can be accessed as described in the
documentation for ga\_access. Each call to ga\_access\_block should
be followed by a call to either nga\_release\_block or nga\_release\_update\_block.

This is a local operation. GA\_ACCESS\_BLOCK\_GRID

subroutine nga\_access\_block\_grid(g\_a, subscript, index, ld)

integer g\_a array handle {[}input{]} integer ndim number of array
dimensions integer subscript(ndim) subscript of block in array {[}input{]}
integer index reference to local data {[}output{]} integer ld(ndim-1)
array of leading dimensions {[}output{]}

This function can be used to gain direct access to the data represented
by a single block in a global array with a SCALAPACK block-cyclic
data distribution that is based on an underlying processor grid. The
subscript array contains the subscript of the block in the array of
blocks. This subscript is based on the location of the block in a
grid, each of whose dimensions is equal to the number of blocks that
fit along that dimension. Once the index has been returned, local
data can be accessed as described in the documentation for ga\_access.
Each call to ga\_access\_block\_grid should be followed by a call
to either nga\_release\_block\_grid or nga\_release\_update\_block\_grid.

This is a local operation.

GA\_RELEASE

subroutine ga\_release(g\_a, ilo, ihi, jlo, jhi)

integer g\_a {[}input{]} integer ilo, ihi, jlo, jhi {[}input{]}

Releases access to a global array when the data was read only.

Your code should look like:

call ga\_distribution(g\_a, myproc, ilo,ihi,jlo,jhi) call ga\_access(g\_a,
ilo, ihi, jlo, jhi, index, ld) < operate on the data >

call ga\_release(g\_a, ilo, ihi, jlo, jhi)

NOTE: see restrictions specified for ga\_access. This is a local operation.

subroutine nga\_release(g\_a, lo, hi)

integer g\_a array handle {[}input{]} integer ndim number of array
dimensions integer lo(ndim),hi(ndim) patch specification {[}input{]}

Releases access to local elements of global array when the data was
accessed as read only.This is a local operation. GA\_RELEASE\_UPDATE

subroutine ga\_release\_update(g\_a, ilo, ihi, jlo, jhi)

integer g\_a {[}input{]} integer ilo, ihi, jlo, jhi {[}input{]}

Releases access to the data. It must be used if the data was accessed
for writing. NOTE: see restrictions specified for ga\_access. This
is a local operation.

subroutine nga\_release\_update(g\_a, lo, hi)

integer g\_a array handle {[}input{]} integer ndim number of array
dimensions integer lo(ndim),hi(ndim) patch specification {[}input{]}

Releases access to local elements of global array when the data was
accessed in read -write mode. GA\_RELEASE\_BLOCK

subroutine nga\_release\_block(g\_a, index)

integer g\_a array handle {[}input{]} integer index block index {[}input{]}

Releases access to the block of data specified by the integer index
when data was accessed as read only. This is only applicable to block-cyclic
data distributions created using the simple block-cyclic distribution.
This is a local operation. GA\_RELEASE\_UPDATE\_BLOCK

subroutine nga\_release\_update\_block(g\_a, index)

integer g\_a array handle {[}input{]} integer index block index {[}input{]}

Releases access to the block of data specified by the integer index
when data was accessed in read-write mode. This is only applicable
to block-cyclic data distributions created using the simple block-cyclic
distribution. This is a local operation. GA\_RELEASE\_BLOCK\_GRID

subroutine nga\_release\_block\_grid(g\_a, subscript)

integer g\_a array handle {[}input{]} integer subscript(ndim) indices
of block in array {[}input{]}

Releases access to the block of data specified by the subscript array
when data was accessed as read only. This is only applicable to block-cyclic
data distributions created using the SCALAPACK data distribution.
This is a local operation. GA\_RELEASE\_UPDATE\_BLOCK\_GRID

subroutine nga\_release\_update\_block\_grid(g\_a, subscript)

integer g\_a array handle {[}input{]} integer subscript(ndim) indices
of block in array {[}input{]}

Releases access to the block of data specified by the subscript array
when data was accessed in read-write mode. This is only applicable
to block-cyclic data distributions created using the SCALAPACK data
distribution. This is a local operation. GA\_RELEASE\_BLOCK\_SEGMENT

subroutine nga\_release\_block\_segment(g\_a, iproc)

integer g\_a array handle {[}input{]} integer iproc processor ID {[}input{]}

Releases access to the block of locally held data for a block-cyclic
array, when data was accessed as read-only. This is a local operation.
GA\_RELEASE\_UPDATE\_BLOCK\_SEGMENT

subroutine nga\_release\_update\_block\_segment(g\_a, iproc)

integer g\_a array handle {[}input{]} integer iproc processor ID {[}input{]}

Releases access to the block of locally held data for a block-cyclic
array, when data was accessed in read-write mode. This is a local
operation. GA\_RELEASE\_GHOST\_ELEMENT

subroutine nga\_release\_ghost\_element(g\_a, subscript)

integer g\_a array handle {[}input{]} integer subscript(ndim) element
subscript {[}input{]}

Releases access to the locally held data for an array with ghost elements,
when data was accessed as read-only. This is a local operation. GA\_RELEASE\_UPDATE\_GHOST\_ELEMENT

subroutine nga\_release\_update\_ghost\_element(g\_a, subscript)

integer g\_a array handle {[}input{]} integer subscript(ndim) element
subscript {[}input{]}

Releases access to the locally held data for an array with ghost elements,
when data was accessed in read-write mode. This is a local operation.
GA\_RELEASE\_GHOSTS

subroutine nga\_release\_ghosts(g\_a)

integer g\_a array handle {[}input{]}

Releases access to the locally held block of data containing ghost
elements, when data was accessed as read-only. This is a local operation.
GA\_RELEASE\_UPDATE\_GHOSTS

subroutine nga\_release\_update\_ghosts(g\_a)

integer g\_a array handle {[}input{]}

Releases access to the locally held block of data containing ghost
elements, when data was accessed in read-write mode. This is a local
operation.

GA\_READ\_INC

integer function ga\_read\_inc(g\_a, i, j, inc)

integer g\_a {[}input{]} integer i, j, inc {[}input{]}

Atomically read and increment an element in an integer array.

{*}BEGIN CRITICAL SECTION{*} return value = a(i,j) a(i,j) += inc {*}END
CRITICAL SECTION{*}

This is a one-sided/independent operation.

integer function nga\_read\_inc(g\_a, subscript, inc)

integer g\_a {[}input{]} integer i, j, inc {[}input{]}

Atomically read and increment an element in an integer array. An n-dimensional
version of ga\_read\_inc.

This is a one-sided/independent operation. GA\_SCATTER

subroutine ga\_scatter(g\_a, v, i, j, n)

integer g\_a {[}input{]} double precision v(n) {[}input{]} integer
i(n), j(n), n {[}input{]}

Scatters the array elements into the array. The contents of the input
arrays (v, i, j) are preserved, but their contents might be (consistently)
shuffled on return.

do k = 1, nv a(i(k),j(k)) = v(k) enddo

This is a one-sided/independent operation.

subroutine nga\_scatter(g\_a, v, subsArray, n) integer g\_a - global
array handle {[}input{]} integer n - number of elements {[}input{]}
<type> v(n) - array containing values {[}input{]} integer ndim - number
of array dimensions subsArray (ndim,n) - array of subscripts for each
element {[}input{]}

Scatters array elements into a global array. The contents of the input
arrays (v,subscrArray) are preserved, but their contents might be
(consistently) shuffled on return.

do k = 1, n a(subsArray(1,k),subsArray(2,k), ...) = v(k) enddo

This is a one-sided/independent operation. GA\_GATHER

subroutine ga\_gather(g\_a, v, i, j, n)

integer g\_a {[}input{]} double precision v(n) {[}output{]} integer
i(n), j(n), n {[}input{]}

Gathers specified elements (i,j) from the global array into a local
single-dimesional array. The contents of the arrays (v, i, j) might
be (consistently) shuffled on return.

do k = 1, nv v(k) = a(i(k),j(k)) enddo

This is a one-sided/independent operation.

subroutine nga\_gather(g\_a, v, subsArray, n) integer g\_a - global
array handle {[}input{]} {[}output{]} integer n - number of elements
{[}input{]} <type> v(n) - array containing values {[}output{]} integer
ndim - number of array dimensions subsArray (ndim,n) - array of subscripts
for each element {[}input{]}

Gathers array elements from a global array into a local array. The
contents of the input arrays (v, subscrArray) are preserved, but their
contents might be (consistently) shuffled on return.

do k = 1, n v(k) = a(subsArray(1,k),subsArray(2,k), ...) enddo

This is a one-sided/independent operation. GA\_SCATTER\_ACC

subroutine ga\_scatter\_acc(g\_a, v, i, j, n, alpha) integer g\_a
- global array handle {[}input{]} integer n - number of elements {[}input{]}
<type> v(n) - array containing value {[}input{]} integer i(n),j(n)
- arrays of indices {[}input{]} double precision/complex alpha - multiplicative
value {[}input{]} 

Scatters array elements from a local array into a global array. Adds
values from the local array to existing values in the global array
after multiplying by alpha. The contents of the arrays v, i, j may
be (consistently) shuffled on return.

This is a one-sided/independent operation.

subroutine nga\_scatter\_acc(g\_a, v, subsArray, n, alpha) integer
g\_a - global array handle {[}input{]} integer n - number of elements
{[}input{]} <type> v(n) - array containing value {[}input{]} integer
ndim - number of array dimensions subsArray(ndim,n) - array of subscripts
{[}input{]} double precision/complex alpha - multiplicative value
{[}input{]}

Scatters array elements from a local array into a global array. Adds
values from the local array to existing values in the global array
after multiplying by alpha. The contents of v and subsArray may be
(consistently) shuffled on return.

This is a one-sided/independent operation.

GA\_ERROR

subroutine ga\_error(message, code) character{*}1 message({*}) {[}input{]}
integer code {[}input{]}

To be called in case of an error. Print an error message and an integer
value that is intended to be the error code. Releases some system
resources. This is the recomended way of aborting the program execution.
GA\_LOCATE

logical function ga\_locate(g\_a, i, j, owner)

integer g\_a array handle {[}input{]} integer i, j element subscript
{[}input{]} integer owner process id {[}output{]}

Return in owner the GA compute process id that 'owns' the data. If
i/j are out of bounds .FALSE. is returned, otherwise .TRUE.. This
is a local operation.

logical function nga\_locate(g\_a, subscript, owner)

integer g\_a array handle {[}input{]} integer subscript element subscript
{[}input{]} integer owner process id {[}output{]}

Return in owner the GA compute process id that 'owns' the array element.
If subscript contains values out of bounds .FALSE. is returned, otherwise
.TRUE. This is a local operation. GA\_LOCATE\_REGION

logical function ga\_locate\_region(g\_a, ilo, ihi, jlo, jhi, map,
np )

integer g\_a, ilo, ihi, jlo, jhi {[}input{]} integer map(5,{*}), np
{[}output{]}

Parts of the specified patch might be actually 'owned' by several
(precisely np) processes. Return the list of the GA processes id that
'own' the data. If i/j are out of bounds .FALSE. is returned, otherwise
.TRUE..

map(1,{*}) - ilo map(2,{*}) - ihi map(3,{*}) - jlo map(4,{*}) - jhi

specify coordinates of a subpatch 'held' by the process which number
is stored in map(5,{*}) This is a local operation.

logical function nga\_locate\_region(g\_a, lo, hi, map, proclist,
np )

integer g\_a array handle {[}input{]} integer ndim number of dimensions
integer lo(ndim),hi(ndim) region(patch) specifications {[}input{]}
integer map(2{*}ndim,{*}) patch ownership array {[}output{]} integer
proclist(np) list of processes {[}output{]} integer np number of processes
{[}output{]}

An n-dimensional version of ga\_locate\_region. The list of processes
that contain parts for the region are in proclist. Array map is defined
as

map(1:ndim,i) - contains lower bound dimensions for part owned by
process proclist(i) map(ndim+1:2{*}ndim,i) - contains upper bound
dimensions for part owned by process proclist(i)

This is a local operation. GA\_INQUIRE

subroutine ga\_inquire(g\_a, type, dim1, dim2)

integer g\_a {[}input{]} integer type {[}output{]} integer dim1 {[}output{]}
integer dim2 {[}output{]}

Returns type and dimensions of array. This is a local operation.

subroutine nga\_inquire(g\_a, type, ndim, dims)

integer g\_a array handle {[}input{]} integer type data type id {[}output{]}
integer ndim number of dimensions {[}output{]} integer dims(ndim)
array of dimensions {[}output{]}

Returns type and dimensions of an n-dimensional array. This is a local
operation. GA\_INQUIRE\_MEMORY

integer function ga\_inquire\_memory()

Returns amount of memory (in bytes) used in the allocated global arrays
on the calling processor. This is a local operation. GA\_INQUIRE\_NAME

subroutine ga\_inquire\_name(g\_a, array\_name)

integer g\_a {[}input{]} character{*}({*}) array\_name {[}output{]}

Returns the name of an array represented by the handle g\_a. This
is a local operation. GA\_NDIM

integer function ga\_ndim(g\_a)

integer g\_a {[}input{]}

Returns the number ot dimensions in an array represented by the handle
g\_a. This is a local operation. GA\_NBLOCK

subroutine ga\_nblock(g\_a, nblock) integer g\_a - array handle {[}input{]}
integer nblock{[}ndim{]} - number of partitions for each dimension
{[}output{]}

Given a distribution of an array represented by the handle g\_a, returns
the number of partitions of each array dimension. This operation is
local. GA\_MEMORY\_AVAIL

integer function ga\_memory\_avail()

Returns amount of memory (in bytes) left for allocation of new global
arrays on the calling processor.

Note: If ga\_uses\_ma returns true, then ga\_memory\_avail returns
the lesser of the amount available under the GA limit and the amount
available from MA (according to ma\_inquire\_avail operation). If
no GA limit has been set, it returns what MA says is available.

If ( ! ga\_uses\_ma() \&\& ! ga\_memory\_limited() ) returns < 0,
indicating that the bound on currently available memory cannot be
determined. This is a local operation. GA\_USES\_MA

logical function ga\_uses\_ma()

Returns .true. if memory in global arrays comes from the Memory Allocator
(MA). .false. means that memory comes from another source, for example
System V shared memory is used. This is a local operation. GA\_MEMORY\_LIMITED

logical function ga\_memory\_limited()

Indicates if limit is set on memory usage in Global Arrays on the
calling processor. This is a local operation. GA\_SET\_MEMORY\_LIMIT

subroutine ga\_set\_memory\_limit(limit)

integer limit {[}input{]}

Sets the amount of memory (in bytes) to be used per process. This
is a local operation. GA\_PROC\_TOPOLOGY

subroutine ga\_proc\_topology(g\_a, proc, prow, pcol)

integer g\_a {[}input{]} integer proc {[}input{]} integer prow, pcol
{[}output{]}

Based on the distribution of the array associated with handle g\_a,
determines block row and column coordinates for the array section
'owned' by specified processor. The numbering starts from 0. The value
of -1 means that the processor doesn't 'own' any section of array
represented by g\_a. This is a local operation. GA\_PRINT\_PATCH

subroutine ga\_print\_patch(g\_a,ilo,ihi,jlo,jhi,pretty) integer g\_a
{[}input{]} integer ilo,ihi,jlo,jhi {[}input{]} coordinates of the
patch integer pretty {[}input{]}

Prints a patch of g\_a array to the standard output. If pretty has
the value 0 then output is printed in a dense fashion. If pretty has
the value 1 then output is formatted and rows/columns labeled. This
is a collective operation. GA\_PRINT

subroutine ga\_print(g\_a) integer g\_a {[}input{]}

Prints an entire array to the standard output. This is a collective
operation. GA\_PRINT\_STATS

subroutine ga\_print\_stats()

This non-collective (MIMD) operation prints information about:

{*} number of calls to the GA create/duplicate, destroy, get, put,
scatter, gather, and read\_and\_inc operations {*} total amount of
data moved in the GA primitive operations {*} amount of data moved
in GA primitive operations to logicaly remote locations {*} maximum
memory consumption in global arrays, and {*} number of requests serviced
in the interrupt-driven implementations by the calling process.

This is a local operation. GA\_PRINT\_DISTRIBUTION

subroutine ga\_print\_distribution(g\_a) integer g\_a {[}input{]}

Prints array distribution.

This is a collective operation. GA\_CHECK\_HANDLE

subroutine ga\_check\_handle(g\_a, string)

integer g\_a {[}input{]} character {*}({*}) string {[}input{]}

Check that the global array handle g\_a is valid ... if not call ga\_error
with the string provided and some more info. GA\_INIT\_FENCE

subroutine ga\_init\_fence()

Initializes tracing of completion status of data movement operations.
This is a local operation. GA\_FENCE

subroutine ga\_fence()

Blocks the calling process until all the data transfers corresponding
to GA operations called after ga\_init\_fence complete. For example,
since ga\_put might return before the data reaches the final destination
ga\_init\_fence and ga\_fence allow process to wait until the data
is actually moved:

call ga\_init\_fence() call ga\_put(g\_a, ...) call ga\_fence()

ga\_fence must be called after ga\_init\_fence. A barrier, ga\_sync,
assures completion of all data transfers and implicitly cancels outstandingga\_init\_fence.
ga\_init\_fence and ga\_fence must be used in pairs, multiple calls
to ga\_fence require the same number of corresponding ga\_init\_fence
calls. ga\_init\_fence/ga\_fence pairs can be nested.

ga\_fence works for multiple GA operations. For example:

call ga\_init\_fence() call ga\_put(g\_a, ...) call ga\_scatter(g\_a,
...) call ga\_put(g\_b, ...) call ga\_fence()

The calling process will be blocked until data movements initiated
by two calls to ga\_put and one ga\_scatter complete. GA\_CREATE\_MUTEXES

logical function ga\_create\_mutexes(number)

integer number {[}input{]}

Creates a set containing the number of mutexes. Returns .true. if
the opereation succeeded or .false. when failed. Mutex is simple synchronization
object used to protect Critical Section. Only one set of mutexes can
exist at a time. Mutexes can be created and destroyed as many times
as needed.

Mutexes are numbered: 0, ..., number -1.

This is a collective operation. GA\_DESTROY\_MUTEXES

logical function ga\_destroy\_mutexes()

Destroys the set of mutexes created with ga\_create\_mutexes. Returns
.true. if the operation succeeded or .false. when failed.

This is a collective operation. GA\_LOCK

subroutine ga\_lock(mutex)

integer mutex {[}input{]} ! mutex id

Locks a mutex object identified by the mutex number. It is a fatal
error for a process to attempt to lock a mutex which was already locked
by this process. GA\_UNLOCK

subroutine ga\_unlock(mutex) integer mutex {[}input{]} ! mutex id

Unlocks a mutex object identified by the mutex number. It is a fatal
error for a process to attempt to unlock a mutex which has not been
locked by this process. GA\_NODEID

integer function ga\_nodeid()

Returns the GA process id (0, ..., ga\_nnodes()-1) of the requesting
compute process. This is a local operation.

NOTE: the GA process id might not be the same as message-passing node
id. GA\_NNODES

integer function ga\_nnodes()

Returns the number of the GA compute (user) processes.

NOTE: the number of the GA processes might not be the same as the
number of the message-passing nodes in releases <3.0 of the GA package.
This is a local operation. GA\_BRDCST

subroutine ga\_brdcst(type, buf, lenbuf, root) integer type {[}input{]}
! message type for broadcast byte buf(lenbuf) {[}input/output{]} integer
lenbuf {[}input{]} integer root {[}input{]}

Broadcast from process root to all other processes a message of length
lenbuf.

This is a convenience operation available regardless of the message-passing
library that GA is running with.

This is a collective operation. GA\_DGOP

subroutine ga\_dgop(type, x, n, op) integer type {[}input{]} double
precision x(n) {[}input/output{]} character{*}({*}) op {[}input{]}

Double Global OPeration.

X(1:N) is a vector present on each process. DGOP 'sums' elements of
X accross all nodes using the commutative operator OP. The result
is broadcast to all nodes. Supported operations include '+', '{*}',
'max', 'min', 'absmax', 'absmin'. The use of lowerecase for operators
is necessary.

This is a convenience operation, available regardless of the message-passing
library that GA is running with.

This is a collective operation. GA\_IGOP

subroutine ga\_igop(type, x, n, op) integer type {[}input{]} integer
x(n) {[}input/output{]} character{*}({*}) OP {[}input{]}

Integer Global OPeration. The integer version of ga\_dgop described
above, also include the bitwise OR operation.

This is a convenience operation available regardless of the message-passing
library that GA is running with.

This is a collective operation. GA\_CLUSTER\_NNODES

integer function ga\_cluster\_nnodes()

This functions returns the total number of nodes that the program
is running on. On SMP architectures, this will be less than or equal
to the total number of processors.

This is a local operation. GA\_CLUSTER\_NODEID

integer function ga\_cluster\_nodeid()

This function returns the node ID of the process. On SMP architectures
with more than one processor per node, several processes may return
the same node id.

This is a local operation. GA\_CLUSTER\_PROC\_NODEID

integer function ga\_cluster\_proc\_nodeid(proc)

integer proc {[}input{]}

This function returns the node ID of the specified process proc. On
SMP architectures with more than one processor per node, several processes
may return the same node id.

This is a local operation. GA\_CLUSTER\_NPROCS

integer function ga\_cluster\_nprocs(inode)

integer inode {[}input{]}

This function returns the number of processors available on node inode.

This is a local operation. GA\_CLUSTER\_PROCID

integer function ga\_cluster\_procid(inode,iproc)

integer inode,iproc {[}input{]}

This function returns the processor id associated with node inode
and the local processor id iproc. If node inode has N processors,
then the value of iproc lies between 0 and N-1.

This is a local operation. GA\_LIST\_NODEID

subroutine ga\_list\_nodeid(list, n)

integer n {[}input{]} integer list(n) {[}output{]}

Returns message-passing process ID (rank in MPI) for the GA processes
in the range 0 .. n-1. For MPI, the ranks are in MPI\_COMM\_WORLD.
Obsolete as of release 3.0. GA\_MPI\_COMMUNICATOR

void ga\_mpi\_communicator(GA\_COMM) MPI\_Comm {*}GA\_COMM; {[}output{]}

Available in C only if MPI is used. Returns communicator handle for
GA processes (group). Obsolete as of release 3.0. GA\_ABS\_VALUE

subroutine ga\_abs\_value(g\_a)

integer g\_a - array handle {[}input{]}

Take element-wise absolute value of the array. This is a collective
operation. GA\_ABS\_VALUE\_PATCH

subroutine ga\_abs\_value\_patch(g\_a, lo, hi)

integer g\_a - array handle {[}input{]} integer lo(ndim), hi(ndim)
- g\_a patch coordinates {[}input{]} 

Take element-wise absolute value of the patch. This is a collective
operation. GA\_ADD\_CONSTANT

subroutine ga\_add\_constant(g\_a, alpha)

integer g\_a - array handle {[}input{]} double/complex/integer/float
alpha {[}input{]}

Add the contant pointed by alpha to each element of the array. This
is a collective operation. GA\_ADD\_CONSTANT\_PATCH

subroutine ga\_add\_constant\_patch(g\_a, lo, hi, alpha)

integer g\_a - array handle {[}input{]} integer ndim - number of dimensions
{[}input{]} integer lo(ndim), hi(ndim) - patch coordinates {[}input{]}
double/complex/integer/float alpha {[}input{]}

Add the contant pointed by alpha to each element of the patch. This
is a collective operation. GA\_RECIP

subroutine ga\_recip(g\_a)

integer g\_a - array handle {[}input{]} 

Take element-wise reciprocal of the array. This is a collective operation.
GA\_RECIP\_PATCH

subroutine ga\_recip\_patch(g\_a, lo, hi)

integer g\_a - array handle {[}input{]} integer ndim - number of dimensions
{[}input{]} integer lo(ndim), hi(ndim) - patch coordinates {[}input{]}

Take element-wise reciprocal of the patch. This is a collective operation.
GA\_ELEM\_MULTIPLY

subroutine ga\_elem\_multiply(g\_a, g\_b, g\_c)

integer g\_a, g\_b - array handles {[}input{]} integer g\_c - array
handle {[}output{]}

Computes the element-wise product of the two arrays which must be
of the same types and same number of elements. For two-dimensional
arrays,

c(i, j) = a(i,j){*}b(i,j)

The result (c) may replace one of the input arrays (a/b). This is
a collective operation. GA\_ELEM\_MULTIPLY\_PATCH

subroutine ga\_elem\_multiply\_patch(g\_a, alo, ahi, g\_b, blo, bhi,
g\_c, clo, chi)

integer g\_a, g\_b - array handles {[}input{]} integer g\_c - array
handle {[}output{]} integer ndim - number of dimensions {[}input{]}
integer alo(ndim), ahi(ndim) - g\_a patch dimensions {[}input{]} integer
blo(ndim), bhi(ndim) - g\_b patch dimensions {[}input{]} integer clo(ndim),
chi(ndim) - g\_c patch dimensions {[}input{]}

Computes the element-wise product of the two patches which must be
of the same types and same number of elements. For two-dimensional
arrays,

c(i, j) = a(i,j){*}b(i,j)

The result (c) may replace one of the input arrays (a/b). This is
a collective operation. GA\_ELEM\_DIVIDE

subroutine ga\_elem\_divide(g\_a, g\_b, g\_c)

integer g\_a, g\_b - array handles {[}input{]} integer g\_c - array
handle {[}output{]}

Computes the element-wise quotient of the two arrays which must be
of the same types and same number of elements. For two-dimensional
arrays,

c(i, j) = a(i,j)/b(i,j)

The result (c) may replace one of the input arrays (a/b). If one of
the elements of array g\_b is zero, the quotient for the element of
g\_c will be set to GA\_NEGATIVE\_INFINITY.

This is a collectiveoperation. GA\_ELEM\_DIVIDE\_PATCH

subroutine ga\_elem\_divide\_patch(g\_a, alo, ahi, g\_b, blo, bhi,
g\_c, clo, chi)

integer g\_a, g\_b - array handles {[}input{]} integer g\_c - array
handle {[}output{]} integer ndim - number of dimensions {[}input{]}
integer alo(ndim), ahi(ndim) - g\_a patch dimensions {[}input{]} integer
blo(ndim), bhi(ndim) - g\_b patch dimensions {[}input{]} integer clo(ndim),
chi(ndim) - g\_c patch dimensions {[}input{]}

Computes the element-wise quotient of the two patches which must be
of the same types and same number of elements. For two-dimensional
arrays,

c(i, j) = a(i,j)/b(i,j)

The result (c) may replace one of the input arrays (a/b). This is
a collective operation. GA\_ELEM\_MAXIMUM

subroutine ga\_elem\_maximum(g\_a, g\_b, g\_c)

integer g\_a, g\_b - array handles {[}input{]} integer g\_c - array
handle {[}output{]}

Computes the element-wise maximum of the two arrays which must be
of the same types and same number of elements. For two dimensional
arrays,

c(i, j) = max\{a(i,j), b(i,j)\}

The result (c) may replace one of the input arrays (a/b). This is
a collective operation. GA\_ELEM\_MAXIMUM\_PATCH

subroutine ga\_elem\_maximum\_patch(g\_a, alo, ahi, g\_b, blo, bhi,
g\_c, clo, chi)

integer g\_a, g\_b - array handles {[}input{]} integer g\_c - array
handle {[}output{]} integer ndim - number of dimensions {[}input{]}
integer alo(ndim), ahi(ndim) - g\_a patch dimensions {[}input{]} integer
blo(ndim), bhi(ndim) - g\_b patch dimensions {[}input{]} integer clo(ndim),
chi(ndim) - g\_c patch dimensions {[}input{]}

Computes the element-wise maximum of the two patches which must be
of the same types and same number of elements. For two-dimensional
of noncomplex arrays,

c(i, j) = max\{a(i,j), b(i,j)\}

If the data type is complex, then c(i, j).real = max\{ |a(i,j)|, |b(i,j)|\}
while c(i,j).image = 0.

The result (c) may replace one of the input arrays (a/b). This is
a collective operation. GA\_ELEM\_MINIMUM

subroutine ga\_elem\_minimum(g\_a, g\_b, g\_c)

integer g\_a, g\_b - array handles {[}input{]} integer g\_c - array
handle {[}output{]}

Computes the element-wise minimum of the two arrays which must be
of the same types and same number of elements. For two dimensional
arrays,

c(i, j) = min\{a(i,j), b(i,j)\}

The result (c) may replace one of the input arrays (a/b). This is
a collective operation. GA\_ELEM\_MINIMUM\_PATCH

subroutine ga\_elem\_minimum\_patch(g\_a, alo, ahi, g\_b, blo, bhi,
g\_c, clo, chi)

integer g\_a, g\_b - array handles {[}input{]} integer g\_c - array
handle {[}output{]} integer ndim - number of dimensions {[}input{]}
integer alo(ndim), ahi(ndim) - g\_a patch dimensions {[}input{]} integer
blo(ndim), bhi(ndim) - g\_b patch dimensions {[}input{]} integer clo(ndim),
chi(ndim) - g\_c patch dimensions {[}input{]}

Computes the element-wise minimum of the two patches which must be
of the same types and same number of elements. For two-dimensional
of noncomplex arrays,

c(i, j) = min\{a(i,j), b(i,j)\}

If the data type is complex, then c(i, j).real = min\{ |a(i,j)|, |b(i,j)|\}
while c(i,j).image = 0.

The result (c) may replace one of the input arrays (a/b). This is
a collective operation. GA\_SHIFT\_DIAGONAL

subroutine ga\_shift\_diagonal(g\_a, c)

integer g\_a - array handle {[}input{]} double/complex/integer/float
{[}input{]}

Adds this constant to the diagonal elements of the matrix. This is
a collective operation. GA\_SET\_DIAGONAL

subroutine ga\_set\_diagonal(g\_a, g\_v)

integer g\_a, g\_v - array handles {[}input{]}

Sets the diagonal elements of this matrix g\_a with the elements of
the vector g\_v. This is a collective operation. GA\_ZERO\_DIAGONAL

subroutine ga\_zero\_diagonal(g\_a)

integer g\_a - array handle {[}input{]}

Sets the diagonal elements of this matrix g\_a with zeros. This is
a collective operation. GA\_ADD\_DIAGONAL

subroutine ga\_add\_diagonal(g\_a, g\_v)

integer g\_a, g\_v - array handles {[}input{]}

Adds the elements of the vector g\_v to the diagonal of this matrix
g\_a. This is a collective operation. GA\_GET\_DIAG

subroutine ga\_get\_diag(g\_a, g\_v)

integer g\_a - array handle {[}input{]} integer g\_v - array handle
{[}input{]}

Inserts the diagonal elements of this matrix g\_a into the vector
g\_v. This is a collective operation. GA\_SCALE\_ROWS

subroutine ga\_scale\_rows(g\_a, g\_v)

integer g\_a, g\_v - array handles {[}input{]}

Scales the rows of this matrix g\_a using the vector g\_v. This is
a collective operation. GA\_SCALE\_COLS

subroutine ga\_scale\_cols(g\_a, g\_v)

integer g\_a, g\_v - array handles {[}input{]}

Scales the columns of this matrix g\_a using the vector g\_v. This
is a collective operation. GA\_NORM1

subroutine ga\_norm1(g\_a, nm)

integer g\_a - array handle {[}input{]} double precision nm - matrix/vector
1-norm value {[}output{]}

Computes the 1-norm of the matrix or vector g\_a. This is a collective
operation. GA\_NORM\_INFINITY

subroutine ga\_norm\_infinity(g\_a, nm)

integer g\_a - array handle {[}input{]} double precision nm - matrix/vector
infinity-norm value {[}output{]}

Computes the infinity-norm of the matrix or vector g\_a. This is a
collective operation. GA\_MEDIAN

subroutine ga\_median(g\_a, g\_b, g\_c, g\_m)

integer g\_a, g\_b, g\_c - array handles {[}input{]} integer g\_m
- array handle {[}output{]}

Computes the componentwise median of three arrays g\_a, g\_b, and
g\_c, and stores the result in this array g\_m. The result (m) may
replace one of the input arrays (a/b/c). This is a collective operation.
GA\_MEDIAN\_PATCH

subroutine ga\_median\_patch(g\_a,alo,ahi,g\_b,blo,bhi,g\_c,clo,chi,g\_m,mlo,mhi)

integer g\_a, g\_b, g\_c - array handles {[}input{]} integer g\_m
- array handle {[}output{]} integer ndim - number of dimensions {[}input{]}
integer alo(ndim), ahi(ndim) - g\_a patch dimensions {[}input{]} integer
blo(ndim), bhi(ndim) - g\_b patch dimensions {[}input{]} integer clo(ndim),
chi(ndim) - g\_c patch dimensions {[}input{]} integer mlo(ndim), mhi(ndim)
- g\_m patch dimensions {[}input{]}

Computes the componentwise median of three patches g\_a, g\_b, and
g\_c, and stores the result in this patch g\_m. The result (m) may
replace one of the input patches (a/b/c). This is a collective operation.
GA\_STEP\_MAX

subroutine ga\_step\_max(g\_a, g\_b, step)

integer g\_a, g\_b - array handles {[}input{]} double precision step
- the maximum step {[}output{]}

Calculates the largest multiple of a vector g\_b that can be added
to this vector g\_a while keeping each element of this vector nonnegative.
This is a collective operation. GA\_STEP\_MAX2

subroutine ga\_step\_max2(g\_xx, g\_vv, g\_xxll, g\_xxuu, step2)

integer g\_xx, g\_vv, g\_xxll, g\_xxuu - array handles {[}input{]}
double precision step2 - the maximum step size {[}output{]}

where

g\_vv - the step direction g\_xxll - lower bounds g\_xxuu - upper
bounds

Calculates the largest step size that should be used in a projected
bound line search. This is a collective operation. GA\_STEP\_MAX\_PATCH

subroutine ga\_step\_max\_patch(g\_a, alo, ahi, g\_b, blo, bhi, step)

integer g\_a, g\_b - array handles where g\_b is step direction {[}input{]}
integer alo, ahi, blo, bhi - patch coordinates of g\_a and g\_b {[}input{]}
double precision step - the maximum step {[}output{]}

Calculates the largest multiple of a vector g\_b that can be added
to this vector g\_a while keeping each element of this vector nonnegative.
This is a collective operation. GA\_STEP\_MAX2\_PATCH

subroutine ga\_step\_max2\_patch(g\_xx, xxlo, xxhi, g\_vv, vvlo, vvhi,
g\_xxll, xxlllo, xxllhi, g\_xxuu, xxuulo, xxuuhi, step2)

integer g\_xx, g\_vv, g\_xxll, g\_xxuu - array handles {[}input{]}
integer xxlo, xxhi, vvlo, vvhi - patch coordinates {[}input{]} integer
xxlllo, xxllhi, xxuulo, xxuuhi - patch coordinates {[}input{]} double
precision step2 - the maximum step size {[}output{]}

where g\_vv - the step direction g\_xxll - lower bounds g\_xxuu -
upper bounds

Calculates the largest step size that should be used in a projected
bound line search. This is a collective operation. GA\_PGROUP\_GET\_DEFAULT

integer function ga\_pgroup\_get\_default()

This function will return a handle to the default processor group
which can then be used to create a global array using one of the nga\_create\_{*}\_config
or ga\_set\_pgroup calls.

This is a local operation. GA\_PGROUP\_GET\_MIRROR

integer function ga\_pgroup\_get\_mirror()

This function will return a handle to the mirrored processor group,
which can then be used to create a mirrored global array using one
of the nga\_create\_{*}\_config or ga\_set\_pgroup calls.

This is a local operation. GA\_PGROUP\_GET\_WORLD

integer function ga\_pgroup\_get\_world()

This function will return a handle to the world group, i.e. the group
containing all processors that the code is running on. This function
can be used to get access to the world group if the default group
has been set to a subset of processors.

This is a local operation. GA\_PGROUP\_SYNC

subroutine ga\_pgroup\_sync(p\_handle) integer p\_handle - processor
group handle {[}input{]}

This operation executes a synchronization group across the processors
in the processor group specified by p\_handle. Nodes outside this
group are unaffected.

This is a collective operation on the processor group specified by
p\_handle. GA\_PGROUP\_BRDCST

subroutine ga\_pgroup\_brdcst(p\_handle, type, buf, lenbuf, root)
integer p\_handle - processor group handle {[}input{]} integer type
- message index {[}input{]} byte buf(lenbuf) - local message buffer
{[}input/output{]} integer lenbuf - length of message {[}input{]}
integer root - processor sending message {[}input{]}

Broadcast data from processor specified by root to all other processors
in the processor group specified by p\_handle. The length of the message
in bytes is specified by lenbuf. Type is an arbitrary index assigned
to each call to distinguish if from other broadcast calls.

This is a collective operation on the processor group specified by
p\_handle. GA\_PGROUP\_DGOP

subroutine ga\_pgroup\_dgop(p\_handle, type, buf, n, op) integer p\_handle
- processor group handle {[}input{]} integer type - message index
{[}input{]} double precision buf(n) - double precision array {[}input/output{]}
integer n - number elements in array {[}input{]} character{*}({*})
op - operation on data {[}input{]}

buf(n) is a double precision array present on each processor in the
processor group p\_handle. The ga\_pgroup\_dgop 'sums' all elements
in buf(n) across all processors in the group specified by p\_handle
using the commutative operation specified by the character string
op. The result is broadcast to all processor in p\_handle. Allowed
strings are '+', '{*}', 'max', 'min', 'absmax', 'absmin'. The use
of lowerecase for operators is necessary.

This is a collective operation on the processor group specifed by
p\_handle. GA\_PGROUP\_IGOP

subroutine ga\_pgroup\_igop(p\_handle, type, buf, n, op) integer p\_handle
- processor group handle {[}input{]} integer type - message index
{[}input{]} integer buf(n) - integer array {[}input/output{]} integer
n - number elements in array {[}input{]} character{*}({*}) op - operation
on data {[}input{]}

buf(n) is an integer array present on each processor in the processor
group p\_handle. The ga\_pgroup\_igop 'sums' all elements in buf(n)
across all processors in the group specified by p\_handle using the
commutative operation specified by the character string op. The result
is broadcast to all processor in p\_handle. Allowed strings are '+',
'{*}', 'max', 'min', 'absmax', 'absmin'. The use of lowerecase for
operators is necessary.

This is a collective operation on the processor group specified by
p\_handle. GA\_PGROUP\_SGOP

subroutine ga\_pgroup\_sgop(p\_handle, type, buf, n, op) integer p\_handle
- processor group handle {[}input{]} integer type - message index
{[}input{]} real buf(n) - single precision array {[}input/output{]}
integer n - number elements in array {[}input{]} character{*}({*})
op - operation on data {[}input{]}

buf(n) is a single precsion array present on each processor in the
processor group p\_handle. The ga\_pgroup\_sgop 'sums' all elements
in buf(n) across all processors in the group specified by p\_handle
using the commutative operation specified by the character string
op. The result is broadcast to all processor in p\_handle. Allowed
strings are '+', '{*}', 'max', 'min', 'absmax', 'absmin'. The use
of lowerecase for operators is necessary.

This is a collective operation on the processor group specified by
p\_handle. GA\_PGROUP\_NNODES

integer function ga\_pgroup\_nnodes(p\_handle) integer p\_handle -
processor group handle {[}input{]}

This function returns the number of processors contained in the group
specifed by p\_handle.

This is a local operation. GA\_PGROUP\_NODEID

integer function ga\_pgroup\_nodeid(p\_handle) integer p\_handle -
processor group handle {[}input{]}

This function returns the relative index of the processor in the processor
group specified by p\_handle. This index will generally differ from
the absolute processor index returned by ga\_nodeid if the processor
group is not the world group.

This is a local operation.

GA\_MERGE\_MIRRORED

subroutine ga\_merge\_mirrored(g\_a) integer g\_a {[}input{]}

This subroutine merges mirrored arrays by adding the contents of each
array across nodes. The result is that the each mirrored copy of the
array represented by g\_a is the sum of the individual arrays before
the merge operation. After the merge, all mirrored arrays are equal.

This is a collective operation.

GA\_IS\_MIRRORED

integer ga\_is\_mirrored(g\_a) integer g\_a array handle {[}input{]}

This subroutine checks if the array is mirrored array or not. Returns
1 if it is a mirrored array, else returns 0.

This is a local operation.

GA\_MERGE\_DISTR\_PATCH

integer nga\_merge\_distr\_patch(g\_a, alo, ahi, g\_b, blo, bhi)

integer g\_a, g\_b - array handles {[}input{]} integer ndim - number
of dimensions of the global array integer alo(ndim), ahi(ndim) - g\_a
patch coordinates {[}input{]} integer blo(ndim), bhi(ndim) - g\_b
patch coordinates {[}input{]}

This function merges all copies of a patch of a mirrored array (g\_a)
into a patch in a distributed array (g\_b).

This is a collective operation. GA\_NBGET

subroutine ga\_nbget(g\_a, ilo, ihi, jlo, jhi, buf, ld, nbhandle)

integer g\_a - global array handle {[}input{]} integer ndim - number
of dimensions of the global array integer ilo, ihi - starting indices
for global array section {[}input{]} integer jlo, jhi - ending indices
for global array section {[}input{]} <type> buf - local buffer array
where the data goes {[}output{]} integer ld - leading dimension/stride/extent
for buffer array {[}input{]} integer nbhandle - non-blocking request
handle {[}input{]}

Non-blocking version of the blocking get operation. The get operation
can be completed locally by making a call to the wait (e.g.NGA\_NbWait)
routine.

This is a non-blocking one-sided operation.

subroutine nga\_nbget(g\_a, lo, hi, buf, ld, nbhandle)

integer g\_a - global array handle {[}input{]} integer ndim - number
of dimensions of the global array integer lo{[}ndim{]} - array of
starting indices for global array section {[}input{]} integer hi{[}ndim{]}
- array of ending indices for global array section {[}input{]} <type>
buf - local buffer array where the data goes {[}output{]} integer
ld{[}ndim-1{]} - array specifying leading dimensions/strides/extents
for buffer array {[}input{]} integer nbhandle - non-blocking request
handle {[}input{]}

Non-blocking version of the blocking get operation. The get operation
can be completed locally by making a call to the wait (e.g.NGA\_NbWait)
routine.

This is a non-blocking one-sided operation. GA\_NBPUT

subroutine ga\_nbput(g\_a, ilo, ihi, jlo, jhi, buf, ld, nbhandle)

integer g\_a - global array handle {[}input{]} integer ilo, ihi -
starting indices for global array section {[}input{]} integer jlo,
jhi - ending indices for global array section {[}input{]} <type>buf
- local buffer array where the data is {[}input{]} integer ld - leading
dimension/stride/extent for buffer array {[}input{]} integer nbhandle
- non-blocking request handle {[}input{]}

Non-blocking version of the blocking put operation. The put operation
can be completed locally by making a call to the wait (e.g.NGA\_NbWait)
routine.

This is a non-blocking one-sided operation.

subroutine nga\_nbput(g\_a, lo, hi, buf, ld, nbhandle)

integer g\_a - global array handle {[}input{]} integer ndim - number
of dimensions of the global array integer lo{[}ndim{]} - array of
starting indices for global array section {[}input{]} integer hi{[}ndim{]}
- array of ending indices for global array section {[}input{]} <type>buf
- local buffer array where the data is {[}input{]} integer ld{[}ndim-1{]}
- array specifying leading dimensions/strides/extents for buffer array
{[}input{]} integer nbhandle - non-blocking request handle {[}input{]}

Non-blocking version of the blocking put operation. The put operation
can be completed locally by making a call to the wait (e.g.NGA\_NbWait)
routine.

This is a non-blocking one-sided operation. GA\_NBACC

subroutine ga\_nbacc(g\_a, ilo, ihi, jlo, jhi, buf, ld, alpha, nbhandle)

integer g\_a - global array handle {[}input{]} integer ilo, ihi -
starting indices for global array section {[}input{]} integer jlo,
jhi - ending indices for global array section {[}input{]} <type>buf
- local buffer array where the data is {[}input{]} integer ld - leading
dimension/stride/extent for buffer array {[}input{]} double precision/complex
alpha - scale factor {[}input{]} integer nbhandle - non-blocking request
handle {[}input{]}

Non-blocking version of the blocking accumulate operation. The accumulate
operation can be completed locally by making a call to the wait (e.g.NGA\_NbWait)
routine.

This is a non-blocking one-sided operation.

subroutine nga\_nbacc(g\_a, lo, hi, buf, ld, alpha, nbhandle)

integer g\_a - global array handle {[}input{]} integer ndim - number
of dimensions of the global array integer lo{[}ndim{]} - array of
starting indices for global array section {[}input{]} integer hi{[}ndim{]}
- array of ending indices for global array section {[}input{]} <type>buf
- local buffer array where the data is {[}input{]} integer ld{[}ndim-1{]}
- array specifying leading dimensions/strides/extents for buffer array
{[}input{]} double precision/complex alpha - scale factor {[}input{]}
integer nbhandle - non-blocking request handle {[}input{]}

Non-blocking version of the blocking accumulate operation. The accumulate
operation can be completed locally by making a call to the wait (e.g.NGA\_NbWait)
routine.

This is a non-blocking one-sided operation. GA\_NBWAIT

subroutine ga\_nbwait(nbhandle)

integer nbhandle - non-blocking request handle {[}input{]}

This function completes a non-blocking one-sided operation locally.
Waiting on a nonblocking put or an accumulate operation assures that
data was injected into the network and the user buffer can be now
be reused. Completing a get operation assures data has arrived into
the user memory and is ready for use. Wait operation ensures only
local completion. Unlike their blocking counterparts, the nonblocking
operations are not ordered with respect to the destination. Performance
being one reason, the other reason is that by ensuring ordering we
incur additional and possibly unnecessary overhead on applications
that do not require their operations to be ordered. For cases where
ordering is necessary, it can be done by calling a fence operation.
The fence operation is provided to the user to confirm remote completion
if needed.

subroutine nga\_nbwait(nbhandle)

integer nbhandle - non-blocking request handle {[}input{]}

This function completes a non-blocking one-sided operation locally.
Waiting on a nonblocking put or an accumulate operation assures that
data was injected into the network and the user buffer can be now
be reused. Completing a get operation assures data has arrived into
the user memory and is ready for use. Wait operation ensures only
local completion. Unlike their blocking counterparts, the nonblocking
operations are not ordered with respect to the destination. Performance
being one reason, the other reason is that by ensuring ordering we
incur additional and possibly unnecessary overhead on applications
that do not require their operations to be ordered. For cases where
ordering is necessary, it can be done by calling a fence operation.
The fence operation is provided to the user to confirm remote completion
if needed. GA\_WTIME

double precision function ga\_wtime()

This function return a wall (or elapsed) time on the calling processor.
Returns time in seconds representing elapsed wall-clock time since
an arbitrary time in the past. Example:

double precision starttime, endtime starttime = GA\_Wtime() .... code
snippet to be timed .... endtime = GA\_Wtime() print {*}, 'Time taken
(seconds) = ', endtime-starttime

This is a local operation. This function is only available in release
4.1 or greater. GA\_SET\_DEBUG

subroutine ga\_set\_debug(flag)

logical flag - value to set internal debug flag {[}input{]}

This function sets an internal flag in the GA library to either true
or false. The value of this flag can be recovered at any time using
the ga\_get\_debug function. The flag is set to false when the the
GA library is initialized. This can be useful in a number of debugging
situations, especially when examining the behavior of routines that
are called in multiple locations in a code.

This is a local operation. GA\_GET\_DEBUG

logical function ga\_get\_debug()

This function returns the value of an internal flag in the GA library
whose value can be set using the ga\_set\_debug subroutine.

This is a local operation GA\_PATCH\_ENUM

subroutine ga\_patch\_enum(g\_a, lo, hi, istart, inc)

integer g\_a - array handle {[}input/output{]} integer lo, hi - low
and high values of array patch {[}input{]} integer istart - starting
value of enumeration {[}input{]} integer inc - increment value {[}input{]}

This subroutine enumerates the values of an array between elements
lo and hi starting with the value istart and incrementing each subsequent
value by inc. This operation is only applicable to 1-dimensional arrays.
An example of its use is shown below:

call ga\_patch\_enum(g\_a, 1, n, 7, 2) g\_a: 7 9 11 13 15 17 19 21
23 ...

This is a collective operation. GA\_SCAN\_ADD

subroutine ga\_scan\_add(g\_src, g\_dest, g\_mask, lo, hi, excl)

integer g\_src - handle for source array {[}input{]} integer g\_dest
- handle for destination array {[}output{]} integer g\_mask - handle
for integer array representing a bit-mask {[}input{]} integer lo,
hi - low and high values of range on which operation is performed
{[}input{]} integer excl - value to signify if masked values are included
in add {[}input{]}

This operation will add successive elements in a source vector g\_src
and put the results in a destination vector g\_dest. The addition
will restart based on the values of the integer mask vector g\_mask.
The scan is performed within the range specified by the integer values
lo and hi. Note that this operation can only be applied to 1-dimensional
arrays. The excl flag determines whether the sum starts with the value
in the source vector corresponding to the location of a 1 in the mask
vector (excl=0) or whether the first value is set equal to 0 (excl=1).
Some examples of this operation are given below.

call ga\_scan\_add(g\_src, g\_dest, g\_mask, 1, n, 0) g\_mask: 1 0
0 0 0 0 1 0 1 0 0 1 0 0 1 1 0 g\_src: 1 2 3 4 5 6 7 8 9 10 11 12 13
14 15 16 17 g\_dest: 1 3 6 10 16 21 7 15 9 19 30 12 25 39 15 16 33

call ga\_scan\_add(g\_src, g\_dest, g\_mask, 1, n, 1) g\_mask: 1 0
0 0 0 0 1 0 1 0 0 1 0 0 1 1 0 g\_src: 1 2 3 4 5 6 7 8 9 10 11 12 13
14 15 16 17 g\_dest: 0 1 3 6 10 15 0 7 0 9 19 0 12 25 0 0 16

This is a collective operation. GA\_SCAN\_COPY

subroutine ga\_scan\_copy(g\_src, g\_dest, g\_mask, lo, hi)

integer g\_src - handle for source array {[}input{]} integer g\_dest
- handle for destination array {[}output{]} integer g\_mask - handle
for integer array representing a bit-mask {[}input{]} integer lo,
hi - low and high values of range on which operation is performed
{[}input{]}

This subroutine does a segmented scan-copy of values in the source
array g\_src into a destination array g\_dest with segments defined
by values in the integer mask array g\_mask. The scan-copy operation
is only applied to the range between the lo and hi indices. This operation
is restriced to 1-dimensional arrays. The resulting destination array
will consist of segments of consecutive elements with the same value.
An example is shown below

call ga\_scan\_copy(g\_src, g\_dest, g\_mask, 1, n) g\_mask: 1 0 0
0 0 0 1 0 1 0 0 1 0 0 1 1 0 g\_src: 1 2 3 4 5 6 7 8 9 10 11 12 13
14 15 16 17 g\_dest: 1 1 1 1 1 1 7 7 9 9 9 12 12 12 15 16 16

This is a collective operation. GA\_PACK

subroutine ga\_pack(g\_src, g\_dest, g\_mask, lo, hi, icount)

integer g\_src - handle for source array {[}input{]} integer g\_dest
- handle for destination array {[}output{]} integer g\_mask - handle
for integer array representing a bit-mask {[}input{]} integer lo,
hi - low and high values of range on which operation is performed
{[}input{]} integer icount - number of values in compressed array
{[}output{]}

The pack subroutine is designed to compress the values in the source
vector g\_src into a smaller destination array g\_dest based on the
values in an integer mask array g\_mask. The values lo and hi denote
the range of elements that should be compressed and icount is a variable
that on output lists the number of values placed in the compressed
array. This operation is the complement of the ga\_unpack operation.
An example is shown below

call ga\_pack(g\_src, g\_dest, g\_mask, 1, n, icount) g\_mask: 1 0
0 0 0 0 1 0 1 0 0 1 0 0 1 1 0 g\_src: 1 2 3 4 5 6 7 8 9 10 11 12 13
14 15 16 17 g\_dest: 1 7 9 12 15 16 icount: 6

This is a collective operation. GA\_UNPACK

subroutine ga\_unpack(g\_src, g\_dest, g\_mask, lo, hi, icount)

integer g\_src - handle for source array {[}input{]} integer g\_dest
- handle for destination array {[}output{]} integer g\_mask - handle
for integer array representing a bit-mask {[}input{]} integer lo,
hi - low and high values of range on which operation is performed
{[}input{]} integer icount - number of values in uncompressed array
{[}output{]}

The unpack subroutine is designed to expand the values in the source
vector g\_src into a larger destination array g\_dest based on the
values in an integer mask array g\_mask. The values lo and hi denote
the range of elements that should be compressed and icount is a variable
that on output lists the number of values placed in the uncompressed
array. This operation is the complement of the ga\_pack operation.
An example is shown below

call ga\_unpack(g\_src, g\_dest, g\_mask, 1, n, icount) g\_src: 1
7 9 12 15 16 g\_mask: 1 0 0 0 0 0 1 0 1 0 0 1 0 0 1 1 0 g\_dest: 1
0 0 0 0 0 7 0 9 0 0 12 0 0 15 16 0 icount: 6

This is a collective operation. Additional Explanations Global arrays
are distributed array objects supported in a message-passing program
through the GA library calls. They can be created, destroyed and manipulated
using a set of GA operations. Attributes of GA operations one-sided/independent
- references shared or remote data without remote process cooperation
(unlike send/receive pair) collective - requires all processes to
make the call local - operation is local to each process and does
not require communication atomic - operation has mutual exclusion
built in non-blocking - Nonblocking operations initiate a communication
call and then return control to the application. A return from a nonblocking
operation call indicates a mere initiation of the data transfer process
and the operation can be completed locally by making a call to the
wait (e.g. nga\_nbwait) routine. Language Interoperability GA provides
C and Fortran interfaces to the same array objects in the same (mixed-language)
program. /Higher-dimensional Array API All releases of GA < 3.0 supported
only 2-dimensional arrays. Starting with the release 3.0 GA arrays
can be be essentially arbitrary dimensional but the package as compiled
supports up to 7 dimensions (Fortran limit for multidimensional arrays).
The 2-D API for backward compatibility with older versions of the
toolkit remained unchanged. For some GA operations, two versions of
the API are available. They are identified with \textquotedbl{}GA\textquotedbl{}
or \textquotedbl{}NGA\textquotedbl{} prefix that correspond to either
the old 2-dimensional or the new arbitrary(N)-dimensional version,
respectively. 

\chapter{Introduction}

\section{Overview}

The Global Pointer Arrays (GP) toolkit is an extension of the Global Arrays
(GA) toolkit that is designed to support much more general distributed data
structures than GA. Instead of creating distributed arrays of elements, where
each element represents the same type of data object, GP arrays support
distributed arrays of ``pointers'', where each element can be an object of
arbitrary size. Once created, however, elements can be accessed from any
processor using the same kind of \emph{put} and \emph{get} symmantics that are familiar from
the GA toolkit.

Like GA, the GP toolkit uses a onesided communication model to move data from
the distributed GP array to local buffers. These operations consist of basic
\emph{get}, \emph{put}, \emph{gather} and \emph{scatter} operations that can be used to
move blocks of data from remote processors to and from local buffers.

The basic functionality of the GP toolkit is similar to that of GA. The user
creates a distributed GP array by specifying the dimensions of the GP array. It
is also possible to specify how the data is distributed by setting variables
such as the chunk size or using the irregular distribution functionality. Unlike
conventional GAs, the user cannot specify data type since this concept is
meaningless for a GP array.

Once created, the GP array needs to be attached to data. This can be done by
having each process identify the set of pointers in the GP array that are local
to the process, allocating a local memory segment that contains the data that
will go in the GP at that location, and assigning the GP pointer to the
allocated memory segment. Once all array elements have been assigned, it is then
possible for any process to access or modify the contents of any array element
using \emph{get} and \emph{put} operations to copy data between the GP array and a local
buffer.

Because the size of individual array elements is arbitrary, more information
about the size of each array element needs to be communicated to the application
than for ordinary GAs. Because of this the interfaces for onesided GP operations
are more complicated than the corresponding operations in the GA toolkit.
Another consequence of having arbitrary array elements is that operations like
addition and multiplication are not defined, so many operations that are
available in the GA toolkit, such as accumulate, add, scale, etc. have no
counterpart in the GP toolkit.

As with GA, GP divides logically shared data structures into ``local'' and ``remote''
portions. It recognizes variable data transfer costs required to access the
data depending on the proximity attributes. A local portion of the shared
memory is assumed to be faster to access and the remainder (remote portion) is
considered slower to access. These differences do not hinder the ease-of-use
since the library provides uniform access mechanisms for all the shared data
regardless where the referenced data is located. In addition, any processes can
access a local portion of the shared data directly/in-place like any other data
in process local memory. Access to other portions of the shared data must be
done through the GP library calls. 

The GP toolkit is currently implemented as a library with C bindings.

\section{Basic Functionality}

The basic shared memory functions supported in the GP toolkit include \emph{get},
\emph{put}, \emph{scatter} and \emph{gather}. Unlike GA, there are no accumulate
or atomic read-and-increment functions, since algebraic operations for generalized array
elements are not defined. These functions can only be used to access data in global
pointer arrays rather than arbitrary memory locations. At least one global
pointer array has to be created before data transfer operations can be used.
These GP functions are one-sided/unilateral and will complete regardless of actions
taken by the remote process(es) that own(s) the referenced data. In particular, like GA,
GP does not offer or rely on a polling operation or require inserting any other GP library
calls to assure communication progress on the remote side. 

The GP data transfer operations use an array index-based interface rather than
addresses of the shared data.  The
higher level array oriented API (application programming interface) makes GP
easier to use, without compromising data locality control. The
library internally performs global array index-to-address translation and then
transfers data between appropriate processes.
The supported array dimensions range from one to seven. This limit follows the
Fortran convention. Because the amount of data represented by a request is
usually unknown, \emph{a priori}, GP provides functionality for determining how
large a data request is likely to be before the user actually acquires the data.
This allows the program to allocate buffers that are large enough to accomodate
the request without statically providing oversized buffers that may waste
memory.

\chapter{Sparse Array Operations}

Global Arrays provides some capability for creating and manipulating sparse 2D
arrays. This includes functionality for creating a 2D array, basic operations
for extracting and setting array diagonals, basic linear algebra operations for
sparse matrix-vector muliply and matrix-matrix multiply and additional
operations for directly accessing the data inside the sparse array. These
operations are sufficient to allow users to build basic linear solvers, such as
conjugate gradient and BiCGSTAB and to support investigations into sparse data
storage and data manipulation.

The sparse array functionality in GA is accessible from both the C and Fortran
APIs, although at present date only the C-interface has been extensively tested.

\section{Creating a Sparse Array}
The model for creating a sparse array is similar to other packages that create
sparse data structures, such as PETSc and Hypre. The sparse array is initially
created using a call to \texttt{NGA\_Sprs\_array\_create(int idim, int jdim, int
type)}, where \texttt{idim} is the row dimension of the sparse array,
\texttt{jdim} is the column dimension of the sparse array and \texttt{type} is
the GA datatype of the array. This function is also available in a version that
supports 64 bit indexing \texttt{NGA\_Sprs\_array\_create64(int64\_t idim,
int64\_t jdim, int64\_t type)}.  \footnote{If a sparse array is created using the
64 bit interface, then the GA library should be build with 64 bit index support.}
The function returns an integer handle that can be used to identify the array
in subsequent array operations.  Sparse arrays in application can typically end
up being
much larger than dense arrays, so the option to use 64 bit indexing is more
important here. If a sparse array is created with the 64 bit interface, then all
other operations on the array should also use the 64 bit versions of the
operation, if they are available. Some internal data structures containing the
locations of the non-zero values of the sparse array elements are sized to 64
bits when the 64 bit interface is used and other operation may fail if the wrong
interfaces is used on the array.
 
Once the sparse array has been created, the non-zero elements can be added to
the array with the \texttt{NGA\_Sprs\_array\_add\_element(int s\_a, int idx, int jdx,
void *val)} or \texttt{NGA\_Sprs\_array\_add\_element64(int s\_a, int64\_t idx,
int64\_t jdx, void *val)} function. The handle \texttt{s\_a} identifies the
sparse array,
\texttt{idx} and \texttt{jdx} are the row and column indices and \texttt{val}
is the value of the array element.  Any process can add any element to the
array, the only restriction is that an array element can only be added once. It
is a good idea to make sure that all diagonal elements of the array are added,
even if they happen to be zero. Some array operations may fail if some diagonal
entries are missing.

Finally, once all elements have been added to the array, it is time to assemble
the array into a final form that can be used for computation by calling the
\texttt{NGA\_Sprs\_array\_assemble(int s\_a)} function. This function sorts
individual elements onto processors and further organizes the data within
processors so that it is ready for computation. The final data layout is
illustrated in Figure \ref{fig:layout}. The row dimension of the sparse array is
divided into roughly equal sized segments and all the elements with a row index
falling within a row segment are assigned to the corresponding processor. The
column dimension is partitioned in the same way and the row block on each
processor is divided into blocks of data corresponding to the column partition.
Nominally, the number of blocks of data corresponds to the number of processors,
but in many applications, the column blocks may be empty. If the block is empty,
then no memory is allocated for that block. Each of the non-zero blocks is stored
using a compressed sparse row format (CSR). This data may be accessed directly
by the process hosting the data or it can be retrieved from a remote locate and
copied into local buffers. The overall data layout of the sparse matrix can be
characterized as a sparse array of sparse blocks.

\begin{figure}
  \centering
  \includegraphics*[width=5in, keepaspectratio=true]{sprs-layout}
  \caption{Schematic layout of sparse matrix data distribution. Original
matrix is partitioned along the row axis and each partition is assigned to a
process. Data within a row block is further partitioned along the column axis
into sparse blocks. Final data structure is a sparse array of sparse blocks.}
  \label{fig:layout}
\end{figure}

Alternatively, it is possible to create a new sparse array from an existing
sparse array using the function \texttt{NGA\_Sprs\_array\_duplicate(int s\_a)}
which returns a handle to a new sparse array that is a duplicate of the original
array \texttt{s\_a}. The new array can then be modified using some of the
functions described in the next section.

Finally, when you are done with array, it can be removed from the system with
the command \texttt{NGA\_Sprs\_array\_destroy(int s\_a)}. This will free up all
resources associated with the array.

\section{Collective Operations on Sparse Arrays}

Once a sparse array has been created, it can be used in several collective
operations on distributed data. The most important of these is sparse
matrix-vector multiplication, where the vector is an ordinary global array of
dimension 1. This operation is of the form
\texttt{NGA\_Sprs\_array\_matvec\_multiply(int s\_a, int g\_x, int g\_b)}
and corresponds to the multiplication $\overline{\overline{A}}\cdot\overline{x}
=\overline{b}$. The handles \texttt{g\_x} and \texttt{g\_b} are ordinary global
arrays of dimension 1. Note that the length of \texttt{g\_x} and \texttt{g\_b}
must match the column dimension of the sparse matrix \texttt{s\_a}. Matrix-vector
multiplication is a major operation used in iterative solvers.

Many algorithms involving sparse matrices manipulate the diagonal,
and some operations are included that enable users to extract and modify
diagonal values. Note that these operations are very likely to fail if not all
diagonal values have been included in the sparse matrix, so it is generatlly a
good idea to include diagonal values in the sparse array even if they are zero.
The first operation extracts the diagonal values and stores them in a
distributed 1 dimensional global array \texttt{NGA\_Sprs\_array\_get\_diag(int
s\_a, int *g\_d)} where \texttt{g\_d} is an ordinary global array of dimension
1.  The diagonals can also be shifted by a constant value using the function
\texttt{NGA\_Sprs\_array\_shift\_diag(int s\_a, void *shift)} where the variable
\texttt{shift} is the amount the diagonal should be incremented.

Other operations include left and right multiplication by a diagonal matrix. The
diagonal matrix is represented by a 1 dimensional global array containing the
diagonal values. The multiplication operations can be completed using the
functions \texttt{NGA\_Sprs\_array\_right\_multiply(int s\_a, int g\_d)} and
\texttt{NGA\_Sprs\_array\_left\_multiply(int s\_a, int g\_d)}. The handle
\texttt{g\_d} is the global array containing the diagonal elemenents.

The sparse matrix library also supports sparse matrix-sparse matrix
multiplication through the function \texttt{NGA\_Sprs\_array\_matmatm\_multiply(int
s\_a, int s\_b)}. This operation multiplies the sparse matrices \texttt{s\_a}
and \texttt{s\_b} together and returns a handle to a new sparse matrix
representing the product.

\section{Direct Access to Sparse Matrix Data}

GA provides basic functionality on sparse matrices that can be used to construct
more complex algorithms. In addition, an array of tools provides direct access
to the data stored in the sparse array enabling users to both access and modify
the sparse array in more complicated ways.\newline
The most basic operations for understanding data layout are the

\texttt{NGA\_Sprs\_array\_row\_distribution(int s\_a, int iproc, int *rlo, int
*rhi)}

and 

\texttt{NGA\_Sprs\_array\_column\_distribution(int s\_a, int iproc, int *clo, int
*chi)} 
\newline
functions. For the given processor \texttt{iproc} these function return
the bounding indices of the corresponding row or column partition. Data for all
rows between \texttt{rlo} and \texttt{rhi} are located on processor
\texttt{iproc}. On each processor, this data is further organized into CSR
formatted subblocks based in the column partition. Note that in many cases, the
subblocks may be empty. These distribution functions are also available with 64 bit
interfaces.


The \texttt{NGA\_Sprs\_array\_col\_block\_list(int s\_a, int **idx\_x, int *n)}
function returns a list of the column blocks that contain data on the calling
process. In C, the pointer \texttt{*idx} is allocated by this function and must
be freed by the calling program. In Fortran, the array \texttt{idx} is assumed
to have already been allocated and the variable \texttt{n} on input is the
length of \texttt{idx}. On output, \texttt{n} is the number of non-zero blocks.
\newline
The data inside the column blocks can be accessed directly using 

\texttt{NGA\_Sprs\_arrayt\_access\_col\_block(int s\_a, int icol, int **idx, int
**jdx, void **val)}. \newline
This function returns pointers to the column block data in
CSR format. The bounding indices for this data can be obtained using the
distribution functions. This function is also available in a 64 bit version. If
the column block corresponding to \texttt{iproc} is empty, the pointers returned
by the function are \texttt{NULL}. The
row indices \texttt{idx} array contains the starting location of the column
indices and values for all non-zero values in the row. The number of entries in
\texttt{idx} is \texttt{rhi-rlo+2}. The extra entry is included so that
the number of non-zero entries in any row can be calculated using
\texttt{idx[i+1]-idx[i]}, even for the last row in the block. The absolute row
index of the row indexed by \texttt{i} in \texttt{idx} can be calculated using
\texttt{i+rlo}. The array \texttt{jdx} contains the absolute column indices for
the entry. The relative column index in the block can be recovered using
\texttt{j=jdx[k]-clo}.

Any column block in the sparse array can be recovered using the \newline
\texttt{NGA\_Sprs\_array\_get\_block(int s\_a, int irow, int icol, int **idx,
int **idx, void *val, int *rlo, int *rhi, int *clo, int *chi)} function. This
function is also available in a 64 bit version. The indices \texttt{irow} and
\texttt{icol} are the row and column indices of the desired block. The values of
these indices are bounded by the number of processors. The CSR formatted data is
returned in the arrays \texttt{idx}, \texttt{jdx} and \texttt{val}. If the block
contains no data, these arrays are \texttt{NULL}. These arrays
are allocated by the function and must be freed by the calling program. The
indices \texttt{rlo,rhi,clo,chi} are the bounding indices for the block
in a 64 bit version. 

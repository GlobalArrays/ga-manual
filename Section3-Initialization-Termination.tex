For historical reasons (the 2-dimensional interface was developed
first), many operations have two interfaces, one for two dimensional
arrays and the other for arbitrary dimensional (one- to seven- dimensional,
to be more accurate) arrays. The latter can definitely handle two
dimensional arrays as well. The supported data types are integer,double
precision, and double complex. Global Arrays provide C and Fortran
interfaces in the same (mixed-language) program to the same array
objects. The underlying data layout is based on the Fortran convention.

GA programs require message-passing and Memory Allocator (MA) libraries
to work. Global Arrays is an extension to the message-passing interface.
GA internally does not allocate local memory from the operating system
- all dynamically allocated local memory comes from MA. We will describe
the details of memory allocation later in this section. 


\section{Message Passing }

The first version of Global Arrays was released in 1994 before robust
MPI implementations became available. At that time, GA worked only
with TCGMSG, a message-passing library that one of the GA authors
(Robert Harrison) had developed before. In 1995, support for MPI was
added. At the present time, the GA distribution still includes the
TCGMSG library for backward compatibility purposes, and because it
is small, fast to comple, and provides a minimal message-passing support
required by GA programs. The user can enable the MPI-compatible version
of GA by defining USE\_MPI environment variable before compiling the
GA toolkit. On systems where vendors provide MPI with interoperable
C and Fortran interfaces, there is no advantage in compiling or using
TCGMSG.

The GA toolkit needs the following functionality from any message-passing
library it runs with:
\begin{itemize}
\item initialization and termination of processes in an SPMD (single-program-multiple-data)
program, 
\item synchronization, 
\item functions that return number of processes and calling process id, 
\item broadcast, 
\item reduction operation for integer and double datatypes, and 
\item a function to abort the running parallel job in case of an error.
\end{itemize}
The message-passing library has to be initialized before the GA library
and terminated after the GA library is terminated.

GA provides two functions ga\_nnodesand ga\_nodeidthat return the
number of processes and the calling process id in a parallel program.
Starting with release 3.0, these functions return the same values
as their message-passing counterparts. In earlier releases of GA on
clusters of workstations, the mapping between GA and message-passing
process ids were nontrivial. In these cases, the ga\_list\_nodeidfunction
(now obsolete) was used to describe the actual mapping.

Although message-passing libraries offer their own barrier (global
synchronization) function, this operation does not wait for completion
of the outstanding GA communication operations. The GA toolkit offers
a ga\_syncoperation that can be used for synchronization, and it has
the desired effect of waiting for all the outstanding GA operations
to complete. 


\section{\label{sub:Memory-Allocation}Memory Allocation }

GA uses a very limited amount of statically allocated memory to maintain
its data structures and state. Most of the memory is allocated dynamically
as needed, primarily to store data in newly allocated global arrays
or as temporary buffers internally used in some operations, and deallocated
when the operation completes.

There are two flavors of dynamically allocated memory in GA: shared
memory and local memory. Shared memory is a special type of memory
allocated from the operating system (UNIX and Windows) that can be
shared between different user processes (MPI tasks). A process that
attaches to a shared memory segment can access it as if it was local
memory. All the data in shared memory is directly visible to every
process that attaches to that segment. On shared memory systems and
clusters of SMP (symmetritc multiprocessor) nodes, shared memory is
used to store global array data and is allocated by the Global Arrays
run-time system called ARMCI. ARMCI uses shared memory to optimize
performance and avoid explicit interprocessor communication within
a single shared memory system or an SMP node. ARMCI allocates shared
memory from the operating system in large segments and then manages
memory in each segment in response to the GA allocation and deallocation
calls. Each segment can hold data in many small global arrays. ARMCI
does not return shared memory segments to the operating system until
the program terminates (calls ga\_terminate).

On systems that do not offer shared-memory capabilities or when a
program is executed in a serial mode, GA uses local memory to store
data in global arrays.

All of the dynamically allocated local memory in GA comes from its
companion library, the Memory Allocator (MA) library. MA allocates
and manages local memory using stack and heap disciplines. Any buffer
allocated and deallocated by a GA operation that needs temporary buffer
space comes from the MA stack. Memory to store data in global arrays
comes fromheap. MA has additional features useful for program debugging
such as:
\begin{itemize}
\item left and right guards: they are stamps that detect if a memory segment
was overwritten by the application, 
\item named memory segments, and 
\item memory usage statistics for the entire program.
\end{itemize}
Explicit use of MA by the application to manage its non-GA local data
structures is not necessary but encouraged. Because MA is used implicitly
by GA, it has to be initialized before the first global array is allocated.
The MA\_init function requires users to specify memory for heap and
stack. This is because MA:
\begin{itemize}
\item allocates from the operating system only one segment equal in size
to the sum of heap and stack, 
\item manages both allocation schemes using memory coming from opposite
ends of the same segment, and 
\item the boundary between free stack and heap memory is dynamic.
\end{itemize}
It is not important what the stack and heap size argument values are
as long as the aggregate memory consumption by a program does not
exceed their sum at any given time. 


\subsection{Determining the Values of MA Stack and Heap Size}

How can I determine what the values of MA stack and heap size should
be? 

The answer to this question depends on the run-time environment of
the program including the availability of shared memory. A part of
GA initialization involves initialization of the ARMCI run-time library.
ARMCI dynamically determines if the program can use shared memory
based on the architecture type and current configuration of the SMP
cluster. For example, on uniprocessor nodes of the IBM SP shared memory
is not used whereas on the SP with SMP nodes it is. This decision
is made at run-time. GA reports the information about the type of
memory used with the function ga\_uses\_ma(). This function returns
false when shared memory is used and true when MA is used.

Based on this information, a programmer who cares about the efficient
usage of memory has to consider the amount of memory per single process
(MPI task) needed to store data in global arrays to set the heap size
argument value in ma\_init. The amount of stack space depends on the
GA operations used by the program (for example ga\_mulmat\_patch orga\_dgemmneed
several MB of buffer space to deliver good performance) but it probably
should not be less than 4MB. The stack space is only used when a GA
operaion is executing and it is returned to MA when it completes. 


\section{GA Initialization }

The GA library is initialized after a message-passing library and
before MA. It is possible to initialize GA after MA but it is not
recommended: GA must first be initialized to determine if it needs
shared or MA memory for storing distributed array data. There are
two alternative functions to initialize GA:
\begin{lyxcode}
\textcolor{blue}{Fortran}~subroutine~GA\_initialize\href{http://www.emsl.pnl.gov/docs/global/ga_ops.html\#ga_initialize}{GA\_{}initialize}()~

\textcolor{blue}{C}~~~~~~~void~\href{http://www.emsl.pnl.gov/docs/global/c_nga_ops.html\#ga_initialize}{GA\_{}Initialize}()~

\textcolor{blue}{C++}~~~~~void~GA::Initialize(int~argc,~char~{*}{*}argv)
\end{lyxcode}
and
\begin{lyxcode}
\textcolor{blue}{Fortran}~~subroutine~\href{http://www.emsl.pnl.gov/docs/global/ga_ops.html\#ga_initialize_ltd}{GA\_{}Initialize\_{}ltd}(limit)~~

\textcolor{blue}{C}~~~~~~~~void~\href{http://www.emsl.pnl.gov/docs/global/c_nga_ops.html\#ga_initialize_ltd}{GA\_{}Initialize\_{}ltd}(size\_t~limit)~

\textcolor{blue}{C++~}~~~~~void~GA::Initialize(int~argc,~char~{*}{*}argv,~size\_t~limit)
\end{lyxcode}
The first interface allows GA to consume as much memory as the application
needs to allocate new arrays. The latter call allows the programmer
to establish and enforce a limit within GA on the memory usage.

\emph{Note}: In GA++, there is an additional functionality as follows: 
\begin{lyxcode}
\textcolor{blue}{C++}~~~~~~void~GA::Initialize(int~argc,~char~{*}argv{[}{]},~

~~~~~~~~~unsigned~long~heapSize,~unsigned~long~stackSize,

~~~~~~~~~int~type,~size\_t~limit=0)~
\end{lyxcode}

\subsection{Limiting Memory Usage by Global Arrays }

GA offers an optional mechanism that allows a programmer to limit
the aggregate memory consumption used by GA for storing Global Array
data. These limits apply regardless of the type of memory used for
storing global array data.They do not apply to temporary buffer space
GA might need to use to execute any particular operation. The limits
are given per process (MPI task) in bytes. If the limit is set, GA
would not allocate more memory in global arrays that would exceed
the specified value - any calls to allocate new arrays that would
simply fail (return false). There are two ways to set the limit:
\begin{enumerate}
\item at initialization time by calling ga\_initialize\_ltd, or 
\item after initialization by calling the function\end{enumerate}
\begin{lyxcode}
\textcolor{blue}{Fortran}~~~subroutine~\href{http://www.emsl.pnl.gov/docs/global/ga_ops.html\#ga_set_memory_limit}{ga\_{}set\_{}memory\_{}limit}(limit)~

\textcolor{blue}{C}~~~~~~~~~void~\href{http://www.emsl.pnl.gov/docs/global/c_nga_ops.html\#ga_set_memory_limit}{GA\_{}Set\_{}memory\_{}limit}(size\_t~limit)~

\textcolor{blue}{C++}~~~~~~~void~GA::GAServices::setMemoryLimit(size\_t~limit)
\end{lyxcode}
It is encouraged that the user choose the first option, even though
the user can intialize the GA normally and set the memory limit later.

\emph{Example}: Initialization of MA and setting GA memory limits
\begin{lyxcode}
call~ga\_initialize()~

if~(ga\_uses\_ma())~then

~~~status~=~ma\_init(MT\_DBL,~stack,~heap+global)~

else~

status~=~ma\_init(mt\_dbl,stack,heap)~

call~ga\_set\_memory\_limit(ma\_sizeof(MT\_DBL,global,MT\_BYTE))~

endif~

if(.not.~status)~...~!we~got~an~error~condition~here
\end{lyxcode}
In this example, depending on the value returned from ga\_uses\_ma(),
we either increase the heap size argument by the amount of memory
for global arrays or set the limit explicitly through ga\_set\_memory\_limit().
When GA memory comes from MA we do not need to set this limit through
the GA interface since MA enforces its memory limits anyway. In both
cases, the maximum amount of memory acquired from the operating system
is capped by the value \emph{stack+heap+global}. 


\section{Termination }

The normal way to terminate a GA program is to call the function
\begin{lyxcode}
\textcolor{blue}{Fortran}~~~subroutine~\href{http://www.emsl.pnl.gov/docs/global/ga_ops.html\#ga_terminate}{ga\_{}terminate}()~

\textcolor{blue}{C~}~~~~~~~~void~\href{http://www.emsl.pnl.gov/docs/global/c_nga_ops.html\#ga_initialize}{GA\_{}Terminate}()~

\textcolor{blue}{C++}~~~~~~~void~GA::Terminate()
\end{lyxcode}
The programmer can also abort a running program for example as part
of handling a programmatically detected error condition by calling
the function
\begin{lyxcode}
\textcolor{blue}{Fortran}~~~subroutine~\href{http://www.emsl.pnl.gov/docs/global/ga_ops.html\#ga_error}{ga\_{}error}(message,~code)

\textcolor{blue}{C}~~~~~~~~~void~\href{http://www.emsl.pnl.gov/docs/global/c_nga_ops.html\#ga_error}{GA\_{}Error}(char~{*}message,~int~code)

\textcolor{blue}{C++}~~~~~~~void~GA::GAServices::error(char~{*}message,~int~code)
\end{lyxcode}

\section{Creating Arrays - I }

There are three ways to create new arrays:
\begin{enumerate}
\item From scratch, for regular distribution, using

\begin{lyxcode}
\textcolor{green}{n-d}~\textcolor{blue}{Fortran}~logical~function~\href{http://www.emsl.pnl.gov/docs/global/ga_ops.html\#ga_create}{nga\_{}create}(type,~ndim,~

~~~~~~~~~~~~~~~~~~~~~~~~dims,~array\_name,~chunk,~g\_a)~

\textcolor{green}{2-d}~\textcolor{blue}{Fortran}~logical~function~\href{http://www.emsl.pnl.gov/docs/global/ga_ops.html\#ga_create}{ga\_{}create}(type,~dim1,~

~~~~~~~~~~~~~~~~~~~~~~~~dim2,~array\_name,~chunk1,~chunk2,~g\_a)~

\textcolor{blue}{C}~~~~~~~~~~~int~\href{http://www.emsl.pnl.gov/docs/global/c_nga_ops.html\#ga_create}{NGA\_{}Create}(int~type,~int~ndim,~int~dims{[}{]},~

~~~~~~~~~~~~~~~~~~~~~~~~char~{*}array\_name,~int~chunk{[}{]})~

\textcolor{blue}{C++}~~~~~~~~~GA::GlobalArray{*}~GA::GAServices::createGA(int~type,~

~~~~~~~~~~~~~~~~~~~~~~~~int~ndim,~int~dims{[}{]},~char~{*}array\_name,~

~~~~~~~~~~~~~~~~~~~~~~~~int~chunk{[}{]})
\end{lyxcode}
or for regular distribution, using
\begin{lyxcode}
\textcolor{green}{n-d}~\textcolor{blue}{Fortran}~logical~function~\href{http://www.emsl.pnl.gov/docs/global/ga_ops.html\#ga_create_irreg}{nga\_{}create\_{}irreg}(type,~ndim,~dims,

~~~~~~~~~~~~~~~~~~~~~~~array\_name,~map,~nblock,~g\_a)~

\textcolor{green}{2-d}~\textcolor{blue}{Fortran}~logical~function~\href{http://www.emsl.pnl.gov/docs/global/ga_ops.html\#ga_create_irreg}{ga\_{}create\_{}irreg}(type,~dim1,~dim2,

~~~~~~~~~~~~~~~~~~~~~~~array\_name,~map1,~nblock1,~map2,~nblock2,~g\_a)~

C~~~~~~~~~~~int~\href{http://www.emsl.pnl.gov/docs/global/c_nga_ops.html\#ga_create_irreg}{NGA\_{}Create\_{}irreg}(int~type,~int~ndim,~int~dims{[}{]},~

C++~~~~~~~~~GA::GlobalArray{*}~GA::GAServices::createGA(int~type,~

~~~~~~~~~~~~~~~~~~~~~~~int~ndim,~int~dims{[}{]},~char~{*}array\_name,~

~~~~~~~~~~~~~~~~~~~~~~~int~map{[}{]},~int~block{[}{]})
\end{lyxcode}
\item Based on a template (an existing array) with the function

\begin{lyxcode}
\textcolor{blue}{Fortran}~logical~function~\href{http://www.emsl.pnl.gov/docs/global/ga_ops.html\#ga_duplicate}{ga\_{}duplicate}(g\_a,~g\_b,~array\_name)~

\textcolor{blue}{C}~~~~~~~int~\href{http://www.emsl.pnl.gov/docs/global/c_nga_ops.html\#ga_duplicate}{GA\_{}Duplicate}(int~g\_a,~char~{*}array\_name)~

\textcolor{blue}{C++}~~~~~int~\href{http://www.emsl.pnl.gov/docs/global/ga++/classGAServices.html\#a17}{GA::GAServices::duplicate}(int~g\_a,~char~{*}array\_name)~

-~or~-~

\textcolor{blue}{C++}~~~~~GA::GlobalArray{*}~GA::GAServices::createGA(int~g\_a,~char

~~~~~~~~~~~~~~~~~~~~~~~~~{*}array\_name)
\end{lyxcode}
\item Refer to the \textquotedbl{}Creating Arrays - II\textquotedbl{} section.
\end{enumerate}
In this case, the new array inherits all the properties such as distribution,
datatype and dimensions from the existing array.

With the regular distribution shown in Figure~\ref{cap:RegularDistribution},
the programmer can specify block size for none or any dimension. If
block size is not specified the library will create a distribution
that attempts to assign the same number of elements to each processor
(for static load balancing purposes). The actual algorithm used is
based on heuristics.

%
\begin{figure}
\begin{centering}
\includegraphics[width=0.9\columnwidth]{distr-1}
\par\end{centering}

\caption{\label{cap:RegularDistribution}Regular Distribution}

\end{figure}


With the irregular distribution shown in Figure~\ref{cap:IrregularDistribution},
the programmer specifies distribution points for every dimension using
map array argument. The library creates an array with the overall
distribution that is a Cartesian product of distributions for each
dimension. A specific example is given in the documentation.

%
\begin{figure}
\begin{centering}
\includegraphics[width=0.9\columnwidth]{distr-2}
\par\end{centering}

\caption{\label{cap:IrregularDistribution}Irregular Distribution}

\end{figure}


If an array cannot be created, for example due to memory shortages
or an enforced memory consumption limit, these calls return failure
status. Otherwise an integer handle is returned. This handle represents
a global array object in all operations involving that array. This
is the only piece of information the programmer needs to store for
that array. All the properties of the object (data type, distribution
data, name, number of dimensions and values for each dimension) can
be obtained from the library based on the handle at any time, see
Section 7.4. It is not necessary to keep track of this information
explicitly in the application code.

Note that regardless of the distribution type at most one block can
be owned/assigned to a processor. 


\subsection{Creating Arrays with Ghost Cells }

Individual processors ordinarily only hold the portion of global array
data that is represent by the lo and hi index arrays returned by a
call to nga\_distribution or that have been set using the nga\_create\_irreg
call. However, it is possible to create global arrays where this data
is padded by a boundary region of array elements representing portions
of the global array residing on other processors. These boundary regions
can be updated with data from neighboring processors by a call to
a single GA function. To create global arrays with these extra data
elements, referred to in the following as ghost cells, the user needs
to call either the functions:
\begin{lyxcode}
\textcolor{blue}{n-d}~Fortran~logical~function~\href{http://www.emsl.pnl.gov/docs/global/ga_ops.html\#ga_create_ghosts}{nga\_{}create\_{}ghosts}(type,~dims,~width,

~~~~~~~~~~~~~~~~~~~~~~~~~~~~array\_name,~chunk,~g\_a)

\textcolor{blue}{C}~~~~~~~~~~~int~int~\href{http://www.emsl.pnl.gov/docs/global/c_nga_ops.html\#nga_create_ghosts}{NGA\_{}Create\_{}ghosts}(int~type,~int~ndim,~int~dims{[}{]},

~~~~~~~~~~~~~~~~~~~~~~~~~~~~int~width{[}{]},~char~{*}array\_name,~int~chunk{[}{]})

\textcolor{blue}{C++}~~~~~~~~~int~GA::GAServices::createGA\_GhostsGA\_Ghosts(int~type,~int

~~~~~~~~~~~~~~~~~~~~~~~~~~~~ndim,~int~dims{[}{]},int~width{[}{]},~

~~~~~~~~~~~~~~~~~~~~~~~~~~~~char~~{*}array\_name,~int~chunk{[}{]})

\textcolor{blue}{n-d~Fortran}~logical~function~\href{http://www.emsl.pnl.gov/docs/global/ga_ops.html\#ga_create_ghosts_irreg}{nga\_{}create\_{}ghosts\_{}irreg}(type,~dims,~width,

~~~~~~~~~~~~~~~~~~~~~~~~~~~~array\_name,~map,~block,~g\_a)~

\textcolor{blue}{C}~~~~~~~~~~~int~int~\href{http://www.emsl.pnl.gov/docs/global/c_nga_ops.html\#nga_create_ghosts_irreg}{NGA\_{}Create\_{}ghosts\_{}irreg}(int~type,~int~ndim,~

~~~~~~~~~~~~~~~~~~~~~~~~~~~~int~dims{[}{]},~int~width{[}{]},~char~{*}array\_name,

~~~~~~~~~~~~~~~~~~~~~~~~~~~~int~map{[}{]},~int~block{[}{]})~

\textcolor{blue}{C++}~~~~~~~~~int~GA::GAServices::createGA\_Ghosts(int~type,~int~ndim,~

~~~~~~~~~~~~~~~~~~~~~~~~~~~~int~dims{[}{]},~int~width{[}{]},~char~{*}array\_name,

~~~~~~~~~~~~~~~~~~~~~~~~~~~~int~map{[}{]},~int~block{[}{]})~


\end{lyxcode}
These two functions are almost identical to the \texttt{nga\_create}
and \texttt{nga\_create\_irreg} functions described above. The only
difference is the parameter array width. This is used to control the
width of the ghost cell boundaries in each dimension of the global
array. Different dimensions can be padded with different numbers of
ghost cells, although it is expected that for most applications the
widths will be the same for all dimensions. If the width has been
set to zero for all dimensions, then these two functions are completely
equivalent to the functions \texttt{nga\_create} and \texttt{nga\_create\_irreg}. 

To illustrate the use of these functions, an ordinary global array
is shown in Figure~\ref{cap:OrdinaryGlobalArray}. The boundaries
represent the data that is held on each processor.

%
\begin{figure}
\begin{centering}
\includegraphics[width=0.9\columnwidth]{ghost003}
\par\end{centering}

\caption{\label{cap:OrdinaryGlobalArray}Ordinary Global Array}

\end{figure}


For a global array with ghost cells, the data distribution can be
visualized as shown in Figure~\ref{cap:GAwGhostCells}:

%
\begin{figure}
\begin{centering}
\includegraphics[width=0.9\columnwidth]{ghost006}
\par\end{centering}

\caption{\label{cap:GAwGhostCells} Global Array with Ghost Cells}

\end{figure}


Each processor holds \textquotedblleft{}visible\textquotedblright{}
data, corresponding to the data held on each processor of an ordinary
global array, and \textquotedblleft{}ghost cell\textquotedblright{}
data, corresponding to neighboring points in the global array that
would ordinarily be held on other processors. This data can be updated
in a single call to \texttt{nga\_update}, described under the collective
operations section of the user documentation. Note that the ghost
cell data duplicates some portion of the data in the visible portion
of the global array. The advantage of having the ghost cells is that
this data ordinarily resides on other processors and can only be retrieved
using additional calls. To access the data in the ghost cells, the
user must use the \texttt{nga\_access\_ghosts} function described
in Section 6.1. 


\section{Creating Arrays - II}

As mentioned in the previous section (\textquotedbl{}Creating arrays
- I\textquotedbl{}), there are 3 ways to create arrays. This section
describes method \#3 to create arrays. Because of the increasingly
varied ways that global arrays can be configured, a set of new interfaces
for creating global arrays has been created. This interface supports
all the configurations that were accessible via the old ga\_create\_XXX
calls, as well as new options that can only be accessed using the
new interface. Creating global arrays using the new interface starts
by a call to ga\_create\_handle that returns the user a new global
array handle. The user then calls several ga\_set\_XXX calls to assign
properties to this handle. These properties include the dimension
of the array, the data type, the size of the array, and any other
properties that may be relevant. At present, the available ga\_set\_XXX
calls largely reflect properties that are accessible via the nga\_create\_XXX
calls, however, it is anticipated that the range of properties that
can be set using these calls will expand considerably in the future.
After all the properties have been set, the user calls ga\_allocate
on the array handle and memory is allocated for the array. The array
can now be used in exactly the same way as arrays created using the
traditional ga\_create\_XXX calls. The calls for obtaining a new global
array handle are
\begin{lyxcode}
\textcolor{blue}{n-d~Fortran}~integer~function~\href{http://www.emsl.pnl.gov/docs/global/ga_ops.html\#NGA_CREATE_HANDLE}{ga\_{}create\_{}handle}()~

\textcolor{blue}{C}~~~~~~~~~~~int~\href{http://www.emsl.pnl.gov/docs/global/c_nga_ops.html\#GA_CREATE_HANDLE}{GA\_{}Create\_{}handle}()
\end{lyxcode}
Properties of the global arrays can be set using the ga\_set\_XXX
calls. Note that the only required call is to ga\_set\_data. The others
are all optional.
\begin{lyxcode}
\textcolor{blue}{n-d~Fortran}~subroutine~\href{http://www.emsl.pnl.gov/docs/global/ga_ops.html\#NGA_SET_DATA}{ga\_{}set\_{}data}(g\_a,~ndim,~dims,~type)~

\textcolor{blue}{C}~~~~~~~~~~~void~\href{http://www.emsl.pnl.gov/docs/global/c_nga_ops.html\#GA_SET_DATA}{GA\_{}Set\_{}data}(int~g\_a,~int~ndim,~int~{*}dims,~int~type)
\end{lyxcode}
The argument \emph{g\_a} is the global array handle, \emph{ndim} is
the dimension of the array, \emph{dims} is an array of \emph{ndim}
numbers containing the dimensions of the array, and \emph{type} is
the data type as defined in either the macdecls.h or mafdecls.h files.
Other options that can be set using these subroutines are:
\begin{lyxcode}
\textcolor{blue}{n-d~Fortran}~subroutine~\href{http://www.emsl.pnl.gov/docs/global/ga_ops.html\#NGA_SET_ARRAY_NAME}{ga\_{}set\_{}array\_{}name}(g\_a,~array\_name)~

\textcolor{blue}{C}~~~~~~~~~~~void~\href{http://www.emsl.pnl.gov/docs/global/c_nga_ops.html\#GA_SET_ARRAY_NAME}{GA\_{}Set\_{}array\_{}name}(int~g\_a,~char~{*}array\_name)
\end{lyxcode}
This subroutine assigns a character string as an array name to the
global array.
\begin{lyxcode}
\textcolor{blue}{n-d~Fortran}~subroutine~\href{http://www.emsl.pnl.gov/docs/global/ga_ops.html\#NGA_SET_CHUNK}{ga\_{}set\_{}chunk}(g\_a,~chunk)~

C~~~~~~~~~~~void~\href{http://www.emsl.pnl.gov/docs/global/c_nga_ops.html\#GA_SET_CHUNK}{GA\_{}Set\_{}chunk}(int~g\_a,~int~{*}chunk)
\end{lyxcode}
The chunk array contains the minimum size dimensions that should be
allocated to a single processor. If the minimum size is set to -1
for some of the dimensions, then the minimum size allocation is left
to the GA toolkit. The default setting of the chunk array is -1 along
all dimensions.
\begin{lyxcode}
\textcolor{blue}{n-d~Fortran}~subroutine~\href{http://www.emsl.pnl.gov/docs/global/ga_ops.html\#NGA_SET_IRREG_DISTR}{ga\_{}set\_{}irreg\_{}distr}(g\_a,~map,~block)~

\textcolor{blue}{C}~~~~~~~~~~~void~\href{http://www.emsl.pnl.gov/docs/global/c_nga_ops.html\#GA_SET_IRREG_DISTR}{GA\_{}Set\_{}irreg\_{}distr}(int~g\_a,~int~{*}map,~int~{*}block)
\end{lyxcode}
The ga\_set\_irreg\_distr subroutine can be used to specify the distribution
of data among processors. The block array contains the processor grid
used to lay out the global array and the map array contains a list
of the first indices of each block along each of the array axes. If
the first value in the block array is M, then the first M values in
the map array are the first indices of each data block along the first
axis in the processor grid. Similarly, if the second value in the
block array is N, then the values in the map array from M+1 to M+N
are the first indices of the each data block along the second axis
and so on through the D dimensions of the global array.
\begin{lyxcode}
\textcolor{blue}{n-d~Fortran}~subroutine~\href{http://www.emsl.pnl.gov/docs/global/ga_ops.html\#NGA_SET_GHOSTS}{ga\_{}set\_{}ghosts}(g\_a,~width)~

\textcolor{blue}{C}~~~~~~~~~~~void~\href{http://www.emsl.pnl.gov/docs/global/c_nga_ops.html\#GA_SET_GHOSTS}{GA\_{}Set\_{}ghosts}(int~g\_a,~int~{*}width)
\end{lyxcode}
This call can be used to set the ghost cell width along each of the
array dimensions.
\begin{lyxcode}
\textcolor{blue}{n-d~Fortran}~subroutine~\href{http://www.emsl.pnl.gov/docs/global/ga_ops.html\#GA_SET_PGROUP}{ga\_{}set\_{}pgroup}(g\_a,~p\_group)~

\textcolor{blue}{C}~~~~~~~~~~~void~\href{http://www.emsl.pnl.gov/docs/global/c_nga_ops.html\#GA_SET_PGROUP}{ga\_{}set\_{}pgroup}(int~g\_a,~int~p\_group)
\end{lyxcode}
This call assigns a processor group to the global array. If no processor
group is assigned to the global array, it is assumed that the global
array is created on the default processor group.

After all the array properties have been set, memory for the global
array is allocated by a call to ga\_allocate. After this call, the
global array is ready for use inside the parallel application.
\begin{lyxcode}
\textcolor{blue}{n-d~Fortran}~logical~function~\href{http://www.emsl.pnl.gov/docs/global/ga_ops.html\#GA_ALLOCATE}{ga\_{}allocate}(g\_a)~

\textcolor{blue}{C}~~~~~~~~~~~int~\href{http://www.emsl.pnl.gov/docs/global/c_nga_ops.html\#GA_ALLOCATE}{GA\_{}Allocate}(int~g\_a)
\end{lyxcode}
This function returns a logical variable that is true if the global
array was successfully allocated and false otherwise. 


\section{Destroying Arrays }

Global arrays can be destroyed by calling the function
\begin{lyxcode}
\textcolor{blue}{Fortran}~logical~\href{http://www.emsl.pnl.gov/docs/global/ga_ops.html\#ga_destroy}{ga\_{}destroy}(g\_a)~

\textcolor{blue}{C}~~~~~~~void~\href{http://www.emsl.pnl.gov/docs/global/c_nga_ops.html\#ga_destroy}{GA\_{}Destroy}(int~g\_a)~

\textcolor{blue}{C++}~~~~~void~GA::GlobalArray::destroy()
\end{lyxcode}
that takes as its argument a handle representing a valid global array.
It is a fatal error to call ga\_destroy with a handle pointing to
an invalid array.

All active global arrays are destroyed implicitly when the user calls
\texttt{ga\_terminate}.

\chapter{Writing, Building and Running GA Programs}

The GA build process has been improved by using the GNU autotools (autoconf,
automake, and libtool) as well as by combining all of the historic GA libraries
(\texttt{blacklinalg, armci, ma, pario}) into a single, monolithic
\texttt{libga}.  Details on configuring GA can be found by running
``\texttt{configure -{}-help}''. The following sections explain some of the
configure options a typical installation might require for configuring and
building \texttt{libga}, its test programs, and how packages can link to and
use GA. 

The web page \url{www.emsl.pnl.gov/docs/global/support.html} contains
information about using GA on different platforms. Please refer to this page
frequently for most recent updates and platform information.  Information on
building GA on older systems is available in Appendix C, Global Arrays on Older
Systems \ref{Appendix_C}.

\section{Platform and Library Dependencies}

The following platforms are supported by Global Arrays. 

\subsection{Supported Platforms}
\begin{itemize}
\item BlueGene/L 
\item BlueGene/P 
\item Cray XT 
\item Cray XE 
\item Fujitsu 
\item IBM SP
\item Linux Cluster with Ethernet, Myrinet, Infiniband, or Quadrics
\item MAC
\item NEC 
\item SGI Altix
\item Solaris
\item Windows (Cygwin) 
\end{itemize}

For most of the platforms, there are two versions available: 32-bit and 64-bit.
64-bit is preferred and will automatically be selected by the configure script
if the size of the C datatype \texttt{void{*}} is 8 bytes. 

To aid the development of fully portable applications, in 64-bit mode the
Fortran integer datatype is 64-bits. It is motivated by 1) the need of
applications to use very large data structures, and 2) Fortran INTEGER{*}8 not
being fully portable. The 64-bit representation of integer datatype is
accomplished by using the appropriate Fortran compiler flag. The configure
script will determine this flag as needed, but one can always override the
configure script by supplying the environment variable \texttt{FFLAG\_INT} e.g.
\texttt{FFLAG\_INT=-i8}. 

\textbf{Note}: configure almost always does the right thing, so overriding this
particular option is rarely needed. The best way to enforce the integer size of
your choosing is to use the configure options \texttt{-{}-enable-i4} or
\texttt{-{}-enable-i8}. These options will force the integer size to be 4 or 8
bytes in size, respectively. 

Because of limited interest in heterogeneous computing among known us GA users,
the Global Arrays library still does not support heterogeneous platforms. This
capability can be added if required by new applications. 

\subsection{Selection of the communication network for ARMCI}

Some cluster installations can be equipped with a high performance network
which offer instead, or in addition to, TCP/IP some special communication
protocol, for example GM on Myrinet network. To achieve high performance in
Global Arrays, \href{http://www.emsl.pnl.gov/docs/parsoft/armci}{ARMCI} must be
built to use these protocols in its implementation of one-sided communication.
Starting with GA 5.0, this is accomplished by passing an option to the
configure script. 

In addition, it might be necessary to provide a location for the header files
and library path corresponding to the location of software supporting the
appropriate protocol API. 

Our ability to automatically locate the required headers and libraries is
currently inadequate. Therefore, you will likely need to specify the optional
ARG pointing to the necessary directories and/or libraries.  Sockets is the
default ARMCI network if nothing else is specified to configure. Note that the
optional argument ARG takes a quoted string of any CPPFLAGS, LDFLAGS, or LIBS
necessary for locating the headers and libraries of the given ARMCI network. On
many systems it is simply necessary to specify the selected network: 
\begin{verbatim}
./configure --with-openib 
\end{verbatim}
On others, you may need to specify the path to the network's installation if it
is in a non-default location. The following will add
-I/path/to/portals/install/include and -L/path/to/portals/install/lib to the
CPPFLAGS and LDFLAGS if those directories are found to exist: 
\begin{verbatim}
./configure --with-portals="/path/to/portals/install"
\end{verbatim}
See section 2.3.1 for details of what you can pass as the quoted string to
configure options. 

\begin{tabular}{|c|c|>{\centering}p{3cm}|}
\hline 
Network & Protocol Name & Configure Option\tabularnewline
\hline
\hline 
IBM BG/L & BGML & -{}-with-bgml\tabularnewline
\hline 
Cray shmem & Cray shmem & -{}-with-cray-shmem\tabularnewline
\hline 
IBM BG/P & Deep Computing Message Framework & -{}-with-dcmf\tabularnewline
\hline 
IBM LAPI & LAPI & -{}-with-lapi\tabularnewline
\hline 
N/A & MPI Spawn - MPI-2 dynamic process management & -{}-with-mpi-spawn\tabularnewline
\hline 
Infiniband & OpenIB & -{}-with-openib\tabularnewline
\hline 
Cray XT & Portals & -{}-with-portals\tabularnewline
\hline 
Ethernet & TCP/IP & -{}-with-sockets (the default, so you don't need to specify this)\tabularnewline
\hline
\end{tabular}

\subsection{Selection of the message-passing library}

As explained in Section \ref{sec:Initialization-and-Termination}, GA works with
either MPI or TCGMSG message-passing libraries. That means GA applications can
use either of these interfaces. Selection of the message-passing library takes
place when GA is configured.  Since the TCGMSG library is small and compiles
fast, it is included with the GA distribution package but as of GA 5.0 it is no
longer built by default. For GA 5.0, MPI is the default message-passing
library.  There are three possible configurations for running GA with the
message-passing libraries:

\begin{enumerate}

\item \underbar{GA with MPI} (\emph{recommended}): directly with MPI. In this
mode, GA program should contain MPI initialization calls. Example:
\texttt{./conifigure}

\item \underbar{GA with TCGMSG-MPI} (MPI and TCGMSG emulation library):
TCGMSG-MPI implements functionality of TCGMSG using MPI.  In this mode, the
message passing library can be initialized using either TCGMSG PBEGIN(F) or
MPI\_Init. Example: \texttt{./configure -{}-with-mpi -{}-with-tcgmsg}

\item \underbar{GA with TCGMSG}: directly with TCGMSG.  In this mode, GA
program should contain TCGMSG initialization calls.  Example:\texttt{
./configure -{}-with-tcgmsg}

\end{enumerate}

For the MPI versions (1 and 2 above), the \texttt{-{}-with-mpi} configure
option can take parameters. If no parameters are specified, configure will
search for the MPI compilers. Using the MPI compilers is the recommended way to
build GA. If the MPI compilers are not found, configure will exit with an
error. The configure script will attempt to determine the underlying Fortran
77, C, and C++ compilers wrapped by the MPI compilers. This is necessary for
other configure tests such as determining compiler-specific optimization flags
or determining the Fortran 77 libraries needed when linking using the C++
linker.

If an argument is specified to \texttt{-{}-with-mpi}, then configure will no
longer use the MPI compilers. Instead, configure will attempt to located the
MPI headers and libraries. The locations of the headers and the locations and
names of the one or more MPI libraries can differ wildly. The argument to
\texttt{-{}-with-mpi} can be a quoted string of any install paths, include
paths, library paths, and/or libraries required for compiling and linking MPI
programs. See section 2.3.1 for details of the possible arguments. 

\subsection{Dependencies on other software}

In addition to the message-passing library, GA requires (internally):

\begin{itemize}

\item \href{http://www.emsl.pnl.gov/docs/parsoft/ma/MAapi.html}{MA} (Memory
Allocator), a library for managment of local memory; 

\item \href{http://www.emsl.pnl.gov/docs/parsoft/armci}{ARMCI}, a one-sided
communication library that GA uses as its run-time system; 

\end{itemize}

GA optionally can use external: 

\begin{itemize}

\item BLAS library is required for the eigensolver and \texttt{ga\_dgemm}
(a subset is included with GA, which is built into \texttt{libga.a});

\item LAPACK library is required for the eigensolver (a subset is included
with GA, which is built into \texttt{libga.a}); 

\end{itemize}

GA may also depend on other software depending on the functions being
used.

\begin{itemize}

\item GA eigensolver, ga\_diag, is a wrapper for the eigensolver from the
PEIGS library; (Please contact \href{mailto:fanngi@ornl.gov}{George Fann}about
PEIGS) 

\item SCALAPACK, PBBLAS, and BLACS libraries are required for \emph{ga\_lu\_solve,
ga\_cholesky, ga\_llt\_solve, ga\_spd\_invert, ga\_solve}. If these
libraries are not installed, the named operations will not be available. 

\end{itemize}

\section{Writing GA Programs}

C programs that use Global Arrays should include files \emph{'ga.h'} and
\emph{`macdecls.h'}. Fortran programs should include the files
\emph{`mafdecls.fh'} and \emph{`global.fh'}. Fortran source must be
preprocessed as a part of compilation.

The GA program should look like:
\begin{itemize}
\item When GA runs with MPI
\end{itemize}
\input{mpif.tex}
\input{mpic.tex}
\begin{itemize}
\item When GA runs with TCGMSG or TCGMSG-MPI
\end{itemize}
\input{tcgmsgf.tex}
\input{tcgmsgc.tex}
{*}{*}The \emph{ma\_init} call looks like:
\begin{verbatim}
status = ma_init(type, stack_size, heap_size)
\end{verbatim}
and it basically just goes to the OS and gets \emph{stack\_size+heap\_size}
elements of size \emph{type}. The amount of memory MA allocates need to be
sufficient for storing global arrays on some platforms. Please refer to section
\ref{sub:Memory-Allocation} for the details and information on more advanced
usage of MA in GA programs. 

MA is optional for C programs. You can replace GA's internal MA memory handling
with \texttt{malloc()} and \texttt{free()} by using the function
\texttt{GA\_Register\_stack\_memory()}.
\input{replace_ma.tex}

\section{Building GA}

GNU Autotools (autoconf, automake, and libtool) are used to help build the GA
library and example programs. GA follows the usual convention of:
\begin{verbatim}
./configure; make; make install 
\end{verbatim}
Before GA 5.0 the user was required to set a TARGET environment variable. This
is no longer required - the configure script will determine the TARGET for the
user. The configure script will also search for appropriate Fortran 77, C, and
C++ compilers. To override the compilers, set the F77, CC, and/or CXX
environment variables or specify them to configure:
\begin{verbatim}
./configure CC=gcc F77=ifort CFLAGS="-O2 -g -Wall"
\end{verbatim}
For the complete list of environment variables which configure recognizes, see
the output of running: 
\begin{verbatim}
./configure --help
\end{verbatim}

\subsection{Building and Running GA Test and Example Programs}

The GA distribution comes with a number of test and example programs located in
./global/testing and ./global/examples, respectively. To build these programs,
after running configure and make, run the additional make target: 
\begin{verbatim}
make checkprogs
\end{verbatim}
To run the GA test suite, you must tell the make program how to run parallel
programs. The following assumes either an interactive session on a queued
system or a workstation: 
\begin{verbatim}
make check MPIEXEC="mpirun -np 4"
\end{verbatim}
Of course, replace the value of MPIEXEC to the appropriate command for the MPI
implementation used to build GA. The test suite has not been tested with the
TCGMSG message-passing library's parallel.x invoker. 

\subsection{Configure Options which Take Arguments}

Certain configure options take arguments which help the configure script locate
the headers and libraries necessary for the particular software. For example,
when specifying the ARMCI network (see section 2.1.2), the location of the MPI
installation (see section 2.1.3), or specifying the location of other external
software such as BLAS, LAPACK, or ScaLAPACK. 

You can put almost anything into the quoted argument to these configure
options. For example,\texttt{ -I{*}, -L{*}, -l{*}, -Wl{*}, -WL{*}, {*}.a,
{*}.so} where the asterisk represents the usual arguments to those compiler and
linker flags or paths to static or shared libraries.  Here are some sample MPI
uses to illustrate our point:
\begin{verbatim}
--with-mpi="/usr/local"
--with-mpi="-I/usr/local/include -L/usr/local/lib -lmpi"
--with-mpi="-lmpichf90 -lmpich"
\end{verbatim}

\subsection{BLAS and LAPACK}

The GA distribution contains a subset of the BLAS and LAPACK routines necessary
for successfully linking GA programs. However, those routines are not
optimized. If optimized BLAS and LAPACK routines are available on your system,
we recommend using them. The configure script will automatically attempt to
locate external BLAS and LAPACK libraries. 

Correctly determining the size of the Fortran INTEGER used when compiling
external BLAS and LAPACK libraries is not automatic. Even on 64-bit platforms,
external BLAS libraries are often compiled using 4-byte Fortran INTEGERs. The
GA interface to the BLAS and LAPACK routines must match the Fortran INTEGER
size used in the external BLAS and LAPACK routines. There are three options to
configure: 
\begin{itemize}
\item \texttt{-{}-with-blas{[}=ARG{]}} is the default and will attempt to
detect the size of the INTEGER, but if it fails (and it often will since this
is no easy task), it will assume 4-byte INTEGERS. Automatic detection of the
INTEGER size may improve in the future.
\item \texttt{-{}-with-blas4{[}=ARG{]}} assumes 4-byte INTEGERs 
\item \texttt{-{}-with-blas8{[}=ARG{]}} assumes 8-byte INTEGERs 
\end{itemize}
If LAPACK is in a separate library, you may need to specify
\texttt{-{}-with-lapack=ARG} where ARG is the path to the LAPACK library. See
section 2.3.1 for details. 

\subsection{ScaLAPACK}

GA interface routines to ScaLAPACK are only available when GA is built with MPI
and ScaLAPACK. Before building GA, the user is required to configure
-{}-with-scalapack and pass the location of ScaLAPACK \& Co. libraries passed
as arguments to those configure options. See section 2.3.2 (Configure Options
which Take Arguments) for details. 

\subsection{GA C++ Bindings}

The configure script automatically determines the Fortran 77 libraries required
for linking a C++ application and places them in the FLIBS variable within the
generated Makefile. Building the C++ bindings is then as simple as specifying: 
\begin{verbatim}
./configure --enable-cxx
\end{verbatim}
Running make will then link the libga++.a library in addition to the libga.a
library. Both are then required for linking C++ GA applications, specifying
libga++.a first and then libga.a (typically as -lga++ -lga). 

\subsection{Disabling Fortran}

Fortran sources have typically been used by the GA and ARMCI distributions.
For GA 5.0 and beyond, Fortran sources have been deprecated in the ARMCI
distribution and are still used by default in the GA source.  Therefore, ARMCI
is free from Fortran dependencies while GA is not.  The GA dependencies can be
removed by specifying 
\begin{verbatim}
./configure --disable-f77
\end{verbatim}
Note that disabling Fortran 77 in GA is not well tested and doing so will also
likely disable the use of external Fortran 77 libraries such as Fortran-based
BLAS or ScaLAPACK. This also disables the use of the GA library in Fortran
applications since the MA sources will no longer be compiled with Fortran 77
support. Use this option with care and only if developing C and/or C++
applications exclusively. 

\subsection{Python Bindings}

GA 5.0 releases with Python bindings which were developed using the Cython
package (http://www.cython.org/). At a minimum, GA must be configured with
shared library support (which is disabled by default.) The configure script
will automatically search for a python interpreter in the PATH environment
variable, so make sure the appropriate Python interpreter can be found before
configuring. For example: 
\begin{verbatim}
./configure --enable-shared
\end{verbatim}
should be all that is required to enable Python bindings. The Python bindings
are not installed by default. You should run the following: 
\begin{verbatim}
make python
\end{verbatim}
in order to build and install the Python bindings. The python make target
depends on the install target (i.e. ``make install'') and will pass the
libga.so and ga.h library and header locations to the python setup.py
invocation. Optionally, you can navigate to the python source directory and
run: 
\begin{verbatim}
python setup.py build_ext
\end{verbatim}
or 
\begin{verbatim}
python setup.py build_ext --inplace
\end{verbatim}
to build the python bindings in a more manual way. The ``make python'' target
is not well tested since specifying dependent libraries can be a difficult
task. 

If GA was built with external BLAS and LAPACK, those libraries must be
specified when linking the Python shared library. Currently, users must edit
the setup.py script within the python source directory in order to add these
libraries to the standard distutils invocation of the linker. The BLAS and
LAPACK libraries on OSX are particularly difficult to find, so this is done
automatically for the user since the configure script will detect the vecLib
framework on OSX automatically.  Unforunately, at this time bulding the Python
bindings is often a manual process but those fluent in Python and Python's
\texttt{distutils} should have few problems editing the \texttt{setup.py}
script. 

\subsection{Writing and Building New GA Programs}

As of GA 5.0, the ability to place small single-file test programs into the
global/testing directory of the distribution is no longer supported. Instead,
you must install the GA headers and libraries using the ``make install''
target.

This will install the GA headers and libraries to the location specified by the
-{}-prefix configure option. If not specified, the default is
\texttt{/usr/local/include} and \texttt{/usr/local/lib}.  For our testing
purposes, we often install GA into the same location as the build. Recall, GA
can be configured from a separate build directory, keeping source and build
trees separate. For example, from the top-level GA source distribution: 
\begin{verbatim}
mkdir bld
cd bld
../configure --prefix=`pwd`
make
make install
\end{verbatim}
will configure, build, and install the GA headers and library into the separate
build directory ``bld''.

More specifically, you would find the GA library \texttt{libga} in
\texttt{./bld/lib} and the GA headers in \texttt{./bld/include}. For packages
using GA, you need to provide appropriate compiler and linker flags to indicate
the locations of the GA header files and libraries. 

\section{Running GA Programs}

Assume the GA program app.x had already been built. To run it,

\emph{Running on shared memory systems and clusters}: (i.e., network of
workstations/linux clusters)

If the app.x is built based on MPI, run the program the same way as any other
MPI programs. 

\emph{Example}: to run on four processes on clusters, use 
\begin{verbatim}
mpirun -np 4 app.x
\end{verbatim}
\emph{Example}: If you are using MPICH (or MPICH-like Implementations), and
mpirun requires a machinefile or hostfile, then run the GA program same as any
other MPI programs. \textit{The only change required is to make sure the
hostnames are specified in a consecutive manner in the machinefile}. Not doing
this will prevent SMP optimizations and would lead to poor resource
utilization.
\begin{verbatim}
mpirun -np 4 -machinefile machines.txt app.x
\end{verbatim}
\emph{Contents of machines.txt}: Let us say we have two 2-way SMP
nodes (host1 and host2, and correct formats for a 4-processor machinefile
is shown in the table below.

\begin{tabular}{|>{\centering}p{2cm}|>{\centering}p{2cm}|>{\raggedright}p{3cm}|}
\hline 
Correct & Correct & Incorrect\tabularnewline
\hline
\hline 
host1

host1

host2

host2 & host2

host2

host1

host1 & host1

host2

host1 (This is wrong, the same hosts should be specified together)

host2\tabularnewline
\hline
\end{tabular}

If app.x is built based on TCGMSG (not including Fujitsu, Cray J90, and
Windows, because there are no native ports of TCGMSG), to execute the program
on Unix workstations/servers, one should use the 'parallel.x' program. After
building the application, a file called 'app.x.p' would also be generated (If
there is not such a file, make it: 
\begin{verbatim}
make app.x.p
\end{verbatim}
This file can be edited to specify how many processors and tasks to use, and
how to load the executables. Make sure that 'parallel.x' is accessible (you
might copy it into your 'bin' directory). To execute, type:
\begin{verbatim}
parallel.x app.x
\end{verbatim}

\begin{enumerate}

\item On MPPs, such as Cray XT3/XT4, or IMB SPs, use the appropriate system
command to specify the number of processors, load and run the programs.
Example: 
\begin{itemize}
\item to run on IBM SP, run as any other parallel programs (i.e., using
\emph{poe}) 
\item to run on Cray XT3/XT4, use \emph{yod}.
\end{itemize}

\item On Microsoft NT, there is no support for TCGMSG, which means you can only
build your application based on MPI. Run the application program the same way
as any other MPI programs. For, WMPI you need to create the .pg file. Example:
\begin{verbatim}
R:\nt\g\global\testing> start /b test.x.exe
\end{verbatim}
\end{enumerate}

\section{Building Intel Trace Analyzer (VAMPIR) Instrumented Global Arrays}

Building GA for use with the intel trace analyzer is no loger supported. As of
GA 5.1 allows the GA functions to be intercepted by user code.

TODO TODO TODO describe new pnga\_ interfaces

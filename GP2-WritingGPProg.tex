\chapter{Writing GP Programs}

The GP library is automatically built when GA is built. To build GP, refer to
the GA documentation for the build process. Similarly, all the GP routines are
automatically included in the GA library, so linking to GA will allow users to
call GP as well. Note that at present, only the C interface is supported, so
only C and C++ programs can call the GP library.

\section{Writing GP Programs}

The GP library is built on top of the same ARMCI runtime as GA and uses much of
the GA functionality in its implementation. It is therefore necessary to
initialize GA before initializing GP. C and C++ programs that use Global Pointers should
include files `gp.h'. If the programs also call any GA routines then they
should inlude `ga.h' and `macdecls.h' as well. Users should refer
to the GA documentation on initializing programs with GA.

After initializing GA, the user should include initialize the GP library by
calling

\begin{verbatim}
      GP_Initialize();
\end{verbatim}

\noindent
This will start up the GP library and initialize all internal data structures.
Similarly, at the end of the program the user should call

\begin{verbatim}
      GP_Terminate();
\end{verbatim}

\noindent
before terminating the GA library and whatever runtime (MPI or TCGMSG) that is
being used with it. The call to GP\_Terminate will clean up the internal GP data
structures and free up resources that have been allocated by the library. All
other calls to the GP library should be bracketed between these two calls.

\section{Creating GP Arrays}

Creating a GP array is straightforward and follows the model for creating GAs
using the new, more flexible interface. The first step in creating a GP array is
to create a handle for it.

\begin{verbatim}
      int GP_Create_handle();
\end{verbatim}

\noindent
This is a global operation and returns the handle as a simple integer. In object
oriented terms, creating a GP handle is similar to creating an instance of a GP
array, all other operations on this array will refer to this handle. The
dimensions of the GP array can be set by calling the function
 
\begin{verbatim}
      void GP_Set_dimensions(int g_p, int ndim, int *dims);
\end{verbatim}

\noindent
Again, this is a global operation that determines the dimension of the GP array
as well as setting the extents of the array axes. The extents are passed in
through the integer array dims which has ndim entries. The variable g\_p is the
GP array handle. Note that unlike the corresponding call in GA (GA\_Set\_data),
there is no argument that controls data type. In the context of a GP array,
specifying the type of an element is meaningless.

Finally, to finish creating the GP array and allocating internal resources for
the array, it is necessary to call
 
\begin{verbatim}
      int GP_Allocate(int g_p);
\end{verbatim}

\noindent
This function returns true if the allocation is successful and false if the
allocation fails for some reason. These three functions represent the minimal
list of functions required to create a GP array. Other functions can be inserted
between creating the GP handle and allocating the GP array that can be used to
control how the GP distributes data among different processors. These are
described below.
 
The GP library uses a cartesian decomposition to divide arrays up among
different processors. This means that each array axis is divide up into some
number of blocks such that the product of the number of blocks along each
dimension is equal to the total number of processors.
If no information about how the user wants to distribute array is provided, the
GP library will allocate data to different processors using an internal
algorithm that tries to guarantee that the blocks on each processor  contain
roughly the same number of array elements. This distribution can be controlled
to a certain extent using one or the other of the following two directives.
The firsts sets a chunk array which can be used to control the maximum number of
blocks a particular axis can be divided up into.

\begin{verbatim}
      void GP_Set_chunk(int g_p, int *chunk);
\end{verbatim}

\noindent
The number of entries in the chunk array is equal to the number of array
dimensions. For each axis, the corresponding entry in the chunk array
represents the \emph{minimum} size that the block dimensions on that axis can be
set to. If the entry in the chunk array is set to 0 or less, then the internal
algorithm will be used to figure out the array decomposition along that axis.
The chunk array is often used to prevent the library from decomposing an array
along a particular axis so that all data along that axis is forced to reside on the
same processor. This can be done by setting the corresponding value in the chunk
array to equal the array dimension.

A second way to control the distribution of data is to allow the user to
completely specify how data is decomposed among processors using the set
irregular distribution method.
 
\begin{verbatim}
      void GP_Set_irreg_distr(int g_p, int *mapc, *nblocks);
\end{verbatim}

\noindent
The two arrays in this method are used to specify how data is decomposed along
each axis. The nblocks array contains ndim entries. Each entry must be greater
than zero and the product of all ndim entries must be equal to the total number
of processors. Each entry specifies how many intervals the corresponding axis is
divided into. The mapc array is used to specify exactly how each axis is
partitioned. The number of entries in this array is equal to the sum of the
entries in the nblocks array. The first nblocks[0] entries in mapc correspond to
the \emph{starting} index for the partitions along the first axis, the next
nblocks[1] entries correspond to the starting indices for the partitions along
the second axis (if there is one) and so on. The indices along each axis must be
monotonically increasing.

\chapter{Writing GP Programs and Creating GP Arrays}

The GP library is automatically built when GA is built. To build GP, refer to
the GA documentation for the build process. Similarly, all the GP routines are
automatically included in the GA library, so linking to GA will allow users to
call GP as well. Note that at present, only the C interface is supported, so
only C and C++ programs can call the GP library.

\section{Initializing the GP Library}

The GP library is built on top of the same ARMCI runtime as GA and uses much of
the GA functionality in its implementation. It is therefore necessary to
initialize GA before initializing GP. C and C++ programs that use Global Pointers should
include files `gp.h'. If the programs also call any GA routines then they
should inlude `ga.h' and `macdecls.h' as well. Users should refer
to the GA documentation on initializing programs with GA.

After initializing GA, the user should initialize the GP library by
calling

\begin{verbatim}
      GP_Initialize();
\end{verbatim}

\noindent
This will start up the GP library and initialize all internal data structures.
Similarly, at the end of the program the user should call

\begin{verbatim}
      GP_Terminate();
\end{verbatim}

\noindent
before terminating the GA library and whatever runtime (MPI or TCGMSG) that is
being used with it. The call to GP\_Terminate will clean up the internal GP data
structures and free up resources that have been allocated by the library. All
other calls to the GP library should be bracketed between these two calls.

\section{Creating GP Arrays}

Creating a GP array is straightforward and follows the model for creating GAs
using the new, more flexible interface. The first step in creating a GP array is
to create a handle for it.

\begin{verbatim}
      int GP_Create_handle();
\end{verbatim}

\noindent
This is a global operation and returns the handle as a simple integer. In object
oriented terms, creating a GP handle is similar to creating an instance of a GP
array, all other operations on this array will refer to this handle. The
dimensions of the GP array can be set by calling the function
 
\begin{verbatim}
      void GP_Set_dimensions(int g_p, int ndim, int *dims);
\end{verbatim}

\noindent
Again, this is a global operation that determines the dimension of the GP array
as well as setting the extents of the array axes. The extents are passed in
through the integer array dims which has ndim entries. The variable g\_p is the
GP array handle. Note that unlike the corresponding call in GA (GA\_Set\_data),
there is no argument that controls data type. In the context of a GP array,
specifying the type of an element is meaningless.

Finally, to finish creating the GP array and allocating internal resources for
the array, it is necessary to call
 
\begin{verbatim}
      int GP_Allocate(int g_p);
\end{verbatim}

\noindent
This function returns true if the allocation is successful and false if the
allocation fails for some reason. These three functions represent the minimal
list of functions required to create a GP array. Other functions can be inserted
between creating the GP handle and allocating the GP array that can be used to
control how the GP distributes data among different processors. These are
described below.
 
The GP library uses a cartesian decomposition to divide arrays up among
different processors. This means that each array axis is divide up into some
number of blocks such that the product of the number of blocks along each
dimension is equal to the total number of processors.
If no information about how the user wants to distribute array is provided, the
GP library will allocate data to different processors using an internal
algorithm that tries to guarantee that the blocks on each processor  contain
roughly the same number of array elements. This distribution can be controlled
to a certain extent using one or the other of the following two directives.
The firsts sets a chunk array which can be used to control the maximum number of
blocks a particular axis can be divided up into.

\begin{verbatim}
      void GP_Set_chunk(int g_p, int *chunk);
\end{verbatim}

\noindent
The number of entries in the chunk array is equal to the number of array
dimensions. For each axis, the corresponding entry in the chunk array
represents the \emph{minimum} size that the block dimensions on that axis can be
set to. If the entry in the chunk array is set to 0 or less, then the internal
algorithm will be used to figure out the array decomposition along that axis.
The chunk array is often used to prevent the library from decomposing an array
along a particular axis so that all data along that axis is forced to reside on the
same processor. This can be done by setting the corresponding value in the chunk
array to equal the array dimension.

A second way to control the distribution of data is to allow the user to
completely specify how data is decomposed among processors using the set
irregular distribution method.
 
\begin{verbatim}
      void GP_Set_irreg_distr(int g_p, int *mapc, *nblocks);
\end{verbatim}

\noindent
The two arrays in this method are used to specify how data is decomposed along
each axis. The nblocks array contains ndim entries. Each entry must be greater
than zero and the product of all ndim entries must be equal to the total number
of processors. Each entry specifies how many intervals the corresponding axis is
divided into. The mapc array is used to specify exactly how each axis is
partitioned. The number of entries in this array is equal to the sum of the
entries in the nblocks array. The first nblocks[0] entries in mapc correspond to
the \emph{starting} index for the partitions along the first axis, the next
nblocks[1] entries correspond to the starting indices for the partitions along
the second axis (if there is one) and so on. The indices along each axis must be
monotonically increasing.

\section{Assigning Data to Global Pointers}

After a GP array is allocated, data needs to be associated with the pointers
before it can be used. To do this, memory for each array element needs to be
allocated on the processor that contains the element and then that memory needs
to be associated with the array element. The first step is to determine which
set of array elements are owned by a process. This can be determined using the
function
 
\begin{verbatim}
      void GP_Distribution(int g_p, int iproc, int *lo, int *hi);
\end{verbatim}

\noindent
This function returns the low and high indices of the array block that is owned
by processor iproc. To determine which set of indices are locally owned, the
process ID should be used for iproc (this is the value returned by the
GA\_Nodeid() function).

Once the block of locally held elements is known, individual elements can be
associated with data. The first step is to allocate memory for a local element
using a specialized version of malloc

\begin{verbatim}
      void* GP_Malloc(size_t size);
\end{verbatim}

\noindent
and free

\begin{verbatim}
      void GP_Free(void *ptr);
\end{verbatim}

\noindent
Because this memory will be used in communication, it is necessary for the GP
library to have access to its network address. This is only determined if
GP\_Malloc is called, ordinary malloc will not work in this context. Similarly,
to make sure that the memory is properly cleaned up, it is necessary to call
GP\_Free and not the ordinary free. This only applies to data that will be
associated with a GP array element, other buffers can be created and destroyed
using conventional C calls.

After the memory segment has been created, it needs to be associated with an
array element. This can be done using the call

\begin{verbatim}
      void GP_Assign_local_element(int g_p, int *subscript,
                                   void *ptr, int size);
\end{verbatim}

\noindent
This function associates the array element at the location indicated by the
array subscript with the memory located at ptr. The size of the memory segment
(in bytes) is also stored in the GP array. The values of subscript must lie
within the ranges found using the GP\_Distribution function and ptr must be
allocated using a GP\_Malloc call. The size of the memory location in the
GP\_Assign\_local\_element call must match the value used in the corresponding
GP\_Malloc call. Once a memory segment has been associated with an element in
the GP array, it is no longer necessary to keep a local pointer to it. A pointer
to the memory can be recovered directly from the GP array.

In addition to associating memory locations with a GP array, it is also possible
to dissociate memory locations with a GP array. This can be done using the
function

\begin{verbatim}
      void* GP_Free_local_element(int g_p, int *subscript);
\end{verbatim}

\noindent
This function deletes the information about the memory location from the GP
array but does \emph{not} delete the memory location itself. The memory location
is returned as a void pointer that can be used in a call to GP\_Free.

Finally, GP arrays can be removed by calling

\begin{verbatim}
     int GP_Destroy(int g_p);
\end{verbatim}

\noindent
This function returns a logical value that is true if the GP is successfully
removed with no problems and false otherwise. Note that calling GP\_Destroy
\emph{only} removes the GP itself, it does not remove the memory segments
associated with each array element. These must be removed first by using calls to
GP\_Free\_local\_element and GP\_Free. A code fragment that completely cleans up
a hypothetical one-dimensional GP array would look like this

\begin{verbatim}
      GP_Distribution(g_p, me, lo, hi)
      for (i=lo; i<=hi; i++) {
        GP_Free(GP_Free_local_element(g_p, &i));
      }
      GP_Destroy(g_p)
\end{verbatim}

\noindent
It is not required to destroy a GP array after using GP\_Free\_local\_element and
GP\_Free. For example, users might want to free up some elements and reassign
them to other memory segments. This can be done without calling GP\_Destroy. It
is also not necessary to have data assigned to all array elements. These can be
considered as null pointers and can be identified by the fact that the size of
the corresponding data element is zero.

Because GP arrays are filled with arbitrary elements, conventional algebraic
operations are meaningless and it is not possible to define as many global
operations as in the case of the GA toolkit. Although the concept of setting a
GP element equal to zero has no meaning, it is possible to think of setting all
the bytes in a GP element to zero. The GP toolkit provides a function

\begin{verbatim}
      void GP_Memzero(int g_p)
\end{verbatim}

\noindent
that allows to users to simultaneously set every byte pointed to by the GP array
to zero. This can be done to reinitialize GP arrays. Note that this
function does not delete the memory. That can be done using the
GP\_Free\_local\_element and GP\_Free routines, as described above.

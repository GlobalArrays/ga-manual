\chapter{Interprocess Synchronization}

Global Arrays provide three types of synchronization calls to support
different synchronization styles. 

\begin{tabular}{|>{\centering}p{3cm}|>{\raggedright}p{6cm}|}
\hline 
\emph{Lock with mutex:}  & is useful for a shared memory model. One can lock a mutex, to exclusively
access a critical section. \tabularnewline
\hline 
\emph{Fence:}  & guarantees that the Global Array operations issued from the calling
process are complete. The fence operation is local. \tabularnewline
\hline 
\emph{Sync:}  & is a barrier. It synchronizes processes and ensures that all Global
Array operations are completed. Sync operation is collective. \tabularnewline
\hline
\end{tabular}


\section{Lock and Mutex }

Lock works together with mutex. It is a simple synchronization mechanism
used to protect a critical section.To enter a critical section, typically,
one needs to do:
\begin{verbatim}
1.~~Create~mutexes~

2.~~Lock~on~a~mutex~

3.~~...~

~~~~Do~the~exclusive~operation~in~the~critical~section~~

~~~~...~

4.~~Unlock~the~mutex~

5.~~Destroy~mutexes
\end{verbatim}
The function
\begin{verbatim}
\textcolor{blue}{Fortran}~logical~function~\href{http://www.emsl.pnl.gov/docs/global/ga_ops.html\#ga_create_mutex}{ga\_{}create\_{}mutexes}(number)~

\textcolor{blue}{C}~~~~~~~int~\href{http://www.emsl.pnl.gov/docs/global/c_nga_ops.html\#ga_create_mutexes}{GA\_{}Create\_{}mutexes}(int~number)~

\textcolor{blue}{C++~}~~~~int~GA::GAServices::createMutexes(int~number)
\end{verbatim}
creates a set containing the number of mutexes. Only one set of mutexes
can exist at a time. Mutexes can be created and destroyed as many
times as needed. Mutexes are numbered: 0, ..., number-1.

The function
\begin{verbatim}
\textcolor{blue}{Fortran}~logical~function~\href{http://www.emsl.pnl.gov/docs/global/ga_ops.html\#ga_destroy_mutex}{ga\_{}destroy\_{}mutexes}()~

\textcolor{blue}{C}~~~~~~~int~\href{http://www.emsl.pnl.gov/docs/global/c_nga_ops.html\#ga_destroy_mutexes}{GA\_{}Destroy\_{}mutexes}()

\textcolor{blue}{C++}~~~~~int~GA::GAServices::destroyMutexes()
\end{verbatim}
destroys the set of mutexes created with ga\_create\_mutexes.

Both \texttt{\emph{ga\_create\_mutexes}} and \texttt{\emph{ga\_destroy\_mutexes}}
are collective operations.

The functions
\begin{verbatim}
\textcolor{blue}{Fortran}~subroutine~\href{http://www.emsl.pnl.gov/docs/global/ga_ops.html\#ga_lock}{ga\_{}lock}(int~mutex)~

~~~~~~~~subroutine~\href{http://www.emsl.pnl.gov/docs/global/ga_ops.html\#ga_unlock}{ga\_{}unlock}(int~mutex)~

\textcolor{blue}{C}~~~~~~~void~\href{http://www.emsl.pnl.gov/docs/global/c_nga_ops.html\#ga_lock}{GA\_{}lock}(int~mutex)~

~~~~~~~~void~\href{http://www.emsl.pnl.gov/docs/global/c_nga_ops.html\#ga_unlock}{GA\_{}unlock}(int~mutex)~

\textcolor{blue}{C++}~~~~~void~GA::GAServices::lock(int~mutex)~

~~~~~~~~void~GA::GAServices::unlock(int~mutex)
\end{verbatim}
lock and unlock a mutex object identified by the mutex number, respectively.
It is a fatal error for a process to attempt to lock a mutex which
has already been locked by this process, or unlock a mutex which has
not been locked by this process.

\emph{\underbar{Example}}\underbar{ }\emph{\underbar{1}}\underbar{:}

Use one mutex and the lock mechanism to enter the critical section.
\begin{verbatim}
status~=~ga\_create\_mutexes(1)~

if(.not.status)~then~

~~~call~ga\_error('ga\_create\_mutexes~failed~',0)~

endif~

call~ga\_lock(0)

~~~...~do~something~in~the~critical~section~

call~ga\_put(g\_a,~...)~

~~~...

call~ga\_unlock(0)~

if(.not.ga\_destroy\_mutexes())~then~

~~~call~ga\_error('mutex~not~destroyed',0)~
\end{verbatim}

\section{Fence }

Fence blocks the calling process until all the data transfers corresponding
to the Global Array operations initiated by this process complete.
The typical scenario that it is being used is
\begin{verbatim}
1.~~Initialize~the~fence~

2.~~...~

~~~~~~~~Global~array~operations~

~~~~...~

3.~~Fence
\end{verbatim}
This would guarantee the operations between step 1 and 3 are complete.

The function
\begin{verbatim}
\textcolor{blue}{Fortran}~subroutine~\href{http://www.emsl.pnl.gov/docs/global/ga_ops.html\#ga_init_fence}{ga\_{}init\_{}fence}()~

\textcolor{blue}{C}~~~~~~~void~\href{http://www.emsl.pnl.gov/docs/global/c_nga_ops.html\#ga_init_fence}{GA\_{}Init\_{}fence}()~

\textcolor{blue}{C++}~~~~~void~GA::GAServices::initFence()
\end{verbatim}
Initializes tracing of completion status of data movement operations.

The function
\begin{verbatim}
\textcolor{blue}{Fortran}~subroutine~\href{http://www.emsl.pnl.gov/docs/global/ga_ops.html\#ga_fence}{ga\_{}fence}()~

\textcolor{blue}{C}\textcolor{blue}{\underbar{~}}~~~~~~void~\href{http://www.emsl.pnl.gov/docs/global/c_nga_ops.html\#ga_fence}{GA\_{}Fence}()~

\textcolor{blue}{C++}~~~~~void~GA::GAServices::fence()
\end{verbatim}
blocks the calling process until all the data transfers corresponding
to GA operations called after \texttt{\emph{ga\_init\_fence}} complete.

\texttt{\emph{ga\_fence}} must be called after \texttt{\emph{ga\_init\_fence}}.
A barrier, \texttt{\emph{ga\_sync}}, assures completion of all data
transfers and implicitly cancels outstanding ga\_init\_fence. \texttt{\emph{ga\_init\_fence}}
and \texttt{\emph{ga\_fence}} must be used in pairs, multiple calls
to \texttt{\emph{ga\_fence}} require the same number of corresponding
\texttt{\emph{ga\_init\_fence}} calls. \texttt{\emph{ga\_init\_fence/ga\_fence}}
pairs can be nested.

\textit{\underbar{Example 1:}}

Since \texttt{\emph{ga\_put}} might return before the data reaches
the final destination \texttt{\emph{ga\_init\_fence}} and \texttt{\emph{ga\_fence}}
allow the process to wait until the data is actually moved:
\begin{verbatim}
call~ga\_init\_fence()~

call~ga\_put(g\_a,~...)~

call~ga\_fence()
\end{verbatim}
\textit{\underbar{Example 2:}}

\texttt{\emph{ga\_fence}} works for multiple GA operations.
\begin{verbatim}
call~ga\_init\_fence()~

call~ga\_put(g\_a,~...)~

call~ga\_scatter(g\_a,~...)~

call~ga\_put(g\_b,~...)~

call~ga\_fence()
\end{verbatim}
The calling process will be blocked until data movements initiated
by two calls to \texttt{ga\_put} and one \texttt{ga\_scatter} complete. 


\section{Sync }

Sync is a collective operation. It acts as a barrier, which synchronizes
all the processes and ensures that all the Global Array operations
are complete at the call.

The function is
\begin{verbatim}
\textcolor{blue}{Fortran}~subroutine~\href{http://www.emsl.pnl.gov/docs/global/ga_ops.html\#ga_sync}{ga\_{}sync}()~

\textcolor{blue}{C}~~~~~~~void~\href{http://www.emsl.pnl.gov/docs/global/c_nga_ops.html\#ga_sync}{GA\_{}Sync}()~

\textcolor{blue}{C++}~~~~~void~GA::GAServices::sync()
\end{verbatim}
Sync should be inserted as necessary. With many sync calls, the application
performance would suffer. 
